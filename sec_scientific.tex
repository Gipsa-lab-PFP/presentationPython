%_______________________________________________________________________________
\section{Description des Paquets Scientifiques}
%_______________________________________________________________________________
%_______________________________________________________________________________
\begin{frame}[fragile]
\frametitle{Quelques Paquets et Outils Scientifiques}
\begin{itemize}
 \item SciPy : Scientific Python 
 \begin{itemize}
  \item Numpy
  \item SciPy library
  \item Matplotlib
  \item Pandas
  \item IPython -> Jupyter notebook
 \end{itemize}
 \item Scikit-learn : machine learning.
 \item Mayavi : objets 3D avancés
 \item \dots
\end{itemize}
\end{frame}
%_______________________________________________________________________________
%_______________________________________________________________________________
\subsection{SciPy}
%_______________________________________________________________________________
%_______________________________________________________________________________
\begin{frame}[fragile]
\frametitle{SciPy}
\framesubtitle{}
\begin{itemize}
 \item \url{http://www.scipy.org/}
\end{itemize}
\begin{center}
\includegraphics[width=9cm]{./fig/scipy.png}
\end{center}
\end{frame}
%_______________________________________________________________________________
%_______________________________________________________________________________
\subsection{Numpy}
%_______________________________________________________________________________
%_______________________________________________________________________________
\begin{frame}[fragile]
\frametitle{Numpy}
\begin{itemize}
 \item \myFig{height=0.5cm}{./fig/numpy-logo.png} \, \url{http://www.numpy.org}
 \item \emph{NumPy is the fundamental package for scientific computing with Python}
\end{itemize}
\begin{pythonConsole}
>>> import numpy
>>> help(numpy)

Help on package numpy:

NAME
    numpy

DESCRIPTION
    NumPy
    =====
    
    Provides
      1. An array object of arbitrary homogeneous items
      2. Fast mathematical operations over arrays
      3. Linear Algebra, Fourier Transforms, Random Number Generation
...
\end{pythonConsole}
\end{frame}
%_______________________________________________________________________________
%_______________________________________________________________________________
%\subsubsection{ndarray}
%_______________________________________________________________________________
%_______________________________________________________________________________
\begin{frame}[fragile]
\frametitle{N-dimensional array Object}
\framesubtitle{}
\begin{itemize}
 \item N-dimensional array : ndarray
 \item Création d'un tableau vide et réservation de l'espace (empty)
 \item Accès aux éléments : A[i, j, \dots] 
 \item indices de {\color{red}{0 à $n-1$}}
 \item indices négatifs de $-n$ à $-1$
\end{itemize}
\begin{pythonConsole}

>>> A = numpy.empty((2, 2))
>>> print(A)
[[ -1.28822975e-231   2.68678092e+154]
 [  2.24497156e-314   2.24499315e-314]]
>>> A[0, 0] = 1
>>> A[1, 0] = 2
>>> A[0, 1] = 11
>>> A[1, 1] = 12
>>> print(A)
[[  1.   2.]
 [ 11.  12.]]
>>> type(A)
<class 'numpy.ndarray'>
>>> A[-1, -2] = 11
\end{pythonConsole}
\end{frame}
%_______________________________________________________________________________
%_______________________________________________________________________________
\begin{frame}[fragile]
\frametitle{N-dimensional array Object}
\framesubtitle{}
\begin{itemize}
 \item N-dimensional array : ndarray
 \item Création d'un tableau vide et réservation de l'espace (empty)
 \item Accès aux éléments : A[i, j, \dots] 
 \item indices de {\color{red}{0 à $n-1$}}
 \item indices négatifs de $-n$ à $-1$
\end{itemize}
\begin{pythonConsole}

>>> A = numpy.empty((2, 2))
>>> print(A)
[[ -1.28822975e-231   2.68678092e+154]
 [  2.24497156e-314   2.24499315e-314]]
>>> A[0, 0] = 1
>>> A[1, 0] = 2
>>> A[0, 1] = 11
>>> A[1, 1] = 12
>>> print(A)
[[  1.   2.]
 [ 11.  12.]]
>>> type(A)
<class 'numpy.ndarray'>
>>> A[-1, -2] = 11
\end{pythonConsole}
\end{frame}
%_______________________________________________________________________________
%_______________________________________________________________________________
\begin{frame}[fragile]
\frametitle{N-dimensional array Object}
\framesubtitle{}
\begin{itemize}
 \item Rappel : accès aux propiétés et méthodes (dir) 
\end{itemize}
\begin{pythonConsole}

>>> dir(A)
['T', ..., 'all', 'any', 'argmax', 'argmin', 'argpartition', 'argsort', 
'astype', 'base', 'byteswap', 'choose', 'clip', 'compress', 'conj', 'conjugate',
'copy', 'ctypes', 'cumprod', 'cumsum', 'data', 'diagonal', 'dot', 'dtype',
'dump', 'dumps', 'fill', 'flags', 'flat', 'flatten', 'getfield', 'imag', 'item',
'itemset', 'itemsize', 'max', 'mean', 'min', 'nbytes', 'ndim', 'newbyteorder',
'nonzero', 'partition', 'prod', 'ptp', 'put', 'ravel', 'real', 'repeat',
'reshape', 'resize', 'round', 'searchsorted', 'setfield', 'setflags', 'shape',
'size', 'sort', 'squeeze', 'std', 'strides', 'sum', 'swapaxes', 'take',
'tobytes', 'tofile', 'tolist', 'tostring', 'trace', 'transpose', 'var', 'view']
\end{pythonConsole}
\end{frame}
%_______________________________________________________________________________
%_______________________________________________________________________________
\begin{frame}[fragile]
\frametitle{N-dimensional array Object}
\framesubtitle{Attributs sur la forme du tableau}
\begin{itemize}
 \item Forme du tableau (shape), c'est un tuple.  
 \item Nombre de dimension (ndim)
 \item Type des éléments (dtype)
 \item Taille du tableau (size), c'est le nombre de cellules totales. 
\end{itemize}
\begin{pythonConsole}
>>> A.shape
(2, 2)
>>> (nRow, nCol) = A.shape
>>> nRow = A.shape[0]
>>> nCol = A.shape[1]
>>> A.ndim
2
>>> A.dtype
dtype('float64')
>>> A.size
4
\end{pythonConsole}
\end{frame}
%_______________________________________________________________________________
%_______________________________________________________________________________
\begin{frame}[fragile]
\frametitle{N-dimensional array Object}
\framesubtitle{Changement de forme}
\begin{itemize}
 \item Pour changer la forme (reshape)
 \item Transposition (T)
\end{itemize}
\begin{pythonConsole}
>>> B = A.reshape((4, 1))
array([[  1.],
       [  2.],
       [ 11.],
       [ 12.]])
>>> B.ndim
2
>>> B.size
4
>>> B.T
array([[ 1.,   2.,  11.,  12.]])
\end{pythonConsole}
\end{frame}
%_______________________________________________________________________________
%_______________________________________________________________________________
\begin{frame}[fragile]
\frametitle{N-dimensional array Object}
\framesubtitle{Vues et copies de tableaux}
\begin{itemize}
 \item Les éléments de B sont les mêmes que ceux de A, seule la forme change. On a une vue de l'objet (view).   
 \item Si on veut une copie : méthode copy(), numpy.copy().
\end{itemize}
\begin{pythonConsole}
>>> B[0, 0] = 21
>>> print(A[0, 0], B[0, 0])
21.0 21.0
>>> B[1, 0]
>>> C = A.copy()
>>> C[0, 0] = 31
>>> print(A[0,0], C[0,0])
(21.0, 31.0)
\end{pythonConsole}
\end{frame}
%_______________________________________________________________________________
%_______________________________________________________________________________
\begin{frame}[fragile]
\frametitle{N-dimensional array Object}
\framesubtitle{Création de tableaux}
\begin{itemize}
 \item tableau vide et réservation de l'espace (empty).
 \item initialisation à zeros (zeros) ou avec des uns (ones).
 \item tableau identité (eye) avec la dimension.
 \item à partir de listes (array) ou suivant une étendue (arange).
\end{itemize}
\begin{pythonConsole}
>>> A = numpy.zeros((2, 4))
>>> print(A)
[[ 0.  0.  0.  0.]
 [ 0.  0.  0.  0.]]
>>> A = numpy.ones((3, 2))
>>> print(A)
[[ 1.  1.]
 [ 1.  1.]
 [ 1.  1.]]
>>> A = numpy.eye(2)
>>> print(A)
[[ 1.  0.]
 [ 0.  1.]]
>>> A = numpy.array([[1, 2], [11, 12]])
>>> print(A)
[[ 1  2]
 [11 12]]
>>> print(numpy.arange(0.5, 1.7, 0.1))
[ 0.5  0.6  0.7  0.8  0.9  1.   1.1  1.2  1.3  1.4  1.5  1.6]
\end{pythonConsole}
\end{frame}
%_______________________________________________________________________________
%_______________________________________________________________________________
\begin{frame}[fragile]
\frametitle{N-dimensional array Object}
\framesubtitle{Types}
\begin{itemize}
 \item Définition du type à la création
 \item Changement de type (astype)
 \item Multiplication ou addition avec un scalaire typé. 
\end{itemize}
\begin{pythonConsole}
>>> A = numpy.array([[1, 2], [11, 12]])
>>> print(A.dtype)
int64
>>> A = numpy.array([[1., 2], [11, 12]])
>>> print(A.dtype)
float64
>>> A = numpy.array([[1, 2], [11, 12]], dtype="float")
>>> print(A.dtype)
float64
>>> A = A.astype("complex")
>>> print(A)
[[  1.+0.j   2.+0.j]
 [ 11.+0.j  12.+0.j]]
>>> A = numpy.array([[1, 2], [11, 12]]) * 1.
>>> print(A.dtype)
float64
\end{pythonConsole}
\end{frame}
%_______________________________________________________________________________
%_______________________________________________________________________________
\begin{frame}[fragile]
\frametitle{N-dimensional array Object}
\framesubtitle{Additions, soustractions, multiplications sur les tableaux}
\begin{itemize}
 \item Addition, soustraction de tableaux ou d'un scalaire (+, -)
 \item Multiplication par un scalaire (*)
 \item Produit élément par élément (*) 
\end{itemize}
\begin{pythonConsole}
>>> A = numpy.array([[1, 2], [11, 12]])
>>> B = numpy.array([[3, 4], [13, 14]])
>>> print(A + 10)
[[ 11.  12.]
 [ 21.  22.]]
>>> print(A + B)
[[  4.   6.]
 [ 24.  26.]]
>>> print(A * 10)
[[  10.   20.]
 [ 110.  120.]]
>>> print(A * B)
[[   3.    8.]
 [ 143.  168.]]
>>> C = numpy.ones((10, ))
>>> print(A * C)
Traceback (most recent call last):
  File £"£<stdin>£"£, line 1, in <module>
ValueError: operands could not be broadcast together with shapes (2,2) (10) 
\end{pythonConsole}
\end{frame}
%_______________________________________________________________________________
%_______________________________________________________________________________
\begin{frame}[fragile]
\frametitle{N-dimensional array Object}
\framesubtitle{Produit scalaire}
\begin{itemize}
 \item Produit scalaire (dot)
 \item See also numpy.dot : en général, pour chaque méthode associée à un ndarray, il existe une fonction équivalente dans numpy. 
\end{itemize}
\begin{pythonConsole}
>>> A = numpy.array([[1, 2], [11, 12]])
>>> B = numpy.array([[3, 4], [13, 14]])
>>> print(A.dot(B))
[[  29.   32.]
 [ 189.  212.]]
>>> print(numpy.dot(A, B))
[[  29.   32.]
 [ 189.  212.]]
>>> C = numpy.ones((10, ))
>>> print(A.dot(C))
Traceback (most recent call last):
  File £"£<stdin>£"£, line 1, in <module>
ValueError: matrices are not aligned
>>> 
\end{pythonConsole}
\end{frame}
%_______________________________________________________________________________
%_______________________________________________________________________________
\begin{frame}[fragile]
\frametitle{N-dimensional array Object}
\framesubtitle{Division}
\begin{itemize}
 \item Division par un scalaire (/)
 \item Division éléments par éléments (/) 
 \item Attention au type en Python 2.7 ! 
\end{itemize}
\begin{pythonConsole}
>>> A = numpy.array([[1, 2], [11, 12]])
>>> B = numpy.array([[3, 4], [13, 14]])
>>> print(A / 2)
python 3.x                      python 2.x
[[ 0.5  1. ]				    [[0 1]
 [ 5.5  6. ]] 				     [5 6]]
>>> print(A / B)
python 3.x                      python 2.x  
[[ 0.33333333  0.5       ]      [[0 0]
 [ 0.84615385  0.85714286]]      [0 0]]
>>> print(A / B.astype("float"))
[[ 0.33333333  0.5       ]
 [ 0.84615385  0.85714286]]
\end{pythonConsole}
\end{frame}
%_______________________________________________________________________________
%_______________________________________________________________________________
\begin{frame}[fragile]
\frametitle{N-dimensional array Object}
\framesubtitle{Autres méthodes}
\begin{itemize}
 \item max, min, sum, mean, std, cumsum, cumprod \dots sur tous les éléments ou sur une dimension particulière (kwarg axis). 
\end{itemize}
\begin{minipage}{5cm}
\begin{pythonConsole}
>>> A = numpy.ones((2, 3, 4))
>>> print(A)
[[[ 1.  1.  1.  1.]
  [ 1.  1.  1.  1.]
  [ 1.  1.  1.  1.]]

 [[ 1.  1.  1.  1.]
  [ 1.  1.  1.  1.]
  [ 1.  1.  1.  1.]]]          
>>> print(A.cumsum())
[  1.   2.   3.   4.   5.   6.   7.   8.   9.  10.  11.  12.  13.  14.  15. 16.  17.  18.  19.  20.  21.  22.  23.  24.]
>>> print(A.cumsum(axis=0))
[[[ 1.  1.  1.  1.]
  [ 1.  1.  1.  1.]
  [ 1.  1.  1.  1.]]

 [[ 2.  2.  2.  2.]
  [ 2.  2.  2.  2.]
  [ 2.  2.  2.  2.]]]
\end{pythonConsole}
\end{minipage}
\begin{minipage}{5cm}
\begin{pythonConsole}
>>> print(A.cumsum(axis=1))
[[[ 1.  1.  1.  1.]
  [ 2.  2.  2.  2.]
  [ 3.  3.  3.  3.]]

 [[ 1.  1.  1.  1.]
  [ 2.  2.  2.  2.]
  [ 3.  3.  3.  3.]]]
>>> print(A.cumsum(2))
[[[ 1.  2.  3.  4.]
  [ 1.  2.  3.  4.]
  [ 1.  2.  3.  4.]]

 [[ 1.  2.  3.  4.]
  [ 1.  2.  3.  4.]
  [ 1.  2.  3.  4.]]]
\end{pythonConsole}
\end{minipage}
\end{frame}
%_______________________________________________________________________________
%_______________________________________________________________________________
\begin{frame}[fragile]
\frametitle{N-dimensional array Object}
\framesubtitle{Fonctions universelles et constantes}
\begin{itemize}
 \item Fonction universelle (universal function, ufunc) : fonction qui opère sur tous les éléments du tableau, un par un. 
 \item Quelques constantes : pi, e, euler\_gamma, inf, nan.
\end{itemize}
\begin{pythonConsole}

>>> pi = numpy.pi
>>> A = numpy.deg2rad(numpy.array([[30 / num, 60], [45, 90]]))
>>> B = numpy.cos(A)
>>> print(B)
[[  8.66025404e-01   5.00000000e-01]
 [  7.07106781e-01   6.12323400e-17]]
>>> B[1, 1] = numpy.nan
>>> print(numpy.isfinite(B))
[[ True  True]
 [ True False]]
\end{pythonConsole}
\end{frame}
%_______________________________________________________________________________
%_______________________________________________________________________________
\begin{frame}[fragile]
\frametitle{N-dimensional array Object}
\framesubtitle{Sélection de sous-tableaux}
\begin{itemize}
 \item découpage, slicing, comme pour les séquences.
 \item ellipse (ellipsis, \dots)
\end{itemize}
\begin{pythonConsole}
>>> A = numpy.array([[1, 2, 3, 4], [11, 12, 13, 14]])
>>> print(A)
[[ 1  2  3  4]
 [11 12 13 14]]
>>> print(A[1, :])
[11 12 13 14]
>>> print(A[:, 1:3])
[[ 2  3]
 [12 13]]
>>> A = numpy.arange(24).reshape((2, 3, 4))
>>> print(A[0, ...])
[[ 0  1  2  3]
 [ 4  5  6  7]
 [ 8  9 10 11]]
\end{pythonConsole}
\end{frame}
%_______________________________________________________________________________
%_______________________________________________________________________________
\begin{frame}[fragile]
\frametitle{N-dimensional array Object}
\framesubtitle{Sélection de sous-tableaux}
\begin{itemize}
 \item Comparaison et opérateurs logiques
 \item Opérations logiques pour sélectionner des éléments (voir aussi advanced indexing)
 \item Récupérer les indices (where)
\end{itemize}
\begin{pythonConsole}
>>> A = numpy.array([[1, 2, 3, 4], [11, 12, 13, 14]])
>>> print(A)
[[ 1  2  3  4]
 [11 12 13 14]]
>>> B = A > 2
>>> print(B)
[[False False  True  True]
 [ True  True  True  True]]
>>> print(A[B])
[ 3  4 11 12 13 14]
>>> indices = numpy.where(B)
>>> print(indices[0])
array([0, 0, 1, 1, 1, 1])
>>> print(indices[1])
array([2, 3, 0, 1, 2, 3])
>>> (i, j) = numpy.where(A > 2)
\end{pythonConsole}
\end{frame}
%_______________________________________________________________________________
%_______________________________________________________________________________
\begin{frame}[fragile]
\frametitle{N-dimensional array Object}
\framesubtitle{Concaténations}
\begin{itemize}
 \item Concaténation verticale (vstack, r\_[])
 \item Concaténation horizontale (hstack, c\_[])
 \item Concatenate (concatenate, kwarg axis)
\end{itemize}
\begin{pythonConsole}
>>> A = numpy.array([[1, 2], [11, 12]])
>>> B = numpy.array([[3, 4], [13, 14]])
>>> C = numpy.vstack((A, B)) # r\_[A, B]
>>> print(C)
[[ 1  2]
 [11 12]
 [ 3  4]
 [13 14]]
>>> D = numpy.hstack((A, B, A, A)) # c\_[A, B, A, A]
>>> print(D)
[[ 1  2  3  4  1  2  1  2]
 [11 12 13 14 11 12 11 12]]
>>> E = numpy.concatenate((A, B, A, A), axis=1)
>>> print(E)
[[ 1  2  3  4  1  2  1  2]
 [11 12 13 14 11 12 11 12]]
\end{pythonConsole}
\end{frame}
%_______________________________________________________________________________
%_______________________________________________________________________________
\begin{frame}[fragile]
\frametitle{N-dimensional array Object}
\framesubtitle{Structured Arrays}
\begin{itemize}
 \item Possibilité de mettre des éléments de types différents.  
 \item Possibilité de tableaux structurés \dots
\end{itemize}
\begin{pythonConsole}
>>> A = numpy.array([["a", 1], ["b", 2]], dtype="object")
>>> print(A)
[['a' 1]
 ['b' 2]]
>>> print(A.dtype)
object

>>> A = numpy.array([(1, "abc"), (2, "def")], dtype=[("index", "int"), 
	("name", "S8")])                       
>>> print(A)              
[(1, b'abc') (2, b'def')]
>>> A["index"]
array([1, 2])    
>>> A["name"]                   
array([b'abc', b'def'], 
      dtype='|S8')
\end{pythonConsole}
\end{frame}
%_______________________________________________________________________________
%_______________________________________________________________________________
%\subsubsection{save/load}
%_______________________________________________________________________________
%_______________________________________________________________________________
\begin{frame}[fragile]
\frametitle{Sauvegarde et lecture de données}
\framesubtitle{}
\begin{itemize}
 \item Enregistrement d'un tableau (save) dans un fichier ".npy"
 \item Enregistrement compressé de plusieurs tableaux (savez) au format ".npz"
 \item Lecture (load) des fichiers ".npy", ".npz"
\end{itemize}
\begin{pythonConsole}
>>> A = numpy.array([[1, 2], [11, 12]])
>>> numpy.save("save_A", A)
>>> del(A)
>>> A = numpy.load("save_A.npy")
>>> print(A)
[[ 1  2]
 [11 12]]
>>> A = numpy.array([[1, 2], [11, 12]])
>>> B = numpy.array([[21, 22], [31, 32]])
>>> numpy.savez("save_AB", tab1=A, B=B)
>>> del(A, B)
>>> data = numpy.load("save_AB.npz")
>>> print(data["tab1"])
[[ 1  2]
 [11 12]]
>>> print(data["B"])
[[21 22]
 [31 32]]
\end{pythonConsole}
\end{frame}
%_______________________________________________________________________________
%_______________________________________________________________________________
\begin{frame}[fragile]
\frametitle{Sauvegarde et lecture de données txt}
\framesubtitle{}
\begin{itemize}
 \item Lecture de fichier texte ".txt" (load ou loadtxt) 
 \item Enregistrement (savetxt) 
\end{itemize}
\begin{pythonConsole}
>>> A = numpy.loadtxt("data.txt")
>>> A
array([[  1.,   2.,   3.,   4.,   5.],
       [ 11.,  12.,  13.,  14.,  15.],
       [ 21.,  22.,  23.,  24.,  25.]])
>>> numpy.savetxt("data.txt", A)
>>> numpy.savetxt("data.csv", A, delimiter=",", header="A, B, C, D, E")
\end{pythonConsole}
\end{frame}
%_______________________________________________________________________________
%_______________________________________________________________________________
%\subsubsection{matrix}
%_______________________________________________________________________________
%_______________________________________________________________________________
\begin{frame}[fragile]
\frametitle{Matrix}
\framesubtitle{Définition}
\begin{itemize}
 \item Classe héritée de ndarray avec ndim = 2.  
\end{itemize}
\begin{pythonConsole}
>>> help(numpy.matrix)
class matrix(numpy.ndarray)         
 |  matrix(data, dtype=None, copy=True)
 |  
 |  Returns a matrix £from£ an array-like £object£, £or from£ a string of data.
 |  A matrix £is£ a specialized 2-D array that retains its 2-D nature
 |  through operations.  It has certain special operators, such as £`££`£*£`££`£
 |  (matrix multiplication) £and£ £`££`£**£`££`£ (matrix power).
...
\end{pythonConsole}
\end{frame}
%_______________________________________________________________________________
%_______________________________________________________________________________
\begin{frame}[fragile]
\frametitle{Matrix}
\framesubtitle{Saisie}
\begin{itemize}
 \item Saisie directe de type ndarray avec des listes imbriquées. 
 \item Possibilité de saisie type Matlab.
\end{itemize}
\begin{pythonConsole}
>>> A = numpy.matrix([[1, 2], [11, 12]])
>>> print(A)
[[ 1  2]
 [11 12]]
>>> type(A)
 <class 'numpy.matrixlib.defmatrix.matrix'>
>>> A = numpy.matrix("[1, 2, 3, 4; 11, 12, 13, 14]")
>>> print(A)
[[ 1  2  3  4]
 [11 12 13 14]]
\end{pythonConsole}
\end{frame}
%_______________________________________________________________________________
%_______________________________________________________________________________
\begin{frame}[fragile]
\frametitle{Matrix}
\framesubtitle{Multiplication et exposant}
\begin{itemize}
 \item Produit de matrices (*)
 \item Exposant de matrice (**)
\end{itemize}
\begin{pythonConsole}
>>> A = numpy.matrix([[1, 2], [11, 12]])
>>> B = numpy.matrix([[3, 4], [13, 14]])
>>> print(A * B)
[[ 29  32]
 [189 212]]
>>> print(A ** 2)
[[ 23  26]
 [143 166]]
\end{pythonConsole}
\end{frame}
%_______________________________________________________________________________
%_______________________________________________________________________________
\begin{frame}[fragile]
\frametitle{Matrix}
\framesubtitle{Opérateurs matriciels courants}
\begin{itemize}
 \item Transposition (T)
 \item Inversion (I)
 \item Opérateur Hermitien (H)
\end{itemize}
\begin{pythonConsole}
>>> A = numpy.matrix([[1, 2], [11, 12]])
>>> print(A)
[[ 1  2]
 [11 12]]
>>> print(A.T)
[[ 1 11]
 [ 2 12]]
>>> print(A.I)
print(A.I)
[[-1.2  0.2]
 [ 1.1 -0.1]]
>>> B = numpy.matrix([[1, 2+1j], [11+1j, 12]])
>>> print(B)
[[  1.+0.j   2.+1.j]
 [ 11.+1.j  12.+0.j]]
>>> print(B.H)
[[  1.-0.j  11.-1.j]
 [  2.-1.j  12.-0.j]]
\end{pythonConsole}
\end{frame}
%_______________________________________________________________________________
%_______________________________________________________________________________
%\subsubsection{subpackages}
%_______________________________________________________________________________
%_______________________________________________________________________________
\begin{frame}[fragile]
\frametitle{Autres opérations d'algèbre linéaire}
\framesubtitle{Sous paquet linalg}
\begin{itemize}
 \item Interface vers des bibliothèques d'algèbre linéaire (numpy.linalg). 
\end{itemize}
\begin{pythonConsole}
>>> help(numpy.linalg)
...
    Linear algebra basics:
    - norm            Vector £or£ matrix norm
    - inv             Inverse of a square matrix
    - solve           Solve a linear system of equations
    - det             Determinant of a square matrix
    - lstsq           Solve linear least-squares problem
    - pinv            Pseudo-inverse (Moore-Penrose)...
    - matrix_power    Integer power of a square matrix
    Eigenvalues £and£ decompositions:
    - eig             Eigenvalues £and£ vectors of a square matrix
    - eigh            Eigenvalues £and£ eigenvectors of a Hermitian matrix
    - eigvals         Eigenvalues of a square matrix
    - eigvalsh        Eigenvalues of a Hermitian matrix
    - qr              QR decomposition of a matrix
    - svd             Singular value decomposition of a matrix
    - cholesky        Cholesky decomposition of a matrix
    Tensor operations:
    - tensorsolve     Solve a linear tensor equation
    - tensorinv       Calculate an inverse of a tensor
...
\end{pythonConsole}
\end{frame}
%_______________________________________________________________________________
%_______________________________________________________________________________
\begin{frame}[fragile]
\frametitle{Autres paquets de numpy}
\framesubtitle{}
\begin{pythonConsole}
>>> help(numpy)
...
    doc
        Topical documentation on broadcasting, indexing, etc.
    lib
        Basic functions used by several sub-packages.
    random
        Core Random Tools
    linalg
        Core Linear Algebra Tools
    fft
        Core FFT routines
    polynomial
        Polynomial tools
    testing
        Numpy testing tools
    f2py
        Fortran to Python Interface Generator.
    distutils
        Enhancements to distutils with support for
        Fortran compilers support and more.
...
\end{pythonConsole}
\end{frame}
%_______________________________________________________________________________
%_______________________________________________________________________________
\begin{frame}[fragile]
\frametitle{Autres paquets de numpy}
\framesubtitle{Random}
\begin{itemize}
 \item Sous paquet random : générateurs de nombres aléatoires. 
\end{itemize}
\begin{pythonConsole}
>>> numpy.random.seed(0)
>>> A = numpy.random.randn(2, 3, 4)
>>> print(A)
[[[ 1.76405235  0.40015721  0.97873798  2.2408932 ]
  [ 1.86755799 -0.97727788  0.95008842 -0.15135721]
  [-0.10321885  0.4105985   0.14404357  1.45427351]]

 [[ 0.76103773  0.12167502  0.44386323  0.33367433]
  [ 1.49407907 -0.20515826  0.3130677  -0.85409574]
  [-2.55298982  0.6536186   0.8644362  -0.74216502]]]
\end{pythonConsole}
\end{frame}
%_______________________________________________________________________________
%_______________________________________________________________________________
\subsection{SciPy library}
%\subsubsection{subpackages}
%_______________________________________________________________________________
%_______________________________________________________________________________
\begin{frame}[fragile]
\frametitle{Scipy library}
\framesubtitle{Librairie scientifique}
\begin{itemize}
 \item \myFig{height=0.5cm}{./fig/scipylib-logo.png} \, \url{http://www.scipy.org/scipylib/index.html}
 \item \emph{It provides many user-friendly and efficient numerical routines such as routines for numerical integration and optimization.}
\end{itemize}
\begin{pythonConsole}
>>> import scipy
>>> help(scipy)
	...
     cluster                      --- Vector Quantization / Kmeans
     fftpack                      --- Discrete Fourier Transform algorithms
     integrate                    --- Integration routines
     interpolate                  --- Interpolation Tools
     io                           --- Data input and output
     linalg                       --- Linear algebra routines
     linalg.blas                  --- Wrappers to BLAS library
     linalg.lapack                --- Wrappers to LAPACK library
     misc                         --- Various utilities that don£'£t have
                                      another home.
     ndimage                      --- n-dimensional image package
     odr                          --- Orthogonal Distance Regression
     optimize                     --- Optimization Tools
     signal                       --- Signal Processing Tools
    ...
\end{pythonConsole}
\end{frame}
%_______________________________________________________________________________
%_______________________________________________________________________________
\begin{frame}[fragile]
\frametitle{Scipy}
\framesubtitle{}
\begin{itemize}
 \item Optimisation, Intégration, Interpolation, Algèbre linéaire, Algèbre linaire creuse, Signal, Image, Statistiques, Fonctions spéciales ($\Gamma$, $\psi$)\dots
\end{itemize}
\begin{pythonConsole}
    ...
     sparse                       --- Sparse Matrices
     sparse.linalg                --- Sparse Linear Algebra
     sparse.linalg.dsolve         --- Linear Solvers
     sparse.linalg.dsolve.umfpack --- :Interface to the UMFPACK library:
                                      Conjugate Gradient Method (LOBPCG)
     sparse.linalg.eigen          --- Sparse Eigenvalue Solvers
     sparse.linalg.eigen.lobpcg   --- Locally Optimal Block Preconditioned
                                      Conjugate Gradient Method (LOBPCG)
     spatial                      --- Spatial data structures and algorithms
     special                      --- Special functions
     stats                        --- Statistical Functions
    ...
\end{pythonConsole}
\end{frame}
%_______________________________________________________________________________
%_______________________________________________________________________________
%\subsubsection{load Matlab data}
%_______________________________________________________________________________
%_______________________________________________________________________________
\begin{frame}[fragile]
\frametitle{Scipy}
\framesubtitle{lecture de fichiers Matlab}
\begin{itemize}
 \item Exemple, lectures de fichiers Matlab dans le subpackage io (scipy.io)
\end{itemize}

\begin{pythonConsole}
>>> import scipy.io
>>> data = scipy.io.loadmat('file.mat')
>>> data
{'__header__': b'MATLAB 5.0 MAT-file, Platform: MACI64, 
	Created on: Tue Nov 14 14:40:01 2017', '__version__': '1.0', 
    'B': array([[ 1,  2],
       [11, 21]], dtype=uint8), 'A': array([[ 1,  2,  3],
       [11, 12, 13]], dtype=uint8), '__globals__': []}
>>> data['A']
array([[ 1,  2,  3],
       [11, 12, 13]], dtype=uint8)

\end{pythonConsole}
\end{frame}
%_______________________________________________________________________________
%_______________________________________________________________________________
\subsection{Matplotlib}
%_______________________________________________________________________________
%_______________________________________________________________________________
\begin{frame}[fragile]
\frametitle{Matplotlib}
\framesubtitle{}
\begin{itemize}
 \item \myFig{height=0.5cm}{./fig/matplotlib-logo.png} : \url{http://www.matplotlib.org}
 \item \emph{matplotlib is a python 2D plotting library which produces publication quality figures in a variety of hardcopy formats and interactive environments across platforms. }
 \item Contient des classes : programmation orientée objets avec différents backends pour différentes interfaces graphiques, graphical user interfaces (GUI) : agg, gtk, qt, svg, ps, pdf\dots
 \item Contient des procédures pour faciliter l'accès à ces classes . 
\end{itemize}
\end{frame}
%_______________________________________________________________________________
%_______________________________________________________________________________
%\subsubsection{pyplot}
%_______________________________________________________________________________
%_______________________________________________________________________________
\begin{frame}[fragile]
\frametitle{Matplotlib}
\framesubtitle{pyplot}
\begin{itemize}
 \item Fonctions procédurales dans le sous-paquet pyplot (matplotlib.pyplot)
\end{itemize}
\begin{pythonConsole}
>>> import matplotlib.pyplot
>>> t = numpy.arange(0, 10, 0.01)
>>> x = numpy.sin(2 * numpy.pi * 3 * t)
>>> matplotlib.pyplot.plot(t, x)
>>> matplotlib.pyplot.xlabel("time (s)")
>>> matplotlib.pyplot.ylabel("x ($\mu V$)")
>>> matplotlib.pyplot.show()
\end{pythonConsole}
\begin{center}
 \includegraphics[width=4.5cm]{./fig/matplotlibSinus.pdf}
\end{center}
\end{frame}
%_______________________________________________________________________________
%_______________________________________________________________________________
%\subsubsection{savefig}
%_______________________________________________________________________________
%_______________________________________________________________________________
\begin{frame}[fragile]
\frametitle{Matplotlib}
\framesubtitle{Saving figures}
\begin{itemize}
 \item Sauvegarde manuelle à partir de la fenêtre ouverte sur l'icône save.  
 \item sauvegarde en ligne de commande (savefig) sans passage par un affichage à l'écran.  
\end{itemize}
\begin{pythonConsole}
...
>>> matplotlib.pyplot.ylabel("x ($\mu V$)")
>>> # matplolib.pyplot.show()
>>> matplotlib.pyplot.savefig("./sinus.ps")
>>> matplotlib.pyplot.savefig("./sinus.pdf")
>>> matplotlib.pyplot.savefig("./sinus.svg")
>>> matplotlib.pyplot.savefig("./sinus.tiff")
>>> matplotlib.pyplot.savefig("./sinus.png")
>>> matplotlib.pyplot.savefig("./sinus.jpg")
\end{pythonConsole}
\end{frame}
%_______________________________________________________________________________
%_______________________________________________________________________________
%\subsubsection{pylab}
%_______________________________________________________________________________
%_______________________________________________________________________________
\begin{frame}[fragile]
\frametitle{Matplotlib}
\framesubtitle{Pylab}
\begin{itemize}
 \item Fonctions à la Matlab dans le sous-paquet pylab (matplotlib.pylab)
 \item Beaucoup de fonctions ($\approx 973$) sous formes abrégées \dots
\end{itemize}
\begin{minipage}[c]{5cm}
\begin{pythonConsole}
>>> import matplotlib.pylab as pylab
>>> len(dir(pylab))
973

>>> from pylab import *
>>> t = arange(0, 10, 0.01)
>>> f = arange(0, 5, 0.005)
>>> x = sin(2 * pi * f * t)
>>> subplot(2, 1, 1)
>>> plot(t, x * 5)
>>> plot(t, f)
>>> subplot(2, 1, 2)
>>> res = specgram(x, NFFT=64, 
... Fs=100, noverlap=8)
>>> show()
\end{pythonConsole}
\end{minipage}
\begin{minipage}[c]{5cm}
 \includegraphics[width=5cm]{./fig/specgram.png}
\end{minipage}
\end{frame}
%_______________________________________________________________________________
%_______________________________________________________________________________
\begin{frame}[fragile]
\frametitle{Matplotlib}
\framesubtitle{Pylab}
\begin{minipage}{5cm}
\begin{pythonConsole}
>>> (X, Y) = meshgrid(linspace(-2, 2, \
... 500), linspace(-2, 2, 500))
>>> Z = X + Y * 1j
>>> for k in range(76): 
...  Z -= (Z / 3 - 1) / (3 * Z ** 2)
>>> close("all")	
>>> imshow(angle(Z))
>>> # savefig('/MonChemin/Lenom.pdf')
>>> show()
\end{pythonConsole}
\end{minipage}
\begin{minipage}{5cm}
\includegraphics[width=5cm]{fig/fractal.png}
\end{minipage}
\end{frame}
%_______________________________________________________________________________
%_______________________________________________________________________________
\begin{frame}[fragile]
\frametitle{Matplotlib}
\framesubtitle{Exemples en 3D}
% \framesubtitle{Exemples de Nicolas Le Bihan}
%\begin{minipage}{0.4\linewidth}
%\begin{figure}
%\includegraphics[width=6.5cm,height=5.5cm]{fig/BrownSphere.png}
%\end{figure}
%\end{minipage}
%\hspace{1cm}
% \includegraphics[width=5.5cm,height=4.5cm]{fig/Distrib.png}
\begin{minipage}{5.2cm}
\begin{pythonConsole}
>>> import matplotlib.pyplot as plt
>>> from mpl_toolkits.mplot3d import \ 
... Axes3D
>>> fig = plt.figure()
>>> ax = fig.add_subplot(111, \ 
...	projection='3d')
>>> X = numpy.random.randn(100, 3)
>>> ax.scatter(X[:,0], X[:,1], X[:,2])
>>> ax.plot([X[0,0], X[0,0]+1], \ 
...	[X[0,1], X[0,1]+1], [X[0,2], \
... X[0,2]+1])
>>> ax.text(X[0,0]+1.2, X[0,1]+1.2, \ 
...	X[0,2]+1.2, 'you are here', \
... fontsize=20)
>>> pl.savefig('plot3D.pdf')
\end{pythonConsole}
\end{minipage}
\begin{minipage}{5.cm}
\begin{center}\includegraphics[width=6.5cm]{fig/plot3D.pdf}\end{center}
\end{minipage}
\end{frame}
%_______________________________________________________________________________
%_______________________________________________________________________________
\subsection{Pandas}
%_______________________________________________________________________________
%_______________________________________________________________________________
\begin{frame}[fragile]
\frametitle{Pandas}
\framesubtitle{}
\begin{itemize}
 \item \myFig{height=0.5cm}{./fig/pandas-logo.png} \, \url{http://www.scipy.org/scipylib/index.html}
 \item \emph{data structure and analysis}
 \item ressemble au traitement de données sous R (data.frame) 
\end{itemize}

\begin{pythonConsole}
>>> import numpy as np
>>> import pandas as pd
>>> A = np.array([[141, 70, 'M'], [176, 75, 'F'], [164, 66, 'M']])
>>> df = pd.DataFrame(A)
>>> print(df)
     0   1  2
0  141  70  M
1  176  75  F
2  164  66  M
>>> l_ind = ['riri', 'fifi', 'loulou']
>>> l_col = ['taille', 'masse', 'genre']
>>> df = pd.DataFrame(A, index=l_ind, columns=l_col)
>>> print(df)
       taille masse genre
riri      141    70    M
fifi      176    75    F
loulou    164    66    M
\end{pythonConsole}
\end{frame}
%_______________________________________________________________________________
%_______________________________________________________________________________
\begin{frame}[fragile]
\frametitle{Pandas}
\framesubtitle{}
\begin{itemize}
 \item Facilités pour traiter les données
 \item sélection d'une colonne : df['\dots']
 \item sélection d'une ligne : df.loc['\dots']
 \item ou par indices, slicing etc : df.iloc[i, j], df.iloc[i:j,k:l]
\end{itemize}

\begin{pythonConsole}
>>> df['masse']
riri      70
fifi      75
loulou    66
Name: masse, dtype: object
>>> df.loc['riri']
df.loc['riri']
taille    141
masse      70
sexe        M
Name: riri, dtype: object
>>> df.iloc[0,0]
'141'
>>> df.iloc[0:2, 1:2]
     masse
riri    70
fifi    75
\end{pythonConsole}
\end{frame}
%_______________________________________________________________________________
%_______________________________________________________________________________
\begin{frame}[fragile]
\frametitle{Pandas}
\framesubtitle{}
\begin{itemize}
 \item Extraction de données
\end{itemize}

\begin{pythonConsole}
>>> df.values
array([['141', '70', 'M'],
       ['176', '75', 'F'],
       ['164', '66', 'M']], dtype=object)
>>> df[df['masse'].astype('int')>67][['masse', 'taille']].values
array([['70', '141'],
       ['75', '176']], dtype=object)
\end{pythonConsole}
\end{frame}
%_______________________________________________________________________________
%_______________________________________________________________________________
\begin{frame}[fragile]
\frametitle{Pandas}
\framesubtitle{}
\begin{itemize}
 \item Beaucoup de méthodes pour traiter les tableaux de données. 
\end{itemize}

\begin{pythonConsole}
>>> taille = 170 + 10 * np.random.randn(100)
>>> masse = 70 + 10 * np.random.randn(100)
>>> dico = {'taille': taille, 'masse': masse}
>>> df = pd.DataFrame(dico)
>>> df.describe()
            masse      taille
count  100.000000  100.000000
mean    70.507943  169.915690
std      9.999280   10.264489
£min£     45.467455  144.149219
25%     63.300037  162.736237
50%     70.043825  171.168028
75%     78.153612  176.164359
£max£     94.118966  201.781565
\end{pythonConsole}
\end{frame}
%_______________________________________________________________________________
%_______________________________________________________________________________
\begin{frame}[fragile]
\frametitle{Pandas}
\framesubtitle{Series}
\begin{itemize}
 \item Séries (Series)
 \item Possibilité d'indexation spéciale.
 \item DataFrame : colonnes de series
\end{itemize}

\begin{pythonConsole}
>>> help(pd.Series)
class Series(Series)
 |  One-dimensional ndarray with axis labels (including time series).
>>> t = np.arange(0, 3, 0.1)
>>> x = np.cos(2 * np.pi * 2 * t)
>>> s = pd.Series(x, t)
>>> print(s[02:0.5])
0.2   -0.809017
0.3   -0.809017
0.4    0.309017
0.5    1.000000
\end{pythonConsole}
\end{frame}
%_______________________________________________________________________________
%_______________________________________________________________________________
\begin{frame}[fragile]
\frametitle{Pandas}
\framesubtitle{Series}
\begin{itemize}
 \item Possibilité d'indexation spéciale.
\end{itemize}

\begin{pythonConsole}
>>> t1 = np.arange(0, 3, 0.3)
>>> t2 = np.arange(1, 2, 0.2)
>>> x1 = np.cos(2 * np.pi * 0.01 * t1)
>>> x2 = np.cos(2 * np.pi * 0.03 * t2)
>>> s1 = pd.Series(x1, t1)
>>> s2 = pd.Series(x2, t2)
>>> df = pd.DataFrame({'x1':s1, 'x2':s2})
>>> print(df)                              >>> print(df.fillna(method='ffill'))
           x1        x2                               x1        x2
0.0  1.000000       NaN                    0.0  1.000000       NaN
0.3  0.999822       NaN                    0.3  0.999822       NaN
0.6  0.999289       NaN                    0.6  0.999289       NaN
0.9  0.998402       NaN                    0.9  0.998402       NaN
1.0       NaN  0.982287                    1.0  0.998402  0.982287
1.2  0.997159  0.974527                    1.2  0.997159  0.974527
1.4       NaN  0.965382                    1.4  0.997159  0.965382
1.5  0.995562       NaN                    1.5  0.995562  0.965382
1.6       NaN  0.954865                    1.6  0.995562  0.954865
1.8  0.993611  0.942991                    1.8  0.993611  0.942991
2.1  0.991308       NaN                    2.1  0.991308  0.942991
2.4  0.988652       NaN                    2.4  0.988652  0.942991
2.7  0.985645       NaN                    2.7  0.985645  0.942991

\end{pythonConsole}
\end{frame}
%_______________________________________________________________________________
%_______________________________________________________________________________
\subsection{IPython}
%_______________________________________________________________________________
%_______________________________________________________________________________
\begin{frame}[fragile]
\frametitle{IPython}
\framesubtitle{}
\begin{itemize}
 \item \myFig{height=0.5cm}{./fig/ipython-logo.png} \, \url{http://ipython.org}
 \item \emph{Enhanced python console}
 \item Attention, ce n'est pas un paquet mais une console améliorée !
 \item Accessible à partir d'un terminal (ipython) 
 \item Aide simplifiée
\end{itemize}

\begin{shell}
$ ipython /* $ */
\end{shell} 

\begin{pythonConsole}
Python 3.5.1 (v3.5.1:37a07cee5969, Dec  5 2015, 21:12:44) 
Type "copyright", "credits" or "license" for more information.

IPython 4.0.1 -- An enhanced Interactive Python.
?         -> Introduction and overview of IPython£'£s features.
%quickref -> Quick reference.
£help£      -> Python£'£s own £help£ system.
£object£?   -> Details about 'object', use 'object??' for extra details.
\end{pythonConsole}
\end{frame}
%_______________________________________________________________________________
%_______________________________________________________________________________
\begin{frame}[fragile]
\frametitle{IPython}
\framesubtitle{}
\begin{itemize}
 \item Complétion automatique avec la touche tab.
 \item Accès rapide à l'historique avec $\uparrow$ et $\downarrow$.
 \item Fonctions magiques (magic functions) : 
 \begin{itemize}
  \item \% line oriented
  \item \%\% cell oriented
 \end{itemize}
\end{itemize}

\begin{pythonConsole}
In [1]: g = lambda x: cos(x ** 2)
In [2]: In [21]: %time g(10)
CPU times: user 41 £$\mu$£s, sys: 1 £$\mu$£s, total: 42 £$\mu$£s
Wall time: 46 £$\mu$£s
Out[2]: 0.862318872287684
In [3]: %%time
    ...: a = 10
    ...: g(a)
    ...: 
CPU times: user 42 £$\mu$£s, sys: 1 £$\mu$£s, total: 43 £$\mu$£s
Wall time: 47 £$\mu$£s

In [4]: magic £\#£ or \%magic
IPython£'£s £'£magic£'£ functions
===========================

The magic function system provides a series of functions which allow you to
control the behavior of IPython itself, plus a lot of system-type
features. There are two kinds of magics, line-oriented and cell-oriented.
\end{pythonConsole}
\end{frame}
%_______________________________________________________________________________
%_______________________________________________________________________________
\begin{frame}[fragile]
\frametitle{IPython}
\framesubtitle{}
\begin{itemize}
 \item exécution de modules Python (\%run) + debugger + profiler (see \%run? for details).  
 \item accès aux fonctions du système. 
 \item rechargement dynamique des modules changés à l'extérieur. 
\end{itemize}

\begin{pythonConsole}
In [1]: %run mod1.py

In [2]: %cd ./Documents/
/Volumes/HD1/Users/becqg/Documents
In [3]: %%system
    ...: pwd
    ...: 
Out[3]: ['/Volumes/HD1/Users/becqg/Documents']

In [4]: %load_ext autoreload           In [4]: £import£ importlib as imp
In [5]: %aimport mod1                  In [5]: imp.£reload£(mod1) 
In [6]: %autoreload 1  
\end{pythonConsole}
\end{frame}
%_______________________________________________________________________________
%_______________________________________________________________________________
\begin{frame}[fragile]
\frametitle{IPython}
\framesubtitle{}
\begin{itemize}
 \item import des fonctions pylab amélioré. 
\end{itemize}

\begin{pythonConsole}
In [1]: %pylab
Using matplotlib backend: MacOSX
Populating the interactive namespace from numpy and matplotlib

In [2]: A = empty((3,3))

In [3]: %pylab? 
...
%pylab makes the following imports::

    import numpy
    import matplotlib
    from matplotlib import pylab, mlab, pyplot
    np = numpy
    plt = pyplot

    from IPython.display import display
    from IPython.core.pylabtools import figsize, getfigs

    from pylab import *
    from numpy import *
\end{pythonConsole}
\end{frame}
%_______________________________________________________________________________
%_______________________________________________________________________________
%\subsubsection{ipython notebook}
%_______________________________________________________________________________
%_______________________________________________________________________________
\begin{frame}[fragile]
\frametitle{IPython}
\framesubtitle{notebook}
\begin{itemize}
 \item Accessible à partir d'un terminal (ipython notebook, \em{jupyter notebook}) 
 \item Ouverture et création de notebooks (*.ipynb)
\end{itemize}
\begin{center}
 \myFig{width=10cm}{./fig/ipython_notebook_server.png}
\end{center}
\end{frame}
%_______________________________________________________________________________
%_______________________________________________________________________________
\begin{frame}[fragile]
\frametitle{Jupyter}
\framesubtitle{jupyter notebook}
\begin{itemize}
 \item Les cellules exécutent du code ou formatent du texte
 \item Chargement des fonctions pylab avec affichage dans le notebook : \%pylab inline 
\end{itemize}
\begin{center}
 \myFig{width=8cm}{./fig/ipython_notebook_p1.png}
\end{center}
\end{frame}
%_______________________________________________________________________________
%_______________________________________________________________________________
\begin{frame}[fragile]
\frametitle{Jupyter}
\framesubtitle{jupyter notebook}
\begin{itemize}
 \item Saisie et affichage en HTML et LaTeX, entre autres.   
\end{itemize}
\begin{center}
 \myFig{width=8cm}{./fig/ipython_notebook_p2.png}
\end{center}
\end{frame}
%_______________________________________________________________________________
%_______________________________________________________________________________
\begin{frame}[fragile]
\frametitle{Jupyter}
\framesubtitle{}
\begin{itemize}
 \item \url{https://jupyter.org}
 \item Interface avec d'autres langages de programmation.  
\end{itemize}
\begin{center}
 \myFig{width=10cm}{./fig/jupyter-site.png}
\end{center}
\end{frame}
%_______________________________________________________________________________
%_______________________________________________________________________________
\subsection{scikit-learn}
%_______________________________________________________________________________
%_______________________________________________________________________________
\begin{frame}[fragile]
\frametitle{Scikit-learn}
\framesubtitle{}
\begin{itemize}
 \item \url{http://scikit-learn.org/} 
 \item \url{https://github.com/scikit-learn/scikit-learn} 
\end{itemize}
\myFigCentered{width=8cm}{./fig/scikit-learn.png}
\end{frame}
%_______________________________________________________________________________
%_______________________________________________________________________________
\begin{frame}[fragile]
\frametitle{Scikit-learn}
\framesubtitle{Help content}
\begin{pythonConsole}
>>> import sklearn
>>> help(sklearn)

DESCRIPTION
    Machine learning module for Python
    ==================================
    
    sklearn £is£ a Python module integrating classical machine
    learning algorithms £in£ the tightly-knit world of scientific Python
    packages (numpy, scipy, matplotlib).
    
    It aims to provide simple £and£ efficient solutions to learning problems
    that are accessible to everybody £and£ reusable £in£ various contexts:
    machine-learning as a versatile tool £for£ science and engineering.
    
    See http://scikit-learn.org £for£ complete documentation.

PACKAGE CONTENTS
    __check_build (package)
    _build_utils
    _isotonic
    base
    calibration
    cluster (package)
    covariance (package)
    cross_decomposition (package)
    cross_validation    
\end{pythonConsole}
\end{frame}
%_______________________________________________________________________________
%_______________________________________________________________________________
\begin{frame}[fragile]
\frametitle{Scikit-learn}
\framesubtitle{Help content}                                                                  
\begin{pythonConsole}
    datasets (package)                       preprocessing (package)   
    decomposition (package)                  qda                       
    discriminant_analysis                    random_projection         
    dummy                                    semi_supervised (package) 
    ensemble (package)                       setup                     
    externals (package)                      svm (package)             
    feature_extraction (package)             tests (package)           
    feature_selection (package)              tree (package)            
    gaussian_process (package)               utils (package)           
    grid_search                              
    isotonic                                 
    kernel_approximation                     
    kernel_ridge                             
    lda                                      
    learning_curve                           
    linear_model (package)                   
    manifold (package)                       
    metrics (package)                        
    mixture (package)                        
    multiclass                               
    naive_bayes                              
    neighbors (package)                      
    neural_network (package)                 
    pipeline
\end{pythonConsole}
\end{frame}
%_______________________________________________________________________________
%_______________________________________________________________________________
\begin{frame}[fragile]
\frametitle{Scikit-learn}
\framesubtitle{Exemple de Classifieur : Linear Discriminant Analysis}
\begin{pythonConsole}
>>> from pylab import *
>>> from sklearn.datasets import load_iris
>>> from sklearn.discriminant_analysis import LinearDiscriminantAnalysis as LDA
>>> dic = load_iris()
>>> x = dic["data"]
>>> y = dic["target"]
>>> nClasses = unique(y).size
>>> clf = LDA()
>>> clf.fit(x[:, :2], y)
LinearDiscriminantAnalysis(n_components=None, priors=None, shrinkage=None,
              solver='svd', store_covariance=False, tol=0.0001)
\end{pythonConsole}
\end{frame}
%_______________________________________________________________________________
%_______________________________________________________________________________
\begin{frame}[fragile]
\frametitle{Scikit-learn}
\framesubtitle{Exemple de Classifieur : Linear Discriminant Analysis}
\begin{pythonConsole}
>>> from matplotlib import colors
>>> cmap = colors.LinearSegmentedColormap.from_list('3_classes', 
[(1, 0.8, 0.8), (0.8, 1, 0.8), (0.8, 0.8, 1)], N=3)
>>> cm.register_cmap(cmap=cmap)
>>> (mx1, mx2) = meshgrid(linspace(4, 8, 100), linspace(2, 5, 100))
>>> Z = clf.predict(vstack((mx1.flatten().T, mx2.flatten())).T)
>>> pcolor(mx1, mx2, Z.reshape(mx1.shape), cmap='3_classes')
>>> colors = ["r", "g", "b"]
>>> for i in range(nClasses): 
...   scatter(x[y==i, 0], x[y==i, 1], marker='o', edgecolor='k', linewidths=1, 
c=colors[i], s=150)
>>> axis(xmin=4, xmax=8, ymin=2, ymax=5); show()
\end{pythonConsole}
\myFigCentered{width=4cm}{./fig/sklearn_lda.png}
\end{frame}
%_______________________________________________________________________________
%_______________________________________________________________________________
\subsection{Mayavi}
%_______________________________________________________________________________
%_______________________________________________________________________________
\begin{frame}
\frametitle{Mayavi}
\framesubtitle{}
\frameCC{%
\begin{itemize}
 \item \url{http://docs.enthought.com/mayavi/mayavi/}
 \item \url{https://github.com/enthought/mayavi}
 \item Manipulation des objets 3D améliorée (objet VTK). 
\end{itemize}
}
{\myFig{width=8cm}{./fig/Mayavi_site.png}}
\end{frame}
%_______________________________________________________________________________
%_______________________________________________________________________________
\begin{frame}
\frametitle{Mayavi}
\framesubtitle{}
\myFigCentered{width=8cm}{./fig/Mayavi.png}
\end{frame}
%_______________________________________________________________________________
%_______________________________________________________________________________
\begin{frame}[fragile]
\frametitle{Mayavi}
\framesubtitle{import mayavi.engine}
\begin{itemize}
 \item Dans une console Python : import mayavi. \dots
\end{itemize}
\begin{pythonConsole}
>>> from mayavi.api import Engine
>>> engine = Engine()
>>> engine.start()
>>> engine.new_scene()
>>> from mayavi.sources.parametric_surface import ParametricSurface
>>> parametric_surface1 = ParametricSurface()
>>> scene = engine.scenes[0]
>>> engine.add_source(parametric_surface1, scene)
>>> from mayavi.modules.surface import Surface
>>> surface1 = Surface()
>>> engine.add_filter(surface1,parametric_surface1)
\end{pythonConsole}
\end{frame}
%_______________________________________________________________________________
%_______________________________________________________________________________
\subsection{Autres ressources : exemple C}
%_______________________________________________________________________________
\begin{frame}[fragile]
\frametitle{Importation de bibliothèques de fonctions écrites en C}
\framesubtitle{Exemple using module ctypes}
\begin{python}
import ctypes
libName = './clz.lib'
libCLZ = ctypes.CDLL(libName)
clz_c = libCLZ.clz
clz_c.restype = ctypes.c_uint
sequence = numpy.ctypeslib.ndpointer(dtype=numpy.int) 
clz_c.argtypes = ([sequence, ctypes.c_uint])
# conversion of s into sequence with numpy.asarray
c = clz_c(numpy.asarray(s, dtype='int'), n)
\end{python}
\end{frame}
%_______________________________________________________________________________

