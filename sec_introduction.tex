%===============================================================================
\section{Introduction}
%_______________________________________________________________________________
\subsection{Historique}
%...............................................................................
\begin{frame}[fragile]
\frametitle{Historique}
\framesubtitle{Python}
\frameLR{7cm}{3cm}
{%
\begin{itemize}
\item 1989--1995 : Hollande
\only<2>{%
\small
\begin{itemize}
 \item Centrum voor Wiskunde en Informatica (CWI). 
 \item \emph{Guido von Rossum}, fan des Monty Python, travaille sur : 
 \begin{itemize}
  \item ABC : langage de script, syntaxe et indentation. 
  \item Modula-3 : gestion des exceptions, orienté objet. 
  \item langage C, Unix.
  \item OS distribué Amoeba: accès difficile en shell.
 \end{itemize}
 \item Créée le langage Python :
 \begin{itemize}
  \item 1991/02 : versions 0.9.06 déposée sur un newsgroup de Usenet
  \item 1995 : dépôt de la version 1.2
 \end{itemize}
\end{itemize}}
\item 1995--1999 : USA 
\only<3>{%
\small
\begin{itemize}
 \item Corporation for National Research Initiatives (CNRI), non profit organisation, Reston, USA.
 \item Grail7 : navigateur internet utilisant Tk. 
 \item 1999 : projet \emph{Computer Programming for Everybody} (CP4E) (CNRI, DARPA Defense Advanced Research Projects Agency) :
 \begin{itemize}
  \item Python comme langage d'enseignement de la programmation. 
  \item Création de l'IDLE (Integrated DeveLopment Environment)
 \end{itemize}
 \item 1999 : Python 1.6
\end{itemize}}
\item 1999+ : Worldwide open source
\only<4>{%
\small
\begin{itemize}
 \item BeOpen.com :
 \begin{itemize} 
  \item compatibilité GPL (General Public Licence)
  \item création de la branche pythonLabs
 \end{itemize}
 \item 2000 : Python Software Foundation 
 \begin{itemize} 
  \item Python 2.1 : changement licence, dérivée de Apache Software Foundation (OO, svn, commons plutôt java) (\url{http://www.apache.org}). 
 \end{itemize} 
 \item 2008: Python 3.0 
\end{itemize}}
\end{itemize}
{\tiny sources : \url{http://www.wikipedia.org}, \url{http://www.python.org}}%
}
{%
\only<2>{\myFig{width=2cm}{./fig/MontyPythonFoot.png} \\ \myFig{width=3cm}{./fig/ordi1990.jpg}}%
\only<3>{\myFig{width=3cm}{./fig/grail.png}}%
\only<4>{\myFig{width=3cm}{./fig/python-logo.png}}%
}
\end{frame}
%...............................................................................
%...............................................................................
\begin{frame}
\frametitle{Historique}
\framesubtitle{Guido van Rossum}
\frameLR{7cm}{3cm}{%
\begin{itemize}
 \item Guido van Rossum : 
 \begin{itemize}
  \item 31 janvier 1956 (60 ans)
  \item Développeur néerlandais
  \item 1982 : M. Sc  
  \item Développeur ABC. 
 \end{itemize} 
 \item Créateur Python : \emph{Benevolent Dictator For Life (BDFL)}
 \begin{itemize}
  \item 1991 : Python 0.9.06
  \item 1999 : Grail
 \end{itemize}
 \item 2002 : Prix pour le développement du logiciel libre 2001 décerné par la \emph{Free Software Foundation}
 \item 2005--2012 : Google (python) 
 \item 2013 : Dropbox
\end{itemize}
{\tiny sources : \url{http://www.wikipedia.org}, \url{http://www.python.org}}%
}
{%
\myFig{width=3cm}{./fig/guido.jpg} \\ 
{\tiny 2006, source wikipedia}%
}
\end{frame}

%...............................................................................
%_______________________________________________________________________________
%_______________________________________________________________________________
\subsection{Spécificités}
%...............................................................................
\begin{frame}
\frametitle{Spécificités}
\framesubtitle{}
\url{http://www.python.org/about/}
\begin{itemize}
 \item Fortement typé.  
 \item Objet.
 \item Script, séquentiel, interprété : fichier génère du byte code.
 \item Comparé à Tcl, Perl, Ruby, Scheme, Java.
\end{itemize}
\end{frame}
%...............................................................................
%...............................................................................
\begin{frame}[fragile]
\frametitle{Spécificités}
\framesubtitle{Exemple 1er programme}
Dans le fichier "hello.py" : 
\begin{python}
print("Bonjour monde")
\end{python}

Exécution dans une interface système (terminal, shell) : 
\begin{shell}
~/python/example> python hello.py
Bonjour monde
\end{shell}
\end{frame}
%...............................................................................
%...............................................................................
\begin{frame}[fragile]
\frametitle{Spécificités}
\framesubtitle{Exemple shell scripting}
Exemple copies des fichiers '.txt' et '.tex' du répertoire 'a' vers 'b'.
\lstinputlisting[style=myPython]{./example/shell/shellScript.py}
\end{frame}
%...............................................................................
%_______________________________________________________________________________
%_______________________________________________________________________________
\subsection{Utilisation}
%...............................................................................
\begin{frame}[fragile]
\frametitle{Utilisations}
\framesubtitle{A partir du shell}
\begin{itemize}
\item Fichiers "*.py" contient des scripts, des définitions de fonctions, de classes\dots 
Ils sont exécutés dans le shell avec la commande "python". 

Exemple : "python hello.py" 

\item La commande "python" ouvre une console utilisateur (\emph{interpreter}) avec une invite de commande (prompt) caractéristique "> > > "
\end{itemize}

\begin{python}
Python 3.5.1 (v3.5.1:37a07cee5969, Dec  5 2015, 21:12:44) 
[GCC 4.2.1 (Apple Inc. build 5666) (dot 3)] on darwin
Type "help", "copyright", "credits" or "license" for more information.
>>> 
\end{python}

\end{frame}
%...............................................................................
%...............................................................................
\begin{frame}[fragile]
\frametitle{Utilisations}
\framesubtitle{Utilisation de l'interpréteur}
\begin{itemize}
\item Association d'une valeur à une variable : a = 'abc'
\item Affichage de la valeur : print()
\item Liste des variables, attributs, fonctions disponibles\dots : dir()
\end{itemize}

\begin{python}
>>> a = "abc"
>>> print(a)
abc
>>> dir()
['__builtins__', '__doc__', '__name__', '__package__', 'a']

>>> dir(a)
['__add__', '__class__', '__contains__', '__delattr__', '__doc__', 
'__eq__', '__format__', '__ge__', '__getattribute__', '__getitem__',
'__getnewargs__', '__getslice__', '__gt__', '__hash__', '__init__', 
'__le__', '__len__', '__lt__', '__mod__', '__mul__', '__ne__', '__new__', 
'__reduce__', '__reduce_ex__', '__repr__', '__rmod__', '__rmul__', 
'__setattr__', '__sizeof__', '__str__', '__subclasshook__', 
'_formatter_field_name_split', '_formatter_parser', 'capitalize', 
'center', 'count', 'decode', 'encode', 'endswith', 'expandtabs', 'find',
'format', 'index', 'isalnum', 'isalpha', 'isdigit', 'islower', 'isspace',
'istitle', 'isupper', 'join', 'ljust', 'lower', 'lstrip', 'partition',
'replace', 'rfind', 'rindex', 'rjust', 'rpartition', 'rsplit', 'rstrip',
'split', 'splitlines', 'startswith', 'strip', 'swapcase', 'title', 
'translate', 'upper', 'zfill']
\end{python}
\end{frame}
%...............................................................................
%...............................................................................
\begin{frame}[fragile]
\frametitle{Utilisations}
\framesubtitle{Utilisation de l'interpréteur}
\begin{itemize}
 \item Besoin d'aide : help()
\end{itemize}
\begin{pythonConsole}
>>> help(a.find)

Help on built-£in£ function find:

find(...)
    S.find(sub [,start [,end]]) -> int
    
    Return the lowest index £in£ S where substring sub £is£ found,
    such that sub £is£ contained within S[start:end].  Optional
    arguments start £and£ end are interpreted as £in£ slice notation.
    
    Return -1 on failure.
(END) 
\end{pythonConsole}
\end{frame}
%...............................................................................
%_______________________________________________________________________________
%_______________________________________________________________________________
\subsection{Site web}
%...............................................................................
\begin{frame}
\frametitle{Site Web}
\framesubtitle{www.python.org}
\frameC{\myFig{width=10cm}{./fig/sitePython.png}}
\end{frame}
%...............................................................................
%_______________________________________________________________________________
%_______________________________________________________________________________
\subsection{Documentation}
%...............................................................................
\begin{frame}
\frametitle{Documentation}
\framesubtitle{Trouver la documentation}
\frameC{\myFig{width=10cm}{./fig/sitePythonDoc.png}}
\end{frame}
%...............................................................................
%...............................................................................
\begin{frame}
\frametitle{Documentation}
\framesubtitle{Lire la documentation - Tutorial}
\frameC{\myFig{width=10cm}{./fig/PDFTutorial.png}}
\end{frame}
%...............................................................................
%...............................................................................
\begin{frame}
\frametitle{Documentation}
\framesubtitle{Lire la documentation - Library reference}
\frameC{\myFig{width=10cm}{./fig/PDFLibrary.png}}
\end{frame}
%...............................................................................
%...............................................................................
\begin{frame}
\frametitle{Documentation}
\framesubtitle{Lire la documentation - Library reference}
\frameC{\myFig{width=10cm}{./fig/PDFLibraryOS.png}}
\end{frame}
%...............................................................................
%...............................................................................
\begin{frame}
\frametitle{Documentation}
\framesubtitle{Lire la documentation - Language reference}
\frameC{\myFig{width=10cm}{./fig/PDFReference.png}%

{\scriptsize\textit{p. ex. grammaire sous forme de Backus-Naur (BNF)}}}
\end{frame}
%...............................................................................
%...............................................................................
\begin{frame}
\frametitle{Python Enhancements Proposals}
\framesubtitle{PEPs}
\frameCC{% 
Propositions et conseils pour l'utilisation et l'amélioration du langage.}
{%
\myFig{width=8cm}{./fig/sitePythonPEPs.png}
}
\end{frame}
%...............................................................................
%_______________________________________________________________________________
%_______________________________________________________________________________
\subsection{Installation}
%...............................................................................
\begin{frame}
\frametitle{Installation}
\framesubtitle{}
\begin{itemize}
 \item Téléchargement sous \urlPython\ pour les OS courants : Windows, Linux, Mac. 
 \item 32 bits vs 64 bits : 
  \begin{itemize}
   \small
   \item Performances vs compatibilités bibliothèques (paquets) 
   \item Problèmes de configuration des compilateurs (gcc).
  \end{itemize}
 \item Autres distributions : Enthought Canopy, pythonXY, Anaconda, WinPython, \dots 
 \item Implémentations alternatives : 
 \begin{itemize}
  \small
  \item Langage identique mais implémentation différentes permet par exemple : d'interfacer du java facilement (Jython), d'être plus performant pour les calculs avec un compilateur JIT (Pypy)\dots
 \end{itemize}
\end{itemize}
\end{frame}
%...............................................................................
%...............................................................................
\begin{frame}
\frametitle{Installation}
\framesubtitle{Python 2.7 vs 3.x}

\begin{itemize}
\item 2.7 : figée.
\item 3.x : présent et futur : 
\begin{itemize}
 \item {\it better Unicode support}
 \item Quelques inconsistences du langage (ex print 'a' vs print('a')).  
 \item Résultats de la division des entiers (1/2 : 2.x, = 0 ; 3.x, = 0.5). 
 \item \dots
\end{itemize}
\item Peut poser des problèmes :
\begin{itemize}
 \item Paquets portées sur 3.x ?
 \item Compatibilité 64 bits ? 
\end{itemize}
\end{itemize}
\end{frame}
%...............................................................................
%_______________________________________________________________________________
%===============================================================================

