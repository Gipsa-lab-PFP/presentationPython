\documentclass[]{beamer}
\usepackage{CJKutf8}
\usepackage[utf8]{inputenc}
\usepackage[T1]{fontenc}
%\usepackage{lmodern}
\usepackage{listings}

%============================================================================
\author{G.~Becq}
\title{Une pas si courte introduction au langage de programmation Python comme alternative à Matlab pour réaliser des calculs scientifiques ou d'autres applications.}
\date{\today}
%============================================================================
\input{command}
\AtBeginSection{\framePlanCurrentSection}
%============================================================================
%============================================================================
\includeonly{%
sec_introduction, 
sec_language, 
sec_scientific, 
sec_environment, 
sec_conclusion, 
sec_appendix, 
}

% \includeonlylecture{day1}
%\includeonlylecture{day2}

\comment{%
\AtBeginLecture{
\begin{itemize}
\item S1: 
\begin{itemize}
	\item Introduction
	\item Description du langage
	\item Packaging
\end{itemize}
\item S2: 
\begin{itemize}
	\item Python scientifique
	\item Distibution, environnements, 
	\item Conclusion
\end{itemize}
\end{itemize}
}
}% end Comment
%============================================================================
\begin{document}
%____________________________________________________________________________


\maketitle

\framePlan
\lecture[Python base]{Introduction au langage Python}{day1}
%===============================================================================
\section{Introduction}
%_______________________________________________________________________________
\subsection{Historique}
%...............................................................................
\begin{frame}[fragile]
\frametitle{Historique}
\framesubtitle{Python}
\frameLR{7cm}{3cm}
{%
\begin{itemize}
\item 1989--1995 : Hollande
\only<2>{%
\small
\begin{itemize}
 \item Centrum voor Wiskunde en Informatica (CWI). 
 \item \emph{Guido von Rossum}, fan des Monty Python, travaille sur : 
 \begin{itemize}
  \item ABC : langage de script, syntaxe et indentation. 
  \item Modula-3 : gestion des exceptions, orienté objet. 
  \item langage C, Unix.
  \item OS distribué Amoeba: accès difficile en shell.
 \end{itemize}
 \item Créée le langage Python :
 \begin{itemize}
  \item 1991/02 : versions 0.9.06 déposée sur un newsgroup de Usenet
  \item 1995 : dépôt de la version 1.2
 \end{itemize}
\end{itemize}}
\item 1995--1999 : USA 
\only<3>{%
\small
\begin{itemize}
 \item Corporation for National Research Initiatives (CNRI), non profit organisation, Reston, USA.
 \item Grail7 : navigateur internet utilisant Tk. 
 \item 1999 : projet \emph{Computer Programming for Everybody} (CP4E) (CNRI, DARPA Defense Advanced Research Projects Agency) :
 \begin{itemize}
  \item Python comme langage d'enseignement de la programmation. 
  \item Création de l'IDLE (Integrated DeveLopment Environment)
 \end{itemize}
 \item 1999 : Python 1.6
\end{itemize}}
\item 1999+ : Worldwide open source
\only<4>{%
\small
\begin{itemize}
 \item BeOpen.com :
 \begin{itemize} 
  \item compatibilité GPL (General Public Licence)
  \item création de la branche pythonLabs
 \end{itemize}
 \item 2000 : Python Software Foundation 
 \begin{itemize} 
  \item Python 2.1 : changement licence, dérivée de Apache Software Foundation (OO, svn, commons plutôt java) (\url{http://www.apache.org}). 
 \end{itemize} 
 \item 2008: Python 3.0 
\end{itemize}}
\end{itemize}
{\tiny sources : \url{http://www.wikipedia.org}, \url{http://www.python.org}}%
}
{%
\only<2>{\myFig{width=2cm}{./fig/MontyPythonFoot.png} \\ \myFig{width=3cm}{./fig/ordi1990.jpg}}%
\only<3>{\myFig{width=3cm}{./fig/grail.png}}%
\only<4>{\myFig{width=3cm}{./fig/python-logo.png}}%
}
\end{frame}
%...............................................................................
%...............................................................................
\begin{frame}
\frametitle{Historique}
\framesubtitle{Guido van Rossum}
\frameLR{7cm}{3cm}{%
\begin{itemize}
 \item Guido van Rossum : 
 \begin{itemize}
  \item 31 janvier 1956 (59 ans)
  \item Développeur néerlandais
  \item 1982 : M. Sc  
  \item Développeur ABC. 
 \end{itemize} 
 \item Créateur Python : \emph{Benevolent Dictator For Life (BDFL)}
 \begin{itemize}
  \item 1991 : Python 0.9.06
  \item 1999 : Grail
 \end{itemize}
 \item 2002 : Prix pour le développement du logiciel libre 2001 décerné par la \emph{Free Software Foundation}
 \item 2005--2012 : Google (python) 
 \item 2013 : Dropbox
\end{itemize}
{\tiny sources : \url{http://www.wikipedia.org}, \url{http://www.python.org}}%
}
{%
\myFig{width=3cm}{./fig/guido.jpg} \\ 
{\tiny 2006, source wikipedia}%
}
\end{frame}

%...............................................................................
%_______________________________________________________________________________
%_______________________________________________________________________________
\subsection{Spécificités}
%...............................................................................
\begin{frame}
\frametitle{Spécificités}
\framesubtitle{}
\url{http://www.python.org/about/}
\begin{itemize}
 \item Fortement typé.  
 \item Objet.
 \item Script, séquentiel, interprété : fichier génère du byte code.
 \item Comparé à Tcl, Perl, Ruby, Scheme, Java.
\end{itemize}
\end{frame}
%...............................................................................
%...............................................................................
\begin{frame}[fragile]
\frametitle{Spécificités}
\framesubtitle{Exemple 1er programme}
Dans le fichier "hello.py" : 
\begin{python}
print("Bonjour monde")
\end{python}

Exécution dans une interface système (terminal, shell) : 
\begin{shell}
~/python/example> python hello.py
Bonjour monde
\end{shell}
\end{frame}
%...............................................................................
%...............................................................................
\begin{frame}[fragile]
\frametitle{Spécificités}
\framesubtitle{Exemple shell scripting}
Exemple copies des fichiers '.txt' et '.tex' du répertoire 'a' vers 'b'.
\lstinputlisting[style=myPython]{./example/shell/shellScript.py}
\end{frame}
%...............................................................................
%_______________________________________________________________________________
%_______________________________________________________________________________
\subsection{Utilisation}
%...............................................................................
\begin{frame}[fragile]
\frametitle{Utilisations}
\framesubtitle{A partir du shell}
\begin{itemize}
\item Fichiers "*.py" contient des scripts et des définitions de fonctions.
Ils sont exécutés dans le shell avec la commande "python". 

Exemple : "python hello.py" 

\item La commande "python" ouvre une console utilisateur (\emph{interpreter}) avec une invite de commande (prompt) caractéristique "> > > "
\end{itemize}

\begin{python}
Enthought Canopy Python 2.7.3 | 64-bit | (default, Jun 14 2013, 18:17:36) 
[GCC 4.2.1 (Apple Inc. build 5666) (dot 3)] on darwin
Type "help", "copyright", "credits" or "license" for more information.
>>> 
\end{python}

\end{frame}
%...............................................................................
%...............................................................................
\begin{frame}[fragile]
\frametitle{Utilisations}
\framesubtitle{Utilisation de l'interpréteur}
\begin{itemize}
\item Association d'une valeur à une variable : a = 'abc'
\item Affichage de la valeur : print()
\item Liste des noms, attributs, fonctions disponibles : dir()
\end{itemize}

\begin{python}
>>> a = "abc"
>>> print(a)
abc
>>> dir()
['__builtins__', '__doc__', '__name__', '__package__', 'a']

>>> dir(a)
['__add__', '__class__', '__contains__', '__delattr__', '__doc__', 
'__eq__', '__format__', '__ge__', '__getattribute__', '__getitem__',
'__getnewargs__', '__getslice__', '__gt__', '__hash__', '__init__', 
'__le__', '__len__', '__lt__', '__mod__', '__mul__', '__ne__', '__new__', 
'__reduce__', '__reduce_ex__', '__repr__', '__rmod__', '__rmul__', 
'__setattr__', '__sizeof__', '__str__', '__subclasshook__', 
'_formatter_field_name_split', '_formatter_parser', 'capitalize', 
'center', 'count', 'decode', 'encode', 'endswith', 'expandtabs', 'find',
'format', 'index', 'isalnum', 'isalpha', 'isdigit', 'islower', 'isspace',
'istitle', 'isupper', 'join', 'ljust', 'lower', 'lstrip', 'partition',
'replace', 'rfind', 'rindex', 'rjust', 'rpartition', 'rsplit', 'rstrip',
'split', 'splitlines', 'startswith', 'strip', 'swapcase', 'title', 
'translate', 'upper', 'zfill']
\end{python}
\end{frame}
%...............................................................................
%...............................................................................
\begin{frame}[fragile]
\frametitle{Utilisations}
\framesubtitle{Utilisation de l'interpréteur}
\begin{itemize}
 \item Besoin d'aide : help()
\end{itemize}
\begin{pythonConsole}
>>> help(a.find)

Help on built-in function find:

find(...)
    S.find(sub [,start [,end]]) -> int
    
    Return the lowest index in S where substring sub is found,
    such that sub is contained within S[start:end].  Optional
    arguments start and end are interpreted as in slice notation.
    
    Return -1 on failure.
(END) 
\end{pythonConsole}
\end{frame}
%...............................................................................
%_______________________________________________________________________________
%_______________________________________________________________________________
\subsection{Site web}
%...............................................................................
\begin{frame}
\frametitle{Site Web}
\framesubtitle{www.python.org}
\frameC{\myFig{width=10cm}{./fig/sitePython.png}}
\end{frame}
%...............................................................................
%_______________________________________________________________________________
%_______________________________________________________________________________
\subsection{Documentation}
%...............................................................................
\begin{frame}
\frametitle{Documentation}
\framesubtitle{Trouver la documentation}
\frameC{\myFig{width=10cm}{./fig/sitePythonDoc.png}}
\end{frame}
%...............................................................................
%...............................................................................
\begin{frame}
\frametitle{Documentation}
\framesubtitle{Lire la documentation - Tutorial}
\frameC{\myFig{width=10cm}{./fig/PDFTutorial.png}}
\end{frame}
%...............................................................................
%...............................................................................
\begin{frame}
\frametitle{Documentation}
\framesubtitle{Lire la documentation - Library reference}
\frameC{\myFig{width=10cm}{./fig/PDFLibrary.png}}
\end{frame}
%...............................................................................
%...............................................................................
\begin{frame}
\frametitle{Documentation}
\framesubtitle{Lire la documentation - Library reference}
\frameC{\myFig{width=10cm}{./fig/PDFLibraryOS.png}}
\end{frame}
%...............................................................................
%...............................................................................
\begin{frame}
\frametitle{Documentation}
\framesubtitle{Lire la documentation - Language reference}
\frameC{\myFig{width=10cm}{./fig/PDFReference.png}}
\end{frame}
%...............................................................................
%...............................................................................
\begin{frame}
\frametitle{Python Enhancements Proposals}
\framesubtitle{PEPs}
\frameCC{% 
Propositions et conseils pour l'utilisation et l'amélioration du langage.}
{%
\myFig{width=8cm}{./fig/sitePythonPEPs.png}
}
\end{frame}
%...............................................................................
%_______________________________________________________________________________
%_______________________________________________________________________________
\subsection{Installation}
%...............................................................................
\begin{frame}
\frametitle{Installation}
\framesubtitle{}
\begin{itemize}
 \item Téléchargement sous \urlPython\ pour les OS courants : Windows, Linux, Mac. 
 \item 32 bits vs 64 bits : 
  \begin{itemize}
   \small
   \item Performances vs compatibilités bibliothèques (paquets) 
   \item Problèmes de configuration des compilateurs (gcc).
   \item Actuellement, peut générer des problèmes. 
  \end{itemize}
 \item Autres distributions : Enthought Canopy, pythonXY, WinPython, \dots 
 \item Implémentations alternatives : 
 \begin{itemize}
  \small
  \item Langage identique mais implémentation différentes permet par exemple : d'interfacer du java facilement (Jython), d'être plus performant pour les calculs avec un compilateur JIT (Pypy)\dots
 \end{itemize}
\end{itemize}
\end{frame}
%...............................................................................
%...............................................................................
\begin{frame}
\frametitle{Installation}
\framesubtitle{Python 2.7 vs 3.x}

\begin{itemize}
\item 2.7 : figée.
\item 3.x : présent et futur : 
\begin{itemize}
 \item {\it better Unicode support}
 \item Quelques inconsistences du langage (ex print 'a' vs print('a')).  
 \item Résultats de la division des entiers (1/2 : 2.x, = 0 ; 3.x, = 0.5). 
 \item \dots
\end{itemize}
\item Peut poser des problèmes :
\begin{itemize}
 \item Paquets portées sur 3.x ?
 \item Compatibilité 64 bits ? 
\end{itemize}
\end{itemize}
\end{frame}
%...............................................................................
%_______________________________________________________________________________
%===============================================================================


\include{sec_language}

\lecture[Python scientifique]{Paquets, Python Scientifique et Distributions}{day2}
%_______________________________________________________________________________
\section{Description des paquets scientifiques}
%_______________________________________________________________________________
%_______________________________________________________________________________
\begin{frame}[fragile]
\frametitle{Quelques Paquets et Outils Scientifiques}
\begin{itemize}
 \item SciPy : scientific python 
 \begin{itemize}
  \item Numpy
  \item SciPy library
  \item Matplotlib
  \item IPython
 \end{itemize}
 \item Mayavi : objets 3D avancés
 \item Scikit-learn : machine learning.
 \item \dots
\end{itemize}
\end{frame}
%_______________________________________________________________________________
%_______________________________________________________________________________
\subsection{SciPy}
%_______________________________________________________________________________
%_______________________________________________________________________________
\begin{frame}[fragile]
\frametitle{SciPy}
\framesubtitle{}
\begin{itemize}
 \item \url{http://www.scipy.org/}
\end{itemize}
\begin{center}
\includegraphics[width=10cm]{./fig/scipy.png}
\end{center}
\end{frame}
%_______________________________________________________________________________
%_______________________________________________________________________________
\subsection{Numpy}
%_______________________________________________________________________________
%_______________________________________________________________________________
\begin{frame}[fragile]
\frametitle{Numpy}
\begin{itemize}
 \item \myFig{height=0.5cm}{./fig/numpy-logo.png} \, \url{http://www.numpy.org}
 \item \emph{NumPy is the fundamental package for scientific computing with Python}
\end{itemize}
\begin{pythonConsole}
>>> import numpy
>>> help(numpy)

Help on package numpy:

NAME
    numpy

FILE
    /Users/becqg/Library/Enthought/Canopy_64bit/User/lib/python2.7/site-packages
    /numpy/__init__.py

DESCRIPTION
    NumPy
    =====
    
    Provides
      1. An array object of arbitrary homogeneous items
      2. Fast mathematical operations over arrays
      3. Linear Algebra, Fourier Transforms, Random Number Generation
    
...
\end{pythonConsole}
\end{frame}
%_______________________________________________________________________________
%_______________________________________________________________________________
\subsubsection{ndarray}
%_______________________________________________________________________________
%_______________________________________________________________________________
\begin{frame}[fragile]
\frametitle{N-dimensional array Object}
\framesubtitle{}
\begin{itemize}
 \item N-dimensional array : ndarray
 \item Création d'un tableau vide et réservation de l'espace (empty)
 \item Accès aux éléments : A[i, j, \dots] 
 \item indices de {\color{red}{0 à $n-1$}}
 \item indices négatifs de $-n$ à $-1$
\end{itemize}
\begin{pythonConsole}

>>> A = numpy.empty((2, 2))
>>> print(A)
[[ -1.28822975e-231   2.68678092e+154]
 [  2.24497156e-314   2.24499315e-314]]
>>> A[0, 0] = 1
>>> A[1, 0] = 2
>>> A[0, 1] = 11
>>> A[1, 1] = 12
>>> print(A)
[[  1.   2.]
 [ 11.  12.]]
>>> type(A)
<type 'numpy.ndarray'>
>>> A[-1, -2] = 11
\end{pythonConsole}
\end{frame}
%_______________________________________________________________________________
%_______________________________________________________________________________
\begin{frame}[fragile]
\frametitle{N-dimensional array Object}
\framesubtitle{}
\begin{itemize}
 \item Rappel : accès aux propiétés et méthodes (dir) 
\end{itemize}
\begin{pythonConsole}

>>> dir(A)
'T', ..., 'all', 'any', 'argmax', 'argmin', 'argpartition', 'argsort', 'astype',
'base', 'byteswap', 'choose', 'clip', 'compress', 'conj', 'conjugate', 'copy',
'ctypes', 'cumprod', 'cumsum', 'data', 'diagonal', 'dot', 'dtype', 'dump',
'dumps', 'fill', 'flags', 'flat', 'flatten', 'getfield', 'imag', 'item',
'itemset', 'itemsize', 'max', 'mean', 'min', 'nbytes', 'ndim', 'newbyteorder',
'nonzero', 'partition', 'prod', 'ptp', 'put', 'ravel', 'real', 'repeat',
'reshape', 'resize', 'round', 'searchsorted', 'setfield', 'setflags', 'shape',
'size', 'sort', 'squeeze', 'std', 'strides', 'sum', 'swapaxes', 'take',
'tofile', 'tolist', 'tostring', 'trace', 'transpose', 'var', 'view'
\end{pythonConsole}
\end{frame}
%_______________________________________________________________________________
%_______________________________________________________________________________
\begin{frame}[fragile]
\frametitle{N-dimensional array Object}
\framesubtitle{Attributs sur la forme du tableau}
\begin{itemize}
 \item Forme du tableau (shape), c'est un tuple.  
 \item Nombre de dimension (ndim)
 \item Type des éléments (dtype)
 \item Taille du tableau (size), c'est le nombre de cellules totales. 
\end{itemize}
\begin{pythonConsole}
>>> A.shape
(2, 2)
>>> (nRow, nCol) = A.shape
>>> nRow = A.shape[0]
>>> nCol = A.shape[1]
>>> A.ndim
2
>>> A.dtype
dtype('float64')
>>> A.size
4
\end{pythonConsole}
\end{frame}
%_______________________________________________________________________________
%_______________________________________________________________________________
\begin{frame}[fragile]
\frametitle{N-dimensional array Object}
\framesubtitle{Changement de forme}
\begin{itemize}
 \item Pour changer la forme (reshape)
 \item Transposition (T)
\end{itemize}
\begin{pythonConsole}
>>> B = A.reshape((4, 1))
array([[  1.],
       [  2.],
       [ 11.],
       [ 12.]])
>>> B.ndim
2
>>> B.size
4
>>> B.T
array([[ 1.,   2.,  11.,  12.]])
\end{pythonConsole}
\end{frame}
%_______________________________________________________________________________
%_______________________________________________________________________________
\begin{frame}[fragile]
\frametitle{N-dimensional array Object}
\framesubtitle{Copie de tableaux}
\begin{itemize}
 \item Les éléments de B sont les mêmes que ceux de A, seule la forme change. 
 \item Si on veut une copie (copy)
\end{itemize}
\begin{pythonConsole}
>>> B[0, 0] = 21
>>> print(A)
[[ 21.   2.]
 [ 11.  12.]]
>>> B[1, 0]
>>> C = A.copy()
>>> C[0, 0] = 31
>>> print(A[0,0], C[0,0])
(21.0, 31.0)
\end{pythonConsole}
\end{frame}
%_______________________________________________________________________________
%_______________________________________________________________________________
\begin{frame}[fragile]
\frametitle{N-dimensional array Object}
\framesubtitle{Création de tableaux}
\begin{itemize}
 \item tableau vide et réservation de l'espace (empty)
 \item initialisation à zeros (zeros)
 \item initialisation avec des uns (ones)
 \item tableau identité (eye) avec la dimension. 
 \item à partir de listes (array)
 \item suivant une étendue (arange)
\end{itemize}
\begin{pythonConsole}
>>> A = numpy.zeros((2, 4))
>>> print(A)
[[ 0.  0.  0.  0.]
 [ 0.  0.  0.  0.]]
>>> A = numpy.ones((3, 2))
>>> print(A)
[[ 1.  1.]
 [ 1.  1.]
 [ 1.  1.]]
>>> A = numpy.eye(2)
>>> print(A)
[[ 1.  0.]
 [ 0.  1.]]
>>> A = numpy.array([[1, 2], [11, 12]])
>>> print(A)
[[ 1  2]
 [11 12]]
>>> print(numpy.arange(0.5, 1.7, 0.1))
[ 0.5  0.6  0.7  0.8  0.9  1.   1.1  1.2  1.3  1.4  1.5  1.6]
\end{pythonConsole}
\end{frame}
%_______________________________________________________________________________
%_______________________________________________________________________________
\begin{frame}[fragile]
\frametitle{N-dimensional array Object}
\framesubtitle{Types}
\begin{itemize}
 \item Définition du type à la création
 \item Changement de type (astype)
 \item Multiplication ou addition avec un scalaire typé. 
\end{itemize}
\begin{pythonConsole}
>>> A = numpy.array([[1, 2], [11, 12]])
>>> print(A.dtype)
int64
>>> A = numpy.array([[1., 2], [11, 12]])
>>> print(A.dtype)
float64
>>> A = numpy.array([[1, 2], [11, 12]], dtype="float")
>>> print(A.dtype)
float64
>>> A = A.astype("complex")
>>> print(A)
[[  1.+0.j   2.+0.j]
 [ 11.+0.j  12.+0.j]]
>>> A = numpy.array([[1, 2], [11, 12]]) * 1.
>>> print(A.dtype)
float64
\end{pythonConsole}
\end{frame}
%_______________________________________________________________________________
%_______________________________________________________________________________
\begin{frame}[fragile]
\frametitle{N-dimensional array Object}
\framesubtitle{Additions, soustractions, multiplications sur les tableaux}
\begin{itemize}
 \item Addition, soustraction de matrices ou d'un scalaire (+, -)
 \item Multiplication par un scalaire (*)
 \item Produit élément par élément (*) 
\end{itemize}
\begin{pythonConsole}
>>> A = numpy.array([[1, 2], [11, 12]])
>>> B = numpy.array([[3, 4], [13, 14]])
>>> print(A + 10)
[[ 11.  12.]
 [ 21.  22.]]
>>> print(A + B)
[[  4.   6.]
 [ 24.  26.]]
>>> print(A * 10)
[[  10.   20.]
 [ 110.  120.]]
>>> print(A * B)
[[   3.    8.]
 [ 143.  168.]]
>>> C = numpy.ones((10, ))
>>> print(A * C)
Traceback (most recent call last):
  File £"£<stdin>£"£, line 1, in <module>
ValueError: operands could not be broadcast together with shapes (2,2) (10) 
\end{pythonConsole}
\end{frame}
%_______________________________________________________________________________
%_______________________________________________________________________________
\begin{frame}[fragile]
\frametitle{N-dimensional array Object}
\framesubtitle{Produit scalaire}
\begin{itemize}
 \item Produit scalaire (dot)
 \item See also numpy.dot : en général, pour chaque méthode associée à un ndarray, il existe une fonction équivalente dans numpy. 
\end{itemize}
\begin{pythonConsole}
>>> A = numpy.array([[1, 2], [11, 12]])
>>> B = numpy.array([[3, 4], [13, 14]])
>>> print(A.dot(B))
[[  29.   32.]
 [ 189.  212.]]
>>> print(numpy.dot(A, B))
[[  29.   32.]
 [ 189.  212.]]
>>> C = numpy.ones((10, ))
>>> print(A.dot(C))
Traceback (most recent call last):
  File £"£<stdin>£"£, line 1, in <module>
ValueError: matrices are not aligned
>>> 
\end{pythonConsole}
\end{frame}
%_______________________________________________________________________________
%_______________________________________________________________________________
\begin{frame}[fragile]
\frametitle{N-dimensional array Object}
\framesubtitle{Division}
\begin{itemize}
 \item Division par un scalaire (/)
 \item Division éléments par éléments (/) 
 \item Attention au type en Python 2.7 ! 
\end{itemize}
\begin{pythonConsole}
>>> A = numpy.array([[1, 2], [11, 12]])
>>> B = numpy.array([[3, 4], [13, 14]])
>>> print(A / 2)
[[0 1]
 [5 6]]
>>> print(A / B)
[[0 0]
 [0 0]]
>>> print(A / B.astype("float"))
[[ 0.33333333  0.5       ]
 [ 0.84615385  0.85714286]]
\end{pythonConsole}
\end{frame}
%_______________________________________________________________________________
%_______________________________________________________________________________
\begin{frame}[fragile]
\frametitle{N-dimensional array Object}
\framesubtitle{Autres méthodes}
\begin{itemize}
 \item max, min, sum, mean, std, cumsum, cumprod \dots sur tous les éléments ou sur une dimension particulière (kwarg axis). 
\end{itemize}
\begin{minipage}{5cm}
\begin{pythonConsole}
>>> A = numpy.ones((2, 3, 4))
>>> print(A)
[[[ 1.  1.  1.  1.]
  [ 1.  1.  1.  1.]
  [ 1.  1.  1.  1.]]

 [[ 1.  1.  1.  1.]
  [ 1.  1.  1.  1.]
  [ 1.  1.  1.  1.]]]
>>> print(A.cumsum())
[  1.   2.   3.   4.   5.   6.   7.   8.   9.  10.  11.  12.  13.  14.  15. 
  16.  17.  18.  19.  20.  21.  22.  23.  24.]
>>> print(A.cumsum(axis=0))
[[[ 1.  1.  1.  1.]
  [ 1.  1.  1.  1.]
  [ 1.  1.  1.  1.]]

 [[ 2.  2.  2.  2.]
  [ 2.  2.  2.  2.]
  [ 2.  2.  2.  2.]]]
\end{pythonConsole}
\end{minipage}
\begin{minipage}{5cm}
\begin{pythonConsole}
>>> print(A.cumsum(axis=1))
[[[ 1.  1.  1.  1.]
  [ 2.  2.  2.  2.]
  [ 3.  3.  3.  3.]]

 [[ 1.  1.  1.  1.]
  [ 2.  2.  2.  2.]
  [ 3.  3.  3.  3.]]]
>>> print(A.cumsum(2))
[[[ 1.  2.  3.  4.]
  [ 1.  2.  3.  4.]
  [ 1.  2.  3.  4.]]

 [[ 1.  2.  3.  4.]
  [ 1.  2.  3.  4.]
  [ 1.  2.  3.  4.]]]
\end{pythonConsole}
\end{minipage}
\end{frame}
%_______________________________________________________________________________
%_______________________________________________________________________________
\begin{frame}[fragile]
\frametitle{N-dimensional array Object}
\framesubtitle{Sélection de sous-tableaux}
\begin{itemize}
 \item découpage, slicing, comme pour les séquences.
\end{itemize}
\begin{pythonConsole}
>>> A = numpy.array([[1, 2, 3, 4], [11, 12, 13, 14]])
>>> print(A)
[[ 1  2  3  4]
 [11 12 13 14]]
>>> print(A[1, :])
[11 12 13 14]
>>> print(A[:, 1:3])
[[ 2  3]
 [12 13]]
\end{pythonConsole}
\end{frame}
%_______________________________________________________________________________
%_______________________________________________________________________________
\begin{frame}[fragile]
\frametitle{N-dimensional array Object}
\framesubtitle{Sélection de sous-tableaux}
\begin{itemize}
 \item Comparaison et opérateurs logiques
 \item Opérations logiques pour sélectionner des éléments (masking)
 \item Récupérer les indices (where)
\end{itemize}
\begin{pythonConsole}
>>> A = numpy.array([[1, 2, 3, 4], [11, 12, 13, 14]])
>>> print(A)
[[ 1  2  3  4]
 [11 12 13 14]]
>>> B = A > 2
>>> print(B)
[[False False  True  True]
 [ True  True  True  True]]
>>> print(A[B])
[ 3  4 11 12 13 14]
>>> indices = numpy.where(B)
>>> print(indices[0])
array([0, 0, 1, 1, 1, 1])
>>> print(indices[1])
array([2, 3, 0, 1, 2, 3])
>>> (i, j) = numpy.where(A > 2)
\end{pythonConsole}
\end{frame}
%_______________________________________________________________________________
%_______________________________________________________________________________
\begin{frame}[fragile]
\frametitle{N-dimensional array Object}
\framesubtitle{Concaténations}
\begin{itemize}
 \item Concaténation horizontale (hstack)
 \item Concaténation verticale (vstack)
 \item concatenate (concatenate, kwarg axis)
\end{itemize}
\begin{pythonConsole}
>>> A = numpy.array([[1, 2], [11, 12]])
>>> B = numpy.array([[3, 4], [13, 14]])
>>> C = numpy.vstack((A, B))
>>> print(C)
[[ 1  2]
 [11 12]
 [ 3  4]
 [13 14]]
>>> D = numpy.hstack((A, B, A, A))
>>> print(D)
[[ 1  2  3  4  1  2  1  2]
 [11 12 13 14 11 12 11 12]]
>>> E = numpy.concatenate((A, B, A, A), axis=1)
>>> print(E)
[[ 1  2  3  4  1  2  1  2]
 [11 12 13 14 11 12 11 12]]
\end{pythonConsole}
\end{frame}
%_______________________________________________________________________________
%_______________________________________________________________________________
\begin{frame}[fragile]
\frametitle{N-dimensional array Object}
\framesubtitle{Structured Arrays}
\begin{itemize}
 \item Possibilité de mettre des éléments de types différents.  
 \item Possibilité de tableaux structurés \dots
\end{itemize}
\begin{pythonConsole}
>>> A = numpy.array([["a", 1], ["b", 2]], dtype="object")
>>> print(A)
[['a' 1]
 ['b' 2]]
>>> print(A.dtype)
object

>>> A = numpy.array([(1, "abc"), (2, "def")], dtype=[("index", "int"), 
	("name", "S8")])
>>> print(A)
[(1, 'abc') (2, 'def')]
>>> A["index"]
array([1, 2])
>>> A["name"]
array(['abc', 'def'], 
      dtype='|S8')
\end{pythonConsole}
\end{frame}
%_______________________________________________________________________________
%_______________________________________________________________________________
\subsubsection{save/load}
%_______________________________________________________________________________
%_______________________________________________________________________________
\begin{frame}[fragile]
\frametitle{Sauvegarde et lecture de données}
\framesubtitle{}
\begin{itemize}
 \item Enregistrement d'un tableau (save) dans un fichier ".npy"
 \item Enregistrement compressé de plusieurs tableaux (savez) au format ".npz"
 \item Lecture (load) des fichiers ".npy", ".npz"
\end{itemize}
\begin{pythonConsole}
>>> A = numpy.array([[1, 2], [11, 12]])
>>> numpy.save("save_A", A)
>>> del(A)
>>> A = numpy.load("save_A.npy")
>>> print(A)
[[ 1  2]
 [11 12]]
>>> A = numpy.array([[1, 2], [11, 12]])
>>> B = numpy.array([[21, 22], [31, 32]])
>>> numpy.savez("save_AB", tab1=A, B=B)
>>> del(A, B)
>>> data = numpy.load("save_AB.npz")
>>> print(data["tab1"])
[[ 1  2]
 [11 12]]
>>> print(data["B"])
[[21 22]
 [31 32]]
\end{pythonConsole}
\end{frame}
%_______________________________________________________________________________
%_______________________________________________________________________________
\begin{frame}[fragile]
\frametitle{Sauvegarde et lecture de données txt}
\framesubtitle{}
\begin{itemize}
 \item Lecture de fichier texte ".txt" (load ou loadtxt) 
 \item Enregistrement (savetxt) 
\end{itemize}
\begin{pythonConsole}
>>> A = numpy.loadtxt("data.txt")
>>> A
array([[  1.,   2.,   3.,   4.,   5.],
       [ 11.,  12.,  13.,  14.,  15.],
       [ 21.,  22.,  23.,  24.,  25.]])
>>> numpy.savetxt("data.txt", A)
\end{pythonConsole}
\end{frame}
%_______________________________________________________________________________
%_______________________________________________________________________________
\subsubsection{matrix}
%_______________________________________________________________________________
%_______________________________________________________________________________
\begin{frame}[fragile]
\frametitle{Matrix}
\framesubtitle{Définition}
\begin{itemize}
 \item Classe héritée de ndarray avec ndim = 2.  
\end{itemize}
\begin{pythonConsole}
>>> help(numpy.matrix)
class matrix(numpy.ndarray)
 |  matrix(data, dtype=None, copy=True)
 |  
 |  Returns a matrix £from£ an array-like object, £or from£ a string of data.
 |  A matrix £is£ a specialized 2-D array that retains its 2-D nature
 |  through operations.  It has certain special operators, such as £`££`£*£`££`£
 |  (matrix multiplication) £and£ £`££`£**£`££`£ (matrix power).
...
\end{pythonConsole}
\end{frame}
%_______________________________________________________________________________
%_______________________________________________________________________________
\begin{frame}[fragile]
\frametitle{Matrix}
\framesubtitle{Saisie}
\begin{itemize}
 \item Saisie directe de type ndarray avec des listes imbriquées. 
 \item Possibilité de saisie type Matlab.
\end{itemize}
\begin{pythonConsole}
>>> A = numpy.matrix([[1, 2], [11, 12]])
>>> print(A)
[[ 1  2]
 [11 12]]
>>> type(A)
 <class 'numpy.matrixlib.defmatrix.matrix'>
>>> A = numpy.matrix("[1, 2, 3, 4; 11, 12, 13, 14]")
>>> print(A)
[[ 1  2  3  4]
 [11 12 13 14]]
\end{pythonConsole}
\end{frame}
%_______________________________________________________________________________
%_______________________________________________________________________________
\begin{frame}[fragile]
\frametitle{Matrix}
\framesubtitle{Multiplication et exposant}
\begin{itemize}
 \item Produit de matrices (*)
 \item Exposant de matrice (**)
\end{itemize}
\begin{pythonConsole}
>>> A = numpy.matrix([[1, 2], [11, 12]])
>>> B = numpy.matrix([[3, 4], [13, 14]])
>>> print(A * B)
[[ 29  32]
 [189 212]]
>>> print(A ** 2)
[[ 23  26]
 [143 166]]
\end{pythonConsole}
\end{frame}
%_______________________________________________________________________________
%_______________________________________________________________________________
\begin{frame}[fragile]
\frametitle{Matrix}
\framesubtitle{Opérateurs matriciels courants}
\begin{itemize}
 \item Transposition (T)
 \item Inversion (I)
 \item Opérateur Hermitien (H)
\end{itemize}
\begin{pythonConsole}
>>> A = numpy.matrix([[1, 2], [11, 12]])
>>> print(A)
[[ 1  2]
 [11 12]]
>>> print(A.T)
[[ 1 11]
 [ 2 12]]
>>> print(A.I)
print(A.I)
[[-1.2  0.2]
 [ 1.1 -0.1]]
>>> B = numpy.matrix([[1, 2+1j], [11+1j, 12]])
>>> print(B)
[[  1.+0.j   2.+1.j]
 [ 11.+1.j  12.+0.j]]
>>> print(B.H)
[[  1.-0.j  11.-1.j]
 [  2.-1.j  12.-0.j]]
\end{pythonConsole}
\end{frame}
%_______________________________________________________________________________
%_______________________________________________________________________________
\subsubsection{subpackages}
%_______________________________________________________________________________
%_______________________________________________________________________________
\begin{frame}[fragile]
\frametitle{Autres opérations d'algèbre linéaire}
\framesubtitle{Sous paquet linalg}
\begin{itemize}
 \item Interface vers Lapack (numpy.linalg)
\end{itemize}
\begin{pythonConsole}
>>> help(numpy.linalg)
...
    Linear algebra basics:
    
    - norm            Vector £or£ matrix norm
    - inv             Inverse of a square matrix
    - solve           Solve a linear system of equations
    - det             Determinant of a square matrix
    - lstsq           Solve linear least-squares problem
    - pinv            Pseudo-inverse (Moore-Penrose)...
    - matrix_power    Integer power of a square matrix
    
    Eigenvalues £and£ decompositions:
    
    - eig             Eigenvalues £and£ vectors of a square matrix
    - eigh            Eigenvalues £and£ eigenvectors of a Hermitian matrix
    - eigvals         Eigenvalues of a square matrix
    - eigvalsh        Eigenvalues of a Hermitian matrix
    - qr              QR decomposition of a matrix
    - svd             Singular value decomposition of a matrix
    - cholesky        Cholesky decomposition of a matrix
    
    Tensor operations:
    
    - tensorsolve     Solve a linear tensor equation
    - tensorinv       Calculate an inverse of a tensor
...
\end{pythonConsole}
\end{frame}
%_______________________________________________________________________________
%_______________________________________________________________________________
\begin{frame}[fragile]
\frametitle{Autres paquets de numpy}
\framesubtitle{}
\begin{pythonConsole}
>>> help(numpy)
...
    doc
        Topical documentation on broadcasting, indexing, etc.
    lib
        Basic functions used by several sub-packages.
    random
        Core Random Tools
    linalg
        Core Linear Algebra Tools
    fft
        Core FFT routines
    polynomial
        Polynomial tools
    testing
        Numpy testing tools
    f2py
        Fortran to Python Interface Generator.
    distutils
        Enhancements to distutils with support for
        Fortran compilers support and more.
...
\end{pythonConsole}
\end{frame}
%_______________________________________________________________________________
%_______________________________________________________________________________
\begin{frame}[fragile]
\frametitle{Autres paquets de numpy}
\framesubtitle{Random}
\begin{itemize}
 \item Sous paquet random : générateurs de nombres aléatoires. 
\end{itemize}
\begin{pythonConsole}
>>> numpy.random.seed(0)
>>> A = numpy.random.randn(2, 3, 4)
>>> print(A)
[[[ 1.76405235  0.40015721  0.97873798  2.2408932 ]
  [ 1.86755799 -0.97727788  0.95008842 -0.15135721]
  [-0.10321885  0.4105985   0.14404357  1.45427351]]

 [[ 0.76103773  0.12167502  0.44386323  0.33367433]
  [ 1.49407907 -0.20515826  0.3130677  -0.85409574]
  [-2.55298982  0.6536186   0.8644362  -0.74216502]]]
\end{pythonConsole}
\end{frame}
%_______________________________________________________________________________
%_______________________________________________________________________________
\subsection{SciPy library}
\subsubsection{subpackages}
%_______________________________________________________________________________
%_______________________________________________________________________________
\begin{frame}[fragile]
\frametitle{Scipy library}
\framesubtitle{Librairie scientifique}
\begin{itemize}
 \item \myFig{height=0.5cm}{./fig/scipylib-logo.png} \, \url{http://www.scipy.org/scipylib/index.html}
 \item \emph{It provides many user-friendly and efficient numerical routines such as routines for numerical integration and optimization.}
\end{itemize}
\begin{pythonConsole}
>>> import scipy
>>> help(scipy)
	...
     cluster                      --- Vector Quantization / Kmeans
     fftpack                      --- Discrete Fourier Transform algorithms
     integrate                    --- Integration routines
     interpolate                  --- Interpolation Tools
     io                           --- Data input £and£ output
     lib                          --- Python wrappers to external libraries
     lib.lapack                   --- Wrappers to LAPACK library
     linalg                       --- Linear algebra routines
     misc                         --- Various utilities that don£'£t have
                                      another home.
     ndimage                      --- n-dimensional image package
     odr                          --- Orthogonal Distance Regression
     optimize                     --- Optimization Tools
     signal                       --- Signal Processing Tools
     sparse                       --- Sparse Matrices
     sparse.linalg                --- Sparse Linear Algebra
    ...
\end{pythonConsole}
\end{frame}
%_______________________________________________________________________________
%_______________________________________________________________________________
\begin{frame}[fragile]
\frametitle{Scipy}
\framesubtitle{}
\begin{itemize}
 \item Optimisation, Intégration, Interpolation, Algèbre linéaire, Algèbre linaire creuse, Signal, Image, Statistiques, Fonctions spéciales ($\Gamma$, $\psi$)\dots
\end{itemize}
\begin{pythonConsole}
    ...
     sparse.linalg.dsolve         --- Linear Solvers
     sparse.linalg.dsolve.umfpack --- :Interface to the UMFPACK library:
                                      Conjugate Gradient Method (LOBPCG)
     sparse.linalg.eigen.lobpcg   --- Locally Optimal Block Preconditioned
                                      Conjugate Gradient Method (LOBPCG) [*]
     special                      --- Airy Functions [*]
     lib.blas                     --- Wrappers to BLAS library [*]
     sparse.linalg.eigen          --- Sparse Eigenvalue Solvers [*]
     stats                        --- Statistical Functions [*]
     lib                          --- Python wrappers to external libraries
                                      [*]
     lib.lapack                   --- Wrappers to LAPACK library [*]
     integrate                    --- Integration routines [*]
     ndimage                      --- n-dimensional image package [*]
     linalg                       --- Linear algebra routines [*]
     spatial                      --- Spatial data structures £and£ algorithms
     special                      --- Airy Functions
     stats                        --- Statistical Functions
    ...
\end{pythonConsole}
\end{frame}
%_______________________________________________________________________________
%_______________________________________________________________________________
\subsubsection{load Matlab data}
%_______________________________________________________________________________
%_______________________________________________________________________________
\begin{frame}[fragile]
\frametitle{Scipy}
\framesubtitle{lecture de fichiers Matlab}
\begin{itemize}
 \item Exemple, lectures de fichiers Matlab dans le subpackage io (scipy.io)
\end{itemize}

\begin{pythonConsole}
>>> import scipy.io
>>> data = scipy.io.loadmat('file.mat')
\end{pythonConsole}
\end{frame}
%_______________________________________________________________________________
%_______________________________________________________________________________
\subsection{Matplotlib}
%_______________________________________________________________________________
%_______________________________________________________________________________
\begin{frame}[fragile]
\frametitle{Matplotlib}
\framesubtitle{}
\begin{itemize}
 \item \myFig{height=0.5cm}{./fig/matplotlib-logo.png} : \url{http://www.matplotlib.org}
 \item \emph{matplotlib is a python 2D plotting library which produces publication quality figures in a variety of hardcopy formats and interactive environments across platforms. }
 \item Contient des classes : programmation orientée objets avec différents backends pour différentes interfaces graphiques, graphical user interfaces (GUI) : agg, gtk, qt, svg, ps, pdf\dots
 \item Contient des procédures pour faciliter l'accès à ces classes : matlab style. 
\end{itemize}
\end{frame}
%_______________________________________________________________________________
%_______________________________________________________________________________
\subsubsection{pyplot}
%_______________________________________________________________________________
%_______________________________________________________________________________
\begin{frame}[fragile]
\frametitle{Matplotlib}
\framesubtitle{pyplot}
\begin{itemize}
 \item Fonctions procédurales dans le sous-paquet pyplot (matplotlib.pyplot)
\end{itemize}
\begin{pythonConsole}
>>> import matplotlib.pyplot
>>> t = numpy.arange(0, 10, 0.01)
>>> x = numpy.sin(2 * numpy.pi * 3 * t)
>>> matplotlib.pyplot.plot(t, x)
>>> matplotlib.pyplot.xlabel("time (s)")
>>> matplotlib.pyplot.ylabel("x ($\mu V$)")
>>> matplotlib.pyplot.show()
\end{pythonConsole}
\begin{center}
 \includegraphics[width=6cm]{./fig/matplotlibSinus.pdf}
\end{center}
\end{frame}
%_______________________________________________________________________________
%_______________________________________________________________________________
\subsubsection{savefig}
%_______________________________________________________________________________
%_______________________________________________________________________________
\begin{frame}[fragile]
\frametitle{Matplotlib}
\framesubtitle{Saving figures}
\begin{itemize}
 \item Sauvegarde manuelle à partir de la fenêtre ouverte sur l'icône save.  
 \item sauvegarde en ligne de commande (savefig) sans passage par un affichage à l'écran.  
\end{itemize}
\begin{pythonConsole}
...
>>> matplotlib.pyplot.ylabel("x ($\mu V$)")
>>> # matplolib.pyplot.show()
>>> matplotlib.pyplot.savefig("./sinus.ps")
>>> matplotlib.pyplot.savefig("./sinus.pdf")
>>> matplotlib.pyplot.savefig("./sinus.svg")
>>> matplotlib.pyplot.savefig("./sinus.tiff")
>>> matplotlib.pyplot.savefig("./sinus.png")
>>> matplotlib.pyplot.savefig("./sinus.jpg")
\end{pythonConsole}
\end{frame}
%_______________________________________________________________________________
%_______________________________________________________________________________
\subsubsection{pylab}
%_______________________________________________________________________________
%_______________________________________________________________________________
\begin{frame}[fragile]
\frametitle{Matplotlib}
\framesubtitle{Pylab}
\begin{itemize}
 \item Fonctions à la Matlab dans le sous-paquet pylab (matplotlib.pylab)
 \item Beaucoup de fonctions sous formes abrégées \dots
\end{itemize}
\begin{minipage}[c]{5cm}
\begin{pythonConsole}
>>> import matplotlib.pylab as mpl
>>> len(dir(mpl))
955

>>> from pylab import *
>>> t = arange(0, 10, 0.01)
>>> f = arange(0, 5, 0.005)
>>> x = sin(2 * pi * f * t)
>>> subplot(2, 1, 1)
>>> plot(t, x * 5)
>>> plot(t, f)
>>> subplot(2, 1, 2)
>>> res = specgram(x, NFFT=64, 
... Fs=100, noverlap=8)
>>> show()
\end{pythonConsole}
\end{minipage}
\begin{minipage}[c]{5cm}
 \includegraphics[width=5cm]{./fig/specgram.png}
\end{minipage}
\end{frame}
%_______________________________________________________________________________
%_______________________________________________________________________________
\begin{frame}[fragile]
\frametitle{Matplotlib}
\framesubtitle{Pylab}
\begin{minipage}{5cm}
\begin{pythonConsole}
>>> (X, Y) = meshgrid(linspace(-2, 2, \
... 500), linspace(-2, 2, 500))
>>> Z = X + Y * 1j
>>> for k in range(76): 
...  Z -= (Z / 3 - 1) / (3 * Z ** 2)
>>> close("all")	
>>> imshow(angle(Z))
>>> # savefig('/MonChemin/Lenom.pdf')
>>> show()
\end{pythonConsole}
\end{minipage}
\begin{minipage}{5cm}
\includegraphics[width=5cm]{fig/fractal.png}
\end{minipage}
\end{frame}
%_______________________________________________________________________________
%_______________________________________________________________________________
\begin{frame}
\frametitle{Matplotlib}
\framesubtitle{Exemples de Nicolas Le Bihan}
\begin{minipage}{0.4\linewidth}
\begin{figure}
\includegraphics[width=6.5cm,height=5.5cm]{fig/BrownSphere.png}
\end{figure}
\end{minipage}
\hspace{1cm}
\begin{minipage}{0.4\linewidth}
\includegraphics[width=5.5cm,height=4.5cm]{fig/Distrib.png}
\end{minipage}
\end{frame}
%_______________________________________________________________________________
%_______________________________________________________________________________
\subsection{IPython}
%_______________________________________________________________________________
%_______________________________________________________________________________
\begin{frame}[fragile]
\frametitle{IPython}
\framesubtitle{}
\begin{itemize}
 \item \myFig{height=0.5cm}{./fig/ipython-logo.png} \, \url{http://www.scipy.org/scipylib/index.html}
 \item \emph{Enhanced python console}
 \item Attention, ce n'est pas un paquet mais une console améliorée !
 \item Accessible à partir d'un terminal (ipython) 
 \item Complétion automatique avec la touche tab
 \item Fonctions magiques (magic)
\end{itemize}

\begin{shell}
$ ipython /* $ */
\end{shell} 

\begin{pythonConsole}
Python 2.7.3 | 64-bit | (default, Jun 14 2013, 18:17:36) 
Type "copyright", "credits" £or£ "license" £for£ more information.

IPython 2.2.0 -- An enhanced Interactive Python.
?         -> Introduction £and£ overview of IPython£'£s features.
%quickref -> Quick reference.
help      -> Python£'£s own help system.
object?   -> Details about 'object', use 'object??' £for£ extra details.

In [1]: %pylab
Using matplotlib backend: MacOSX
Populating the interactive namespace £from£ numpy £and£ matplotlib


In [2]: 
\end{pythonConsole}
\end{frame}
%_______________________________________________________________________________
%_______________________________________________________________________________
\subsubsection{ipython notebook}
%_______________________________________________________________________________
%_______________________________________________________________________________
\begin{frame}[fragile]
\frametitle{IPython}
\framesubtitle{ipython notebook}
\begin{itemize}
 \item Accessible à partir d'un terminal (ipython notebook) 
 \item Ouverture et création de notebooks (*.pynb)
\end{itemize}
\begin{center}
 \myFig{width=8cm}{./fig/ipython_notebook_server.png}
\end{center}
\end{frame}
%_______________________________________________________________________________
%_______________________________________________________________________________
\begin{frame}[fragile]
\frametitle{IPython}
\framesubtitle{ipython notebook}
\begin{itemize}
 \item Les cellules exécutent du code ou formatent du texte
 \item Chargement des fonctions pylab avec affichage dans le notebook : \%pylab inline 
\end{itemize}
\begin{center}
 \myFig{width=8cm}{./fig/ipython_notebook_p1.png}
\end{center}
\end{frame}
%_______________________________________________________________________________
%_______________________________________________________________________________
\begin{frame}[fragile]
\frametitle{IPython}
\framesubtitle{ipython notebook}
\begin{itemize}
 \item Saisie et affichage en HTML et LaTeX, entre autres.   
\end{itemize}
\begin{center}
 \myFig{width=8cm}{./fig/ipython_notebook_p2.png}
\end{center}
\end{frame}
%_______________________________________________________________________________
%_______________________________________________________________________________
\subsection{Mayavi}
%_______________________________________________________________________________
%_______________________________________________________________________________
\begin{frame}
\frametitle{Mayavi}
\framesubtitle{}
\frameCC{%
\begin{itemize}
 \item \url{http://docs.enthought.com/mayavi/mayavi}
 \item \url{https://github.com/enthought/mayavi}
 \item Manipulation des objets 3D améliorée (objet vtk). 
\end{itemize}
}
{\myFig{width=8cm}{./fig/Mayavi_site.png}}
\end{frame}
%_______________________________________________________________________________
%_______________________________________________________________________________
\begin{frame}
\frametitle{Mayavi}
\framesubtitle{}
\myFigCentered{width=8cm}{./fig/Mayavi.png}
\end{frame}
%_______________________________________________________________________________
%_______________________________________________________________________________
\begin{frame}[fragile]
\frametitle{Mayavi}
\framesubtitle{import mayavi.engine}
\begin{itemize}
 \item Dans une console Python : import mayavi. \dots
\end{itemize}
\begin{pythonConsole}
>>> from mayavi.api import Engine
>>> engine = Engine()
>>> engine.start()
>>> engine.new_scene()
>>> from mayavi.sources.parametric_surface import ParametricSurface
>>> parametric_surface1 = ParametricSurface()
>>> scene = engine.scenes[0]
>>> engine.add_source(parametric_surface1, scene)
>>> from mayavi.modules.surface import Surface
>>> surface1 = Surface()
>>> engine.add_filter(surface1,parametric_surface1)
\end{pythonConsole}
\end{frame}
%_______________________________________________________________________________
%_______________________________________________________________________________
\subsection{scikit-learn}
%_______________________________________________________________________________
%_______________________________________________________________________________
\begin{frame}[fragile]
\frametitle{Scikit-learn}
\framesubtitle{}
\begin{itemize}
 \item \url{http://scikit-learn.org/} 
 \item \url{https://github.com/scikit-learn/scikit-learn} 
\end{itemize}
\myFigCentered{width=8cm}{./fig/scikit-learn.png}
\end{frame}
%_______________________________________________________________________________
%_______________________________________________________________________________
\begin{frame}[fragile]
\frametitle{Scikit-learn}
\framesubtitle{Help content}
\begin{pythonConsole}
>>> import sklearn
>>> help(sklearn)

DESCRIPTION
    Machine learning module for Python
    ==================================
    
    sklearn £is£ a Python module integrating classical machine
    learning algorithms £in£ the tightly-knit world of scientific Python
    packages (numpy, scipy, matplotlib).
    
    It aims to provide simple £and£ efficient solutions to learning problems
    that are accessible to everybody £and£ reusable £in£ various contexts:
    machine-learning as a versatile tool £for£ science and engineering.
    
    See http://scikit-learn.org £for£ complete documentation.

PACKAGE CONTENTS
    __check_build (package)
    _build_utils
    _hmmc
    _isotonic
    base
    cluster (package)
    covariance (package)
    cross_decomposition (package)
    cross_validation
    datasets (package)
    decomposition (package)
    dummy
    ensemble (package)
    externals (package)
\end{pythonConsole}
\end{frame}
%_______________________________________________________________________________
%_______________________________________________________________________________
\begin{frame}[fragile]
\frametitle{Scikit-learn}
\framesubtitle{Help content}
\begin{pythonConsole}
    feature_extraction (package)
    feature_selection (package)
    gaussian_process (package)
    grid_search
    hmm
    isotonic
    kernel_approximation
    lda
    learning_curve
    linear_model (package)
    kernel_approximation
    lda
    learning_curve
    linear_model (package)
    manifold (package)
    metrics (package)
    mixture (package)
    multiclass
    naive_bayes
    neighbors (package)
    neural_network (package)
    pipeline
    pls
    preprocessing (package)
    qda
    random_projection
    semi_supervised (package)
    setup
    svm (package)
    tests (package)
    tree (package)
    utils (package)
\end{pythonConsole}
\end{frame}
%_______________________________________________________________________________
%_______________________________________________________________________________
\begin{frame}[fragile]
\frametitle{Scikit-learn}
\framesubtitle{Exemple de Classifieur : Linear Discriminant Analysis}
\begin{pythonConsole}
>>> from sklearn.datasets import load_iris
>>> from sklearn.lda import LDA
>>> dic = load_iris()
>>> x = dic["data"]
>>> y = dic["target"]
>>> nClasses = unique(y).size
>>> clf = LDA()
>>> clf.fit(x[:, :2], y)
>>> (mx1, mx2) = = meshgrid(linspace(4, 8, 100), linspace(2, 5, 100))
>>> Z = clf.predict(vstack((mx1.flatten().T, mx2.flatten())).T)
>>> pcolor(mx1, mx2, Z.reshape(mx1.shape), alpha=0.1)
>>> colors = ["b", "g", "r"]
>>> for i in range(nClasses): 
>>>   scatter(x[y==i, 0], x[y==i, 1], c=colors[i])
>>> axis(xmin=4, xmax=8, ymin=2, ymax=5)
>>> show()
\end{pythonConsole}
\myFigCentered{width=3cm}{./fig/sklearn_lda.png}
\end{frame}
%_______________________________________________________________________________
%_______________________________________________________________________________
\subsection{Autres ressources : exemple C}
%_______________________________________________________________________________
\begin{frame}[fragile]
\frametitle{Importation de bibliothèques de fonctions écrites en C}
\framesubtitle{Exemple using module ctypes}
\begin{python}
import ctypes
libName = './clz.lib'
libCLZ = ctypes.CDLL(libName)
clz_c = libCLZ.clz
clz_c.restype = ctypes.c_uint
sequence = numpy.ctypeslib.ndpointer(dtype=numpy.int) 
clz_c.argtypes = ([sequence, ctypes.c_uint])
# conversion of s into sequence with numpy.asarray
c = clz_c(numpy.asarray(s, dtype='int'), n)
\end{python}
\end{frame}
%_______________________________________________________________________________


%===============================================================================
\section{Distributions et Environnements de travail}
%===============================================================================
%_______________________________________________________________________________
\subsection{Python Software Foundation}
%...............................................................................
\begin{frame}
\frametitle{Python.org}
\framesubtitle{Official}
\frameCC{%
\begin{itemize}
 \item Python 2.7 ou 3.x téléchargeable sur \urlPython.
 \item Livré uniquemenent avec la bibliothèque standard. 
 \item Inclus l'interpréteur Python natif accessible à partir de l'environnement système. 
 \item Parfait pour tester des petits bouts de codes. 
\end{itemize}
}%
{\myFig{width=7cm}{./fig/interpreter.png}}
\end{frame}
%...............................................................................
%...............................................................................
\begin{frame}
\frametitle{Python IDLE}
\framesubtitle{official}
\frameCC{%
\begin{itemize}
 \item Python livré avec un Integrated DeveLopment Environment (IDLE) : 
 \begin{itemize}
 \footnotesize
  \item Une console Python : coloration automatique, autocomplétion \dots 
  \item Un éditeur de texte : indentation automatique, coloration syntaxique, debuggueur, \dots
 \end{itemize}
\end{itemize}
}%
{\myFig{width=7cm}{./fig/Idle.png}}
\end{frame}
%_______________________________________________________________________________
%_______________________________________________________________________________
\subsection{Enthought}
\begin{frame}
\frametitle{Enthought Canopy}
\begin{itemize}
 \item \myFig{height=0.5cm}{./fig/enthought-logo.png}\ : \url{https://www.enthought.com/products/canopy/} 
 \item Contient Python et +100 librairies orientées applications scientifiques. 
 \item Canopy express \em{free license for all users}.
\end{itemize}
\end{frame}
%_______________________________________________________________________________
%_______________________________________________________________________________
\subsection{Enthought}
\begin{frame}
\frametitle{Enthought Canopy}
\begin{itemize}
\footnotesize
 \item \myFig{height=0.5cm}{./fig/enthought-logo.png}\ : \url{https://www.enthought.com/products/canopy/} 
 \item \em{Easy installation and update}.
\end{itemize}
\centering\myFig{width=7cm}{./fig/enthought_canopy_manager.png}
\end{frame}
%_______________________________________________________________________________
%_______________________________________________________________________________
\subsection{Enthought}
\begin{frame}
\frametitle{Enthought Canopy}
\begin{itemize}
\footnotesize
 \item \myFig{height=0.5cm}{./fig/enthought-logo.png}\ : \url{https://www.enthought.com/products/canopy/} 
 \item + Jupyter notebook + Canopy editor. 
\end{itemize}
\centering\myFig{width=7cm}{./fig/enthought_canopy_editor.png}
\end{frame}
%_______________________________________________________________________________
%_______________________________________________________________________________
\subsection{Anaconda}
%_______________________________________________________________________________
%_______________________________________________________________________________
\begin{frame}
\frametitle{Anaconda}
\framesubtitle{}
\begin{itemize}
 \item \myFig{height=0.5cm}{./fig/anaconda_logo_web.png}\, \url{https://www.anaconda.com}
\end{itemize}
\only<1>{\centering\myFig{width=8cm}{./fig/anaconda_distribution.png}}
\only<2>{\centering\myFig{width=8cm}{./fig/anaconda_launcher.png}}
\end{frame}
%_______________________________________________________________________________
%_______________________________________________________________________________
\begin{frame}
\frametitle{Spyder}
\framesubtitle{}
\frameCC{%
\begin{itemize}
\footnotesize
\item Spyder (multiplateforme) : environnement de type Matlab pour Python. 
\item \url{https://github.com/spyder-ide/spyder}
\end{itemize}
}
{\myFig{width=8cm}{./fig/Spyder.png}}
\end{frame}
%_______________________________________________________________________________
%_______________________________________________________________________________
\subsection{Autres}
%_______________________________________________________________________________
%_______________________________________________________________________________
\begin{frame}
\frametitle{Autres distributions}
\begin{itemize}
\item SageMath \myFig{height=0.5cm}{./fig/sage_logo.png} \, \url{http://www.sagemath.org}
\item PythonXY \myFig{height=0.5cm}{./fig/PythonXYLogo.png}\ \url{https://code.google.com/p/pythonxy/}  
\item ? 
\end{itemize}
\end{frame}
%_______________________________________________________________________________
%_______________________________________________________________________________
\begin{frame}
\frametitle{Editeurs de texte et Integrated Development Environment (IDE)}
\framesubtitle{}
\frameCC{%
\begin{itemize}
\small
\item Tout éditeur de texte avec coloration syntaxique : Emacs, Vim, jEdit, gedit, Textpad, Eclipse, Atom, Visual Studio... 
\end{itemize}
}
{\myFig{width=7cm}{./fig/jEdit.png}}
\end{frame}
%_______________________________________________________________________________
%===============================================================================
%===============================================================================
\section{Conclusion}
%_______________________________________________________________________________
\subsection{Python vs Matlab ?}
%...............................................................................
\begin{frame}
\frametitle{Python vs Matlab}
\frameLR{6cm}{4cm}{%
\begin{itemize}
 \footnotesize
 \item Outils équivalents : matrices vs ndarray, console, script, graphisme, GUI, cell mode vs jupyter notebook \dots
 \item Matlab, Matrix Laboratory, a des bibliothèques d'algèbre linéaire plus rapide que Numpy ou Scipy (sauf avec certaines distributions payantes). 
 \item Python est un langage de programmation.   
 \item Python est plus proche du code C pour prototyper.
 \item Chargement des modules à la volée en Python. 
 \item Python est gratuit. 
 \item Votre code en Python peut être utilisé gratuitement. 
\end{itemize}
}
{%
\myFig{width=4cm}{./fig/matlabLicenseManagerError4.png}
}
\end{frame}
%...............................................................................
%_______________________________________________________________________________

%===============================================================================


\appendix
\section{classe et variable 'self'}
%...............................................................................
\begin{frame}[fragile]
\frametitle{Classe exemples avec self}
\begin{itemize}
 \item Dans le corps de la classe, 'self' n'est pas défini. 
\end{itemize}
\begin{pythonConsole}
class Canard(): 
...     self.a = 10
... 
Traceback (most recent call last):
 File £"£<stdin>£"£, line 1, £in£ <module>
 File £"£<stdin>£"£, line 2, £in£ Canard
NameError: name £'self'£ £is not£ defined
\end{pythonConsole}
\end{frame}
%...............................................................................
%...............................................................................
\begin{frame}[fragile]
\frametitle{Classe exemples avec self}
\begin{itemize}
 \item Dans une méthode seul self.nomAttribut est accessible. 
\end{itemize}
\begin{pythonConsole}
class Canard(): 
...     a = 10
...     def __init__(self): 
...             self.b = 100
...             print(self.a)
...             print(b)
... 
riri = Canard()
10
Traceback (most recent call last):
 File £"£<stdin>£"£, line 1, £in£ <module>
 File £"£<stdin>£"£, line 6, £in£ __init__
NameError: £global£ name £'b'£ £is not£ defined
\end{pythonConsole}
\end{frame}
%...............................................................................
%...............................................................................
\begin{frame}[fragile]
\frametitle{Classe exemples avec self}
\begin{itemize}
 \item 'self' est conventionnel. 
 \item Le premier paramètre d'une méthode est considéré comme l'objet lui même. 
\end{itemize}
\begin{pythonConsole}
class Canard(): 
...     def __init__(obj): 
...             obj.a = 10
...             print(obj.a)
... 
riri = Canard()
10
\end{pythonConsole}
\end{frame}
%...............................................................................
%...............................................................................
\begin{frame}[fragile]
\frametitle{Classe exemples avec self}
\begin{itemize}
 \item Possibilité de rajouter des arguments optionnels lors de l'instanciation d'un objet. 
\end{itemize}
\begin{pythonConsole}
class Canard(): 
...     def __init__(self, patte=2):
...             self.a = patte
...             print(self.a)
... 
riri = Canard(patte=3)
3
\end{pythonConsole}
\end{frame}
%...............................................................................
%...............................................................................
\begin{frame}[fragile]
\frametitle{Transposition NDarray exemple 3D}
\begin{itemize}
 \item A.shape (1, 2, 3, 4)
 \item A.T.shape (4, 3, 2, 1) by default. 
 \item can set the axes order. see help(numpy.matrix.transpose)
\end{itemize}
\begin{pythonConsole}
>>> A = numpy.array([[[[1, 2, 3, 4], [11, 12, 13, 14], [21, 22, 23, 24]], 
	[[101, 102, 103, 104], [111, 112, 113, 114], [121, 122, 123, 124]]]])

>>> A
array([[[[  1,   2,   3,   4],
         [ 11,  12,  13,  14],
         [ 21,  22,  23,  24]],

        [[101, 102, 103, 104],
         [111, 112, 113, 114],
         [121, 122, 123, 124]]]])

>>> A.T
array([[[[  1],
         [101]],
         ...
        [[ 24],
         [124]]]])
\end{pythonConsole}
\end{frame}
%...............................................................................
%...............................................................................
\begin{frame}[fragile]
\frametitle{Identity of an object}
\begin{itemize}
 \item Return the identity of an object.
    This is guaranteed to be unique among simultaneously existing objects.
    (CPython uses the object's memory address.)
\end{itemize}
\begin{pythonConsole}
>>> a = 123
>>> id(a)
4297541856
>>> id(123)
4297541856
\end{pythonConsole}
\end{frame}
%...............................................................................
%...............................................................................
\begin{frame}[fragile]
\frametitle{Order of definition in a module}
\begin{itemize}
 \item L'ordre de définition des fonctions n'est pas important à partir du moment où lors de leurs utilisations, celles-ci ont été définies. 
 C'est particulièrement fréquents dans les modules. 
\end{itemize}
\begin{pythonConsole}
>>> def f(a): 
...     g(a)
>>> def g(b): 
...     print(b)
>>> a='abc'
>>> f(a)
abc
\end{pythonConsole}
\end{frame}
%...............................................................................

%...............................................................................
\begin{frame}[fragile]
\frametitle{Order of definition in a module}
\begin{itemize}
 \item l'ordre de définition des fonctions n'est pas important à partir du moment où lors de l'utilisation es focntions ont été définies. 
 Fonctionne dans la console ou dans un module. 
\end{itemize}
\begin{pythonConsole}
>>> def f(a): 
>>>     g(a)
>>> def g(b): 
>>>     print(b)
>>> a='abc'
>>> f(a)
abc
\end{pythonConsole}
\end{frame}
%...............................................................................

%...............................................................................
\begin{frame}[fragile]
\frametitle{slice and ellipsis}
\begin{itemize}
 \item l'ellipse  
\end{itemize}
\begin{pythonConsole}
>>> A = arange(24).reshape(2, 3, 4)
>>> A
array([[[ 0,  1,  2,  3],
        [ 4,  5,  6,  7],
        [ 8,  9, 10, 11]],

       [[12, 13, 14, 15],
        [16, 17, 18, 19],
        [20, 21, 22, 23]]])
>>> A[:,0]
array([[ 0,  1,  2,  3],
       [12, 13, 14, 15]])
>>> A[...,0]
array([[ 0,  4,  8],
       [12, 16, 20]])
\end{pythonConsole}
\end{frame}
%...............................................................................
%...............................................................................
\begin{frame}[fragile]
\frametitle{slice and ellipsis}
\begin{itemize}
 \item A[i, ...] est identique à A[i, :, :, etc.].
 \item A[i, :] est identique à A[i, :, :, etc. ], donc identique à ... 
 \item A[:, i, :] = A[..., i, :] dans l'exemple mais 
 \item A[..., 0] est différent de A[:, 0] car ici c'est A[:, 0, :] et non A[:, :, 0] ! 
\end{itemize}
\begin{pythonConsole}
>>> A = arange(12).reshape(2, 3, 2)
>>>  A
array([[[ 0,  1],
        [ 2,  3],
        [ 4,  5]],

       [[ 6,  7],
        [ 8,  9],
        [10, 11]]])
>>> A[:,0]                       >>> A[:,0,:] 
array([[0, 1],                   array([[0, 1],
       [6, 7]])                         [6, 7]]) 
>>> A[...,0]
array([[ 0,  2,  4],
       [ 6,  8, 10]])
\end{pythonConsole}
\end{frame}
%...............................................................................
%...............................................................................
\begin{frame}[fragile]
\frametitle{Zen of Python - Pythonic}
\begin{scriptsize}
Thanks to Branislav Gerazov (Branko) in regard to his friendly mail about this presentation: \emph{\dots one thing that I find awesome in Python it's its emphasis on aesthetics and readability. On this note I would add the Zen of Python ('import this') and the concept of being 'pythonic', and contrast it to Perl which is the most awful computer language in the world (I am recoding something now from Perl and that --- is really frustrating!). In this sense I've had students often saying they love Python \dots} 
\end{scriptsize}

\begin{pythonConsole}
>>> import this
The Zen of Python, by Tim Peters

Beautiful is better than ugly.
Explicit is better than implicit.
Simple is better than complex.
Complex is better than complicated.
Flat is better than nested.
...
>>> # a pythonic list comprehension for example !
>>> list_of_number = [(i, 2*i, 3*i) for i in range(10)]  
>>> list_of_number 
[(0, 0, 0), (1, 2, 3), (2, 4, 6), (3, 6, 9), (4, 8, 12), (5, 10, 15), 
(6, 12, 18), (7, 14, 21), (8, 16, 24), (9, 18, 27)]
\end{pythonConsole}
\end{frame}
%...............................................................................
%...............................................................................
\begin{frame}
\frametitle{Lien Patricia Ladret}
\begin{itemize}
 \item Cours d'une collègue Patricia Ladret pour Python via Anaconda: 
 \url{http://chamilo1.grenet.fr/ujf/courses/FAMILIARISATIONAVECPYTHONSUITEANACON/index.php}
\end{itemize}
\end{frame}
%...............................................................................

%____________________________________________________________________________
\end{document}
%============================================================================

