\section{classe et variable 'self'}
%...............................................................................
\begin{frame}[fragile]
\frametitle{Classe exemples avec self}
\begin{itemize}
 \item Dans le corps de la classe, 'self' n'est pas défini. 
\end{itemize}
\begin{pythonConsole}
class Canard(): 
...     self.a = 10
... 
Traceback (most recent call last):
 File £"£<stdin>£"£, line 1, £in£ <module>
 File £"£<stdin>£"£, line 2, £in£ Canard
NameError: name £'self'£ £is not£ defined
\end{pythonConsole}
\end{frame}
%...............................................................................
%...............................................................................
\begin{frame}[fragile]
\frametitle{Classe exemples avec self}
\begin{itemize}
 \item Dans une méthode seul self.nomAttribut est accessible. 
\end{itemize}
\begin{pythonConsole}
class Canard(): 
...     a = 10
...     def __init__(self): 
...             self.b = 100
...             print(self.a)
...             print(b)
... 
riri = Canard()
10
Traceback (most recent call last):
 File £"£<stdin>£"£, line 1, £in£ <module>
 File £"£<stdin>£"£, line 6, £in£ __init__
NameError: £global£ name £'b'£ £is not£ defined
\end{pythonConsole}
\end{frame}
%...............................................................................
%...............................................................................
\begin{frame}[fragile]
\frametitle{Classe exemples avec self}
\begin{itemize}
 \item 'self' est conventionnel. 
 \item Le premier paramètre d'une méthode est considéré comme l'objet lui même. 
\end{itemize}
\begin{pythonConsole}
class Canard(): 
...     def __init__(obj): 
...             obj.a = 10
...             print(obj.a)
... 
riri = Canard()
10
\end{pythonConsole}
\end{frame}
%...............................................................................
%...............................................................................
\begin{frame}[fragile]
\frametitle{Classe exemples avec self}
\begin{itemize}
 \item Possibilité de rajouter des arguments optionnels lors de l'instanciation d'un objet. 
\end{itemize}
\begin{pythonConsole}
class Canard(): 
...     def __init__(self, patte=2):
...             self.a = patte
...             print(self.a)
... 
riri = Canard(patte=3)
3
\end{pythonConsole}
\end{frame}
%...............................................................................
%...............................................................................
\begin{frame}[fragile]
\frametitle{Transposition NDarray exemple 3D}
\begin{itemize}
 \item A.shape (1, 2, 3, 4)
 \item A.T.shape (4, 3, 2, 1) by default. 
 \item can set the axes order. see help(numpy.matrix.transpose)
\end{itemize}
\begin{pythonConsole}
>>> A = numpy.array([[[[1, 2, 3, 4], [11, 12, 13, 14], [21, 22, 23, 24]], 
	[[101, 102, 103, 104], [111, 112, 113, 114], [121, 122, 123, 124]]]])

>>> A
array([[[[  1,   2,   3,   4],
         [ 11,  12,  13,  14],
         [ 21,  22,  23,  24]],

        [[101, 102, 103, 104],
         [111, 112, 113, 114],
         [121, 122, 123, 124]]]])

>>> A.T
array([[[[  1],
         [101]],
         ...
        [[ 24],
         [124]]]])
\end{pythonConsole}
\end{frame}
%...............................................................................
%...............................................................................
\begin{frame}[fragile]
\frametitle{Identity of an object}
\begin{itemize}
 \item Return the identity of an object.
    This is guaranteed to be unique among simultaneously existing objects.
    (CPython uses the object's memory address.)
\end{itemize}
\begin{pythonConsole}
>>> a = 123
>>> id(a)
4297541856
>>> id(123)
4297541856
\end{pythonConsole}
\end{frame}
%...............................................................................
%...............................................................................
\begin{frame}[fragile]
\frametitle{Order of definition in a module}
\begin{itemize}
 \item L'ordre de définition des fonctions n'est pas important à partir du moment où lors de leurs utilisations, celles-ci ont été définies. 
 C'est particulièrement fréquents dans les modules. 
\end{itemize}
\begin{pythonConsole}
>>> def f(a): 
...     g(a)
>>> def g(b): 
...     print(b)
>>> a='abc'
>>> f(a)
abc
\end{pythonConsole}
\end{frame}
%...............................................................................

%...............................................................................
\begin{frame}[fragile]
\frametitle{Order of definition in a module}
\begin{itemize}
 \item l'ordre de définition des fonctions n'est pas important à partir du moment où lors de l'utilisation es focntions ont été définies. 
 Fonctionne dans la console ou dans un module. 
\end{itemize}
\begin{pythonConsole}
>>> def f(a): 
>>>     g(a)
>>> def g(b): 
>>>     print(b)
>>> a='abc'
>>> f(a)
abc
\end{pythonConsole}
\end{frame}
%...............................................................................

%...............................................................................
\begin{frame}[fragile]
\frametitle{slice and ellipsis}
\begin{itemize}
 \item l'ellipse  
\end{itemize}
\begin{pythonConsole}
>>> A = arange(24).reshape(2, 3, 4)
>>> A
array([[[ 0,  1,  2,  3],
        [ 4,  5,  6,  7],
        [ 8,  9, 10, 11]],

       [[12, 13, 14, 15],
        [16, 17, 18, 19],
        [20, 21, 22, 23]]])
>>> A[:,0]
array([[ 0,  1,  2,  3],
       [12, 13, 14, 15]])
>>> A[...,0]
array([[ 0,  4,  8],
       [12, 16, 20]])
\end{pythonConsole}
\end{frame}
%...............................................................................
%...............................................................................
\begin{frame}[fragile]
\frametitle{slice and ellipsis}
\begin{itemize}
 \item A[i, ...] est identique à A[i, :, :, etc.].
 \item A[i, :] est identique à A[i, :, :, etc. ], donc identique à ... 
 \item A[:, i, :] = A[..., i, :] dans l'exemple mais 
 \item A[..., 0] est différent de A[:, 0] car ici c'est A[:, 0, :] et non A[:, :, 0] ! 
\end{itemize}
\begin{pythonConsole}
>>> A = arange(12).reshape(2, 3, 2)
>>>  A
array([[[ 0,  1],
        [ 2,  3],
        [ 4,  5]],

       [[ 6,  7],
        [ 8,  9],
        [10, 11]]])
>>> A[:,0]                       >>> A[:,0,:] 
array([[0, 1],                   array([[0, 1],
       [6, 7]])                         [6, 7]]) 
>>> A[...,0]
array([[ 0,  2,  4],
       [ 6,  8, 10]])
\end{pythonConsole}
\end{frame}
%...............................................................................
%...............................................................................
\begin{frame}[fragile]
\frametitle{Zen of Python - Pythonic}
\begin{scriptsize}
Thanks to Branislav Gerazov (Branko) in regard to his friendly mail about this presentation: \emph{\dots one thing that I find awesome in Python it's its emphasis on aesthetics and readability. On this note I would add the Zen of Python ('import this') and the concept of being 'pythonic', and contrast it to Perl which is the most awful computer language in the world (I am recoding something now from Perl and that --- is really frustrating!). In this sense I've had students often saying they love Python \dots} 
\end{scriptsize}

\begin{pythonConsole}
>>> import this
The Zen of Python, by Tim Peters

Beautiful is better than ugly.
Explicit is better than implicit.
Simple is better than complex.
Complex is better than complicated.
Flat is better than nested.
...
>>> # a pythonic list comprehension for example !
>>> list_of_number = [(i, 2*i, 3*i) for i in range(10)]  
>>> list_of_number 
[(0, 0, 0), (1, 2, 3), (2, 4, 6), (3, 6, 9), (4, 8, 12), (5, 10, 15), 
(6, 12, 18), (7, 14, 21), (8, 16, 24), (9, 18, 27)]
\end{pythonConsole}
\end{frame}
%...............................................................................
%...............................................................................
\begin{frame}
\frametitle{Lien Patricia Ladret}
\begin{itemize}
 \item Cours d'une collègue Patricia Ladret pour Python via Anaconda: 
 \url{http://chamilo1.grenet.fr/ujf/courses/FAMILIARISATIONAVECPYTHONSUITEANACON/index.php}
\end{itemize}
\end{frame}
%...............................................................................
