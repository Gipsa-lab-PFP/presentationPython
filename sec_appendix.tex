\section{classe et variable 'self'}
%...............................................................................
\begin{frame}[fragile]
\frametitle{Classe exemples avec self}
\begin{itemize}
 \item Dans le corps de la classe, 'self' n'est pas défini. 
\end{itemize}
\begin{pythonConsole}
class Canard(): 
...     self.a = 10
... 
Traceback (most recent call last):
 File £"£<stdin>£"£, line 1, £in£ <module>
 File £"£<stdin>£"£, line 2, £in£ Canard
NameError: name £'self'£ £is not£ defined
\end{pythonConsole}
\end{frame}
%...............................................................................
%...............................................................................
\begin{frame}[fragile]
\frametitle{Classe exemples avec self}
\begin{itemize}
 \item Dans une méthode seul self.nomAttribut est accessible. 
\end{itemize}
\begin{pythonConsole}
class Canard(): 
...     a = 10
...     def __init__(self): 
...             self.b = 100
...             print(self.a)
...             print(b)
... 
riri = Canard()
10
Traceback (most recent call last):
 File £"£<stdin>£"£, line 1, £in£ <module>
 File £"£<stdin>£"£, line 6, £in£ __init__
NameError: £global£ name £'b'£ £is not£ defined
\end{pythonConsole}
\end{frame}
%...............................................................................
%...............................................................................
\begin{frame}[fragile]
\frametitle{Classe exemples avec self}
\begin{itemize}
 \item 'self' est conventionnel. 
 \item Le premier paramètre d'une méthode est considéré comme l'objet lui même. 
\end{itemize}
\begin{pythonConsole}
class Canard(): 
...     def __init__(obj): 
...             obj.a = 10
...             print(obj.a)
... 
riri = Canard()
10
\end{pythonConsole}
\end{frame}
%...............................................................................
%...............................................................................
\begin{frame}[fragile]
\frametitle{Classe exemples avec self}
\begin{itemize}
 \item Possibilité de rajouter des arguments optionnels lors de l'instanciation d'un objet. 
\end{itemize}
\begin{pythonConsole}
class Canard(): 
...     def __init__(self, patte=2):
...             self.a = patte
...             print(self.a)
... 
riri = Canard(patte=3)
3
\end{pythonConsole}
\end{frame}
%...............................................................................
%...............................................................................
\begin{frame}[fragile]
\frametitle{Transposition NDaray exemple 3D}
\begin{itemize}
 \item A.shape (1, 2, 3, 4)
 \item A.T.shape (4, 3, 2, 1) by default. 
 \item can set the axes order. see help(numpy.matrix.transpose)
\end{itemize}
\begin{pythonConsole}
>>> A = numpy.array([[[[1, 2, 3, 4], [11, 12, 13, 14], [21, 22, 23, 24]], [[101, 102, 103, 104], [111, 112, 113, 114], [121, 122, 123, 124]]]])

>>> A
array([[[[  1,   2,   3,   4],
         [ 11,  12,  13,  14],
         [ 21,  22,  23,  24]],

        [[101, 102, 103, 104],
         [111, 112, 113, 114],
         [121, 122, 123, 124]]]])

>>> A.T
array([[[[  1],
         [101]],
         ...
        [[ 24],
         [124]]]])
\end{pythonConsole}
\end{frame}
%...............................................................................
