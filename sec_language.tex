%===============================================================================
\section{Description du Langage}
%_______________________________________________________________________________
\subsection{Type de données}
%...............................................................................
\begin{frame}
\frametitle{Type de base}
\framesubtitle{}
\begin{itemize}
\item {\bfseries Nombres} : entiers (int, long), réels (float), complex (complex), booléens (bool). 
\item {\bfseries Séquences} : Chaînes de caractères (str), listes (list), tuples (tuple), dictionnaires (dict), ensembles (set)
\end{itemize}
\end{frame}
%...............................................................................
%...............................................................................
\begin{frame}[fragile]
\frametitle{Nombres}
\framesubtitle{Entiers}
\begin{itemize}
 \item Attention à la division entière en 2.7. 
\end{itemize}
\begin{pythonConsole}
>>> a = 1  
>>> b = 2
>>> c = a / b
>>> print(c)
0
\end{pythonConsole}
\end{frame}
%...............................................................................
%...............................................................................
\begin{frame}[fragile]
\frametitle{Nombres}
\framesubtitle{Réels}
\begin{itemize}
 \item Utilisation du point pour passer en réels. 
\end{itemize}
\begin{pythonConsole}
>>> a = 1.0
>>> b = 2.
>>> c = a / b
>>> print(c)
0.5
\end{pythonConsole}
\end{frame}
%...............................................................................
%...............................................................................
\begin{frame}[fragile]
\frametitle{Nombres}
\framesubtitle{Entiers long vs court}
\begin{itemize}
 \item Suffixe 'L' pour indiquer un passage en entier long. 
 \item 'type()' indique le type d'une variable. 
\end{itemize}
\begin{pythonConsole}
>>> a = 2 ** 30
>>> b = 2 ** 31
>>> a
1073741824
>>> b
2147483648L
>>> print(b)
2147483648
>>> type(a)
<type 'int'>
>>> type(b)
<type 'long'>
>>> c = 12345L
>>> type(c)
<type 'long'>
\end{pythonConsole}
% \only<2>{\it remarque : représentation simplifiée (str(a)) vs representation brutes (repr(a)) dans la console.}  
\end{frame}
%...............................................................................
%...............................................................................
\begin{frame}[fragile]
\frametitle{Nombres}
\framesubtitle{Réels notations scientifiques}
\begin{itemize}
 \item Notation scientifique de type mantisse et exposant. 
\end{itemize}
\begin{pythonConsole}
>>> a = 1.234e100
>>> a
1.234e+100
>>> type(a)
<type 'float'>
\end{pythonConsole}
\end{frame}
%...............................................................................
%...............................................................................
\begin{frame}[fragile]
\frametitle{Nombres}
\framesubtitle{Complexes}
\begin{itemize}
 \item z = x + yj (complex)
 \item Attention à la syntaxe. 
 \item Génère une erreur (\emph{exception}) de type "NameError". 
 \item '\_' variable courante (cf. \emph{ans} matlab) 
\end{itemize}
\begin{pythonConsole}
>>> a = 1 + 2j
>>> type(a)
<type 'complex'>
>>> a = 1 + j
Traceback (most recent call last):
  File " <stdin>", line 1, in <module>
NameError: name 'j' is not defined
>>> 123.456j
123.456j
>>> type(_)
<type 'complex'>
>>> a = 1 + 1j
>>> b = 2 + 3j
>>> c = a + b
>>> print(c)
(3+4j)
\end{pythonConsole}
\end{frame}
%...............................................................................
%...............................................................................
\begin{frame}[fragile]
\frametitle{Nombres}
\framesubtitle{Booléens}
\begin{itemize}
 \item True, False (bool)
 \item Attention aux résultats des opérations sur les booléens.
 \item + donne un int
 \item Opérateurs booléens spécifiques.   
\end{itemize}
\begin{pythonConsole}
>>> a = True
>>> type(a)
<type 'bool'>
>>> b = False
>>> c = a + b
>>> print(c)
1
>>> type(c)
<type 'int'>
>>> c = a * b
>>> print(c)
0
>>> c = a & b  
>>> print(c)
False
>>> c = a | b
>>> print(c)
True
\end{pythonConsole}
\end{frame}
%...............................................................................
%...............................................................................
\begin{frame}
\frametitle{Nombres}
\framesubtitle{Opérateurs}
\begin{itemize}
 \item Redéfinition des opérateurs selon les types. 
 \item Pour les règles de priorités, cf 'language reference'.
 \item Bitwise operations on integer, cf 'language reference'. 
\end{itemize}

\begin{tabular}{l|ll|ll}
{\bfseries entiers} & + & addition & - & substraction \\
 &  * & multiplication & / & division \\
 & \% & modulo & // & floor division \\
 & ** & power  & & \\
 & << & shifting left & >> & shifting right \\
{\bfseries réels} & \multicolumn{4}{l}{idem entiers hors shifting et bitwise op.}\\
{\bfseries complexes} & \multicolumn{4}{l}{idem réels}\\
{\bfseries booléens} & \&, and & and & |, or & or \\
 & \^\/ & xor & not & not 
\end{tabular}
\end{frame}
%...............................................................................
%...............................................................................
\begin{frame}[fragile]
\frametitle{Nombres}
\framesubtitle{Opérateurs de comparaisons}
\begin{tabular}{ll}
{\bfseries Comparaisons} & > , < , >=, <=, == \\
 & <>, != (mieux, version C) \\
 & in, not in (sequence) \\
 & is, is not (objet) 
\end{tabular}
\begin{pythonConsole}
>>> a = 1 
>>> a in [0, 1, 2]
True
>>> b = 2
>>> a is b
False
>>> a is 1
True
>>> c = a
>>> a is c
True
>>> 0 < a <= 2
True
\end{pythonConsole}
\end{frame}
%...............................................................................
%...............................................................................
\begin{frame}[fragile]
\frametitle{Nombres}
\framesubtitle{Typage explicite, conversion de type}
\begin{itemize}
 \item possibilité de typer explicitement avec la fonction associée au type voulue : int(), float() \dots
 \item conversion, implicite pour certaines opérations. 
\end{itemize}
\begin{pythonConsole}
>>> a = float(9)
>>> type(a)
<type 'float'>
>>> c = 1 / a
>>> print(c)
0.111111111111
>>> b = 9
>>> c = 1 / float(b)
>>> print(c)
0.111111111111
>>> d = complex(1, 2)
>>> print(d)
(1+2j)
>>> print(str(d))
(1+2j)
\end{pythonConsole}
\end{frame}
%...............................................................................
%...............................................................................
\begin{frame}[fragile]
\frametitle{Nombres et Caractères}
\framesubtitle{Binary, Hexadecimal, Octal, Character, Unicode Character}
\begin{itemize}
 \item binary (0b) (bin), hexadecimal (0x) (hex), octal (0c) (oct) 
 \item character (chr) (< 0xFF) ou unicode (unichr) (< 0xFFFF)
\end{itemize}
\begin{pythonConsole}
>>> a = 0x597d
>>> print(a)
22909
>>> a = 0b0101
>>> print(a)
5
>>> a = chr(42)
>>> print(a)
*
>>> b = unichr(0x597D)
>>> print(b)
chinese symbol ni
>>> a = bin(65447)
>>> print(a)
0b101100101111101
>>> type(a)
<type 'str'>
\end{pythonConsole}
\end{frame}
%...............................................................................
%...............................................................................
\begin{frame}[fragile]
\frametitle{Séquences}
\framesubtitle{Chaînes de caractères}
\begin{itemize}
 \item Texte délimité par ' ' ou " " (str) 
 \item La chaîne de caractère a une longueur (len)
\end{itemize}
\begin{pythonConsole}
>>> a = 'abc'
>>> type(a)
<type 'str'>
>>> b = "def"
>>> c = a + b
>>> c
'abcdef'
>>> len(c)
6
\end{pythonConsole}
\end{frame}
%...............................................................................
%...............................................................................
\begin{frame}[fragile]
\frametitle{Séquences}
\framesubtitle{Chaînes de caractères}
\begin{itemize}
 \item Alternance des single/double quote. 
 \item Caractères spéciaux à la C.
\end{itemize}
\begin{pythonConsole}
>>> a = "1. Bisque de pigeonneaux\n\t Prenez vos pigeonneaux apres qu'ils
seront bien nettoyez\n\t&'Troussez', faites les blanchir, \n\t& Les " + 
'"empottez" avec un petit brin de fines herbes'
>>> print(a)
1. Bisque de pigeonneaux
    Prenez vos pigeonneaux apres qu£'£ils seront bien nettoyez
    & £'£Troussez£'£, faites les blanchir, 
    & Les £"£empottez£"£ avec un petit brin de fines herbes
\end{pythonConsole}
\end{frame}
%...............................................................................
%...............................................................................
\begin{frame}[fragile]
\frametitle{Séquences}
\framesubtitle{Chaînes de caractères}
\begin{itemize}
 \item forme mutliligne """ """ ou '\,'\,' '\,'\,' (! caractères spéciaux).  
 \item Représentation (repr) différente de (str). 
\end{itemize}
\begin{pythonConsole}
>>> a = """1. Bisque de pigeonneaux\n
... \tPrenez vos pigeonneaux apres qu'ils seront bien nettoyez\n
...     & 'Troussez', faites les blanchir, 
...     & Les "empottez" avec un petit brun de fines herbes
... """
>>> print(a)
1. Bisque de pigeonneaux

    Prenez vos pigeonneaux apres qu£'£ils seront bien nettoyez 

    & £'£Troussez£'£, faites les blanchir,
    & Les £"£empottez£"£ avec un petit brun de fines herbes
>>> repr(a)
£'£\£'£1. Bisque de pigeonneaux\\n\\n\\t\\tPrenez vos pigeonneaux apres qu\\\£'£
ils seront bien nettoyez\\n\\n\\t& \\\£'£Troussez\\\£'£, faites les blanchir,\
\n\\t& Les £"£empottez£"£ avec un petit brun de fines herbes\\n\£'££'£
>>> str(a)
"1. Bisque de pigeonneaux\n\n\tPrenez vos pigeonneaux apres qu'ils seront bien 
nettoyez\n\n    & 'Troussez', faites les blanchir, \n    & Les £"£empottez£"£ avec
un petit brun de fines herbes"
\end{pythonConsole}
\end{frame}
%...............................................................................
%...............................................................................
\begin{frame}[fragile]
\frametitle{Séquences}
\framesubtitle{Chaînes de caractères}
\begin{itemize}
 \item forme brute (r" " ou r' ' ou r""" """ ou r'\,'\,' '\,'\,')
 \item forme unicode (u" " ou u' ' ou u""" """ ou u'\,'\,' '\,'\,')
\end{itemize}
\begin{pythonConsole}
>>> a = r"Potage de sant£\color{magenta}{é}£\n\tLe potage de sant£\color{magenta}{é}£ se fait de chapons..."
>>> print(a)
Potage de sant£é£\n\tLe potage de sant£é£ se fait de chapons...
>>> repr(a)
"'Potage de sant\\xc3\\xa9\\\\n\\\\tLe potage de sant\\xc3\\xa9 se fait 
de chapons...'"
>>> a = u"Potage de sant£\color{magenta}{é}£\n\tLe potage de sant£\color{magenta}{é}£ se fait de chapons..."
>>> print(a)
Potage de sant£é£
    Le potage de sant£é£ se fait de chapons...
>>> repr(a)
"u'Potage de sant\\xe9\\n\\tLe potage de sant\\xe9 se fait de chapons...
'"
>>> print(u"\u4F60\u597D")
\end{pythonConsole}
% 你好
\end{frame}
%...............................................................................

%...............................................................................
\begin{frame}[fragile]
\frametitle{Séquences}
\framesubtitle{Listes}
\begin{itemize}
 \item Listes d'éléments ([\ , \ ], list)
 \item Longueur $n$ (len)
 \item Accès aux élément : indice de \color{red}{0 à $n-1$}
\end{itemize}
\begin{pythonConsole}
>>> a = [1, 2, 3]
>>> type(a)
<type £'£list£'£>
>>> len(a)
3
>>> b = ['abc', 'de', 'fghij']
>>> len(b)
3
>>> c = a + b
>>> print(c)
[1, 2, 3, 'abc', 'de', 'fghij']
>>> c[0]
1
>>> c[5]
'fghij'
\end{pythonConsole}
\end{frame}
%...............................................................................
%...............................................................................
\begin{frame}[fragile]
\frametitle{Séquences}
\framesubtitle{Listes}
\begin{itemize}
 \item découpage, slicing (:, i:, :j, i:j, i:j:k, slice(i,j,k))
 \item Indices négatifs de -1 à $-n$ permettent de parcourir en partant par la fin. 
\end{itemize}
\begin{pythonConsole}
>>> a = [1, 2, 3, 4, 5, 6, 7, 8, 9]
>>> a[:]
[1, 2, 3, 4, 5, 6, 7, 8, 9]
>>> a[3:]
[4, 5, 6, 7, 8, 9]
>>> a[:3]
[1, 2, 3]
>>> a[2:2+4] # a[2:6]
[3, 4, 5, 6]
>>> a[2:6:2]
[3, 5]
>>> s = slice(2, 6, 2)
>>> a[s]
[3, 5]
>>> a[-1]
9 
>>> a[-9]
1 
\end{pythonConsole}
\end{frame}
%...............................................................................
%...............................................................................
\begin{frame}[fragile]
\frametitle{Séquences}
\framesubtitle{Listes}
\begin{itemize}
 \item Imbrications, nested.
 \item Attention aux erreurs d'indexation, les listes Python ne sont pas de matrices à la Matlab. 
\end{itemize}
\begin{pythonConsole}
>>> a = [[1, 2, 3], [4, 5, 6]]
>>> a
[[1, 2, 3], [4, 5, 6]]
>>> len(a)
2
>>> a[0]
[1, 2, 3]
>>> a[0][0]
1
>>> a[0, 0]
Traceback (most recent call last):
  File £"£<stdin>£"£, line 1, £in£ <module>
TypeError: list indices must be integers, £not£ tuple
\end{pythonConsole}
\end{frame}
%...............................................................................
%...............................................................................
\begin{frame}[fragile]
\frametitle{Séquences}
\framesubtitle{Tuples}
\begin{itemize}
 \item Couple, triplet, n-uplet, plus généralement tuple ((\ , \ ), tuple)
 \item Ce sont des séquences (de même que les types str, unicode, list, tuple, buffer, xrange) : même indexation que les listes.
\end{itemize}
\begin{pythonConsole}
>>> couple = ('papa', 'maman')
>>> type(couple)
<type 'tuple'>
>>> couple[0]
'papa'
>>> a = ('bob', 'alice')
>>> b = ('pierre', 'paul')
>>> c = a + b
>>> print(c)
('bob', 'alice', 'pierre', 'paul')
\end{pythonConsole}
\end{frame}
%...............................................................................
%...............................................................................
\begin{frame}[fragile]
\frametitle{Séquences}
\framesubtitle{Tuples}
\begin{itemize}
 \item Taille fixe immuable (immutable en anglais). 
 \item Pratique pour le passage de paramètres de taille fixe. 
 \item Affectation multiple implicite. 
 \item On ne peut pas changer les valeurs des tuples. 
\end{itemize}
\begin{pythonConsole}
>>> papa = 'bob'
>>> maman = 'alice'
>>> couple = (papa, maman)
>>> print(couple)
'bob', 'alice'
>>> couple = (papa, maman) = ('bob', 'alice')
>>> couple = papa, maman = 'bob', 'alice'
>>> couple[0] = 'robert'
Traceback (most recent call last):
  File £"£<stdin>£"£, line 1, £in£ <module>
TypeError: 'tuple' object does £not£ support item assignment
\end{pythonConsole}
\end{frame}
%...............................................................................
%...............................................................................
\begin{frame}[fragile]
\frametitle{Séquences}
\framesubtitle{Tuples}
\begin{itemize}
 \item Attention : couples immuables, listes muables 
\end{itemize}
\begin{pythonConsole}
>>> address = '0001'
>>> instr = '011'
>>> code1 = (instr, address)
>>> code2 = ('100', '0010')
>>> code3 = ('110', '0011')
>>> code4 = ('010', '0001')
>>> prog = [code1, code2, code3]
>>> prog.append(code4)
>>> print(progr)
[('011', '0001'), ('100', '0010'), ('110', '0011'), ('010', '0001')]
\end{pythonConsole}
\end{frame}
%...............................................................................
%...............................................................................
\begin{frame}[fragile]
\frametitle{Séquences}
\framesubtitle{Tuples}
\begin{itemize}
 \item Exemple de liste de couples. 
\end{itemize}
\begin{pythonConsole}
>>> name = 'abc'
>>> value = 123
>>> entry1 = (name, value)
>>> entry2 = ('def', 234)
>>> dictionary = [entry1, entry2]
>>> dictionary.append(('efg', 345))
>>> print(dictionary)
[('abc', 123), ('def', 234), ('efg', 345)]
\end{pythonConsole}
\end{frame}
%...............................................................................
%...............................................................................
\begin{frame}[fragile]
\frametitle{Séquences}
\framesubtitle{Dictionnaires}
\begin{itemize}
 \item Dictionnaires (dict) 
 \item key (str): value 
\end{itemize}
\begin{pythonConsole}
>>> d = {'abc': 123, 'def': 234}
>>> type(d)
<type 'dict'>
>>> print(d)
{'abc': 123, 'def': 234}
>>> d['abc']
123
>>> for key in d: 
...  print((key, d[key]))
... 
('abc', 123)
('def', 234)
\end{pythonConsole}
\end{frame}
%...............................................................................
%...............................................................................
\begin{frame}[fragile]
\frametitle{Séquences}
\framesubtitle{Ensemble}
\begin{itemize}
 \item Ensemble (\{\ ,\ \}, set) 
 \item Collections d'éléments uniques non ordonnés.
 \item Opération ensemblistes (add, discard, union, intersect, \dots)
\end{itemize}
\begin{pythonConsole}
>>> E = {1, 2, 3}
>>> type(E)
<type 'set'>
>>> print(E)
set([1, 2, 3])
>>> E[0]
Traceback (most recent call last):
  File £"£<stdin>£"£, line 1, £in£ <module>
TypeError: £'set'£ object does £not£ support indexing
>>> F = {1, 2, 3, 2, 1, 4, 1, -1}
>>> F
set([1, 2, 3, 4, -1])
>>> a = 1
>>> a in F
True
>>> G = E + F
Traceback (most recent call last):
  File £"£<stdin>£"£, line 1, £in£ <module>
TypeError: unsupported operand type(s) £for£ +: £'set'£ £and£ £'set'£
>>> E.intersection(F)
set([1, 2, 3]) 
>>> E.add(-10)
>>> print(E)
set([1, 2, 3, -10])
\end{pythonConsole}
\end{frame}
%...............................................................................
%_______________________________________________________________________________
%_______________________________________________________________________________
\subsection{Syntaxe}
%...............................................................................
\begin{frame}
\frametitle{Syntaxe}
\framesubtitle{}
\begin{itemize}
 \item Fonctions prédéfinies, 'Builtin' Functions  
 \item Structures de contrôles 
 \item Fonctions
 \item Objets et Classes
\end{itemize}
\end{frame}
%...............................................................................
%...............................................................................
\begin{frame}
\frametitle{Syntaxe}
\framesubtitle{Fonctions prédéfinies}
\begin{itemize}
\item Built-in Functions : fonctions de base du langage. 
\end{itemize}
{\tiny%
\begin{tabular}{lllll}
abs() & divmod() & input() &  open() & staticmethod() \\
all() & enumerate() &  int() & ord() & str() \\
any() & eval() & isinstance() & pow() & sum() \\
basestring() &  execfile() &  issubclass() &  print() &  super() \\
bin() &  file() & iter() & property() &  tuple() \\
bool() & filter() & len() &  range() &  type() \\
bytearray() &  float() &  list() &  raw\_input() &  unichr() \\
callable() & format() &  locals() &  reduce() &  unicode() \\
chr() &  frozenset() & long() & reload() & vars() \\
classmethod() & getattr() & map() & repr() &  xrange() \\
cmp() & globals() & max() & reversed() & zip() \\
compile() &  hasattr() & memoryview() & round() &  \_\_import\_\_()\\
complex() & hash() & min() & set() &  apply() \\
delattr() &  help() & next() & setattr() &  buffer() \\
dict() & hex() & object() & slice() & coerce() \\
dir() & id() & oct() &  sorted() &  intern() 
\end{tabular}
}
\end{frame}
%...............................................................................
%...............................................................................
\begin{frame}
\frametitle{Syntaxe}
\framesubtitle{Fonctions prédéfinies}
\begin{itemize}
 \item Built-in Functions : fonctions de base du langage. 
\end{itemize}
{\tiny%
\begin{tabular}{lllll}
abs() & divmod() & input() &  open() & staticmethod() \\
all() & enumerate() &  \color{red}{int()} & ord() & \color{red}{str()} \\
any() & eval() & isinstance() & pow() & sum() \\
basestring() &  execfile() &  issubclass() &  \color{red}{print()} &  super() \\
\color{red}{bin()} &  file() & iter() & property() &  \color{red}{tuple()} \\
\color{red}{bool()} & filter() & \color{red}{len()} &  range() &  \color{red}{type()} \\
bytearray() &  \color{red}{float()} &  \color{red}{list()} &  raw\_input() &  \color{red}{unichr()} \\
callable() & format() &  locals() &  reduce() &  \color{red}{unicode()} \\
chr() &  frozenset() & \color{red}{long()} & reload() & vars() \\
classmethod() & getattr() & map() & \color{red}{repr()} &  xrange() \\
cmp() & globals() & max() & reversed() & zip() \\
compile() &  hasattr() & memoryview() & round() &  \_\_import\_\_()\\
\color{red}{complex()} & hash() & min() & \color{red}{set()} &  apply() \\
delattr() &  \color{red}{help()} & next() & setattr() &  buffer() \\
\color{red}{dict()} & \color{red}{hex()} & object() & \color{red}{slice()} & coerce() \\
\color{red}{dir()} & id() & \color{red}{oct()} &  sorted() &  intern() 
\end{tabular}
}

\color{red}{déjà rencontrées}
\end{frame}
%...............................................................................
%...............................................................................
\begin{frame}
\frametitle{Syntaxe}
\framesubtitle{Fonctions prédéfinies}
\begin{itemize}
 \item Built-in Functions : fonctions de base du langage. 
\end{itemize}
{\tiny%
\begin{tabular}{lllll}
\color{blue}{abs()} & \color{blue}{divmod()} & \color{blue}{input()} &  \color{blue}{open()} & staticmethod() \\
\color{blue}{all()} & \color{blue}{enumerate()} &  int() & \color{blue}{ord()} & str() \\
\color{blue}{any()} & eval() & isinstance() & \color{blue}{pow()} & \color{blue}{sum()} \\
basestring() &  execfile() &  issubclass() &  print() &  super() \\
bin() &  \color{blue}{file()} & \color{blue}{iter()} & property() &  tuple() \\
bool() & filter() & len() &  \color{blue}{range()} &  type() \\
bytearray() &  float() &  list() &  raw\_input() &  unichr() \\
callable() & format() &  locals() &  reduce() &  unicode() \\
\color{blue}{chr()} &  frozenset() & long() & reload() & \color{blue}{vars()} \\
classmethod() & getattr() & map() & repr() &  \color{blue}{xrange()} \\
cmp() & globals() & \color{blue}{max()} & reversed() & \color{blue}{zip()} \\
compile() &  hasattr() & memoryview() & \color{blue}{round()} &  \color{blue}{\_\_import\_\_()}\\
complex() & hash() & \color{blue}{min()} & set() &  apply() \\
delattr() &  help() & \color{blue}{next()} & setattr() &  buffer() \\
dict() & hex() & object() & slice() & coerce() \\
dir() & \color{blue}{id()} & oct() &  sorted() &  intern() 
\end{tabular}
}

\color{blue}{abordées tôt ou tard}
\end{frame}
%...............................................................................
%...............................................................................
\begin{frame}[fragile]
\frametitle{Syntaxe}
\framesubtitle{Structure de contrôle}
\begin{itemize}
\item {\bfseries blocs conditionnels} : if else
\item {\bfseries boucles conditionnelles} : while
\item {\bfseries boucles} : for in
\item {\bfseries gestion d'exceptions} : try except 
\end{itemize}
\end{frame}
%...............................................................................
%...............................................................................
\begin{frame}[fragile]
\frametitle{Syntaxe}
\framesubtitle{Structure de contrôle - if else}
\begin{itemize}
\item blocs conditionnels : if else
\item ':' et indentation suivie d'un bloc d'instruction.  
\end{itemize}
\begin{pythonConsole}
>>> a = 1
>>> if (a == 1): 
...     print("a vaut 1")
... else: 
...     print("a ne vaut pas 1")
...
a vaut 1
\end{pythonConsole}
\end{frame}
%...............................................................................
%...............................................................................
\begin{frame}[fragile]
\frametitle{Syntaxe}
\framesubtitle{Indentation}
\begin{itemize}
\item Pour l'indentation dans un fichier, il est conseillé d'utiliser 4 espaces plutôt qu'une tabulation. 
\item Possibilité de régler ce paramètres dans les éditeurs de texte : tabulation souple émulée avec des espaces ('soft tab' with 4 spaces vs 'hard tab'). 
\end{itemize}
\begin{python}
if (a == 1): 
    print("a vaut 1")
else: 
    print("a ne vaut pas 1")
\end{python}
\end{frame}
%...............................................................................
%...............................................................................
\begin{frame}[fragile]
\frametitle{Syntaxe}
\framesubtitle{Structure de contrôle - if elif else}
\begin{itemize}
\item blocs conditionnels : if elif else
\end{itemize}
\begin{pythonConsole}
>>> a = 2
>>> if (a == 1): 
...     print("a vaut 1")
... elif a == 2: 
...     print("a vaut 2")
... elif a == 3: 
...     print("a vaut 3")
... else: 
...     print("a ne vaut ni 1 ni 2 ni 3")
...
a vaut 2
\end{pythonConsole}
cf matlab (if elseif else end, switch case otherwise end) 
\end{frame}
%...............................................................................
%...............................................................................
\begin{frame}[fragile]
\frametitle{Syntaxe}
\framesubtitle{Structure de contrôle - if elif else}
\begin{itemize}
\item blocs conditionnels : if elif else
\end{itemize}
\begin{pythonConsole}
>>> a = "green"
>>> if (a is "green"): 
...     print("00FF00")
... elif (a == "red"): 
...     print("FF0000")
... elif (a is "blue"): 
...     print("0000FF")
... else: 
...     print("Unknown color")
...
00FF00
\end{pythonConsole}
cf matlab (if elseif else end, switch case otherwise end) 
\end{frame}
%...............................................................................
%...............................................................................
\begin{frame}[fragile]
\frametitle{Syntaxe}
\framesubtitle{Structure de contrôle - while}
\begin{itemize}
\item boucles conditionnelles : while
\end{itemize}
\begin{pythonConsole}
>>> question = "Voulez-vous continuer ? (o, oui, O, OUI) "
>>> cond = True
>>> reponseOK = {"o", "O", "oui", "OUI"}
>>> i = 0
>>> while cond:
...    i += 1
...    print(str(i) + " fois")
...    answer = input(question)
...    if (answer in reponseOK): 
...        cond = True
...    else: 
...        cond = False
...
1 fois 
Voulez-vous continuer ? (o, oui, O, OUI) 'o'
2 fois 
Voulez-vous continuer ? (o, oui, O, OUI) 'oui'
3 fois
Voulez-vous continuer ? (o, oui, O, OUI) 'non'
>>> 
\end{pythonConsole}
\end{frame}
%...............................................................................
%...............................................................................
\begin{frame}[fragile]
\frametitle{Syntaxe}
\framesubtitle{Structure de contrôle - for}
\begin{itemize}
\item boucles : for \ in \ :
\item Dans 'language reference' : \lstinline{for_stmt ::= "for" target_list "in" expression_list ":" suite}
[”else” ”:” suite]
\end{itemize}
\begin{pythonConsole}
>>> for i in [0, 1, 2, 3]: 
...     print(i)
... 
0
1
2
3
\end{pythonConsole}
\end{frame}
%...............................................................................
%...............................................................................
\begin{frame}[fragile]
\frametitle{Syntaxe}
\framesubtitle{Structure de contrôle - for}
\begin{itemize}
\item {[0, 1, 2, 3]} : range(i,j,k) 
\end{itemize}
\begin{pythonConsole}
>>> for i in range(4): 
...     print(i)
... 
0
1
2
3
\end{pythonConsole}
\begin{pythonConsole}
>>> x = [0, 1, 2, 3, 4, 5, 6, 7, 8, 9]
>>> x[range(1, 5, 2)]
Traceback (most recent call last):
  File £"£<stdin>£"£, line 1, in <module>
TypeError: list indices must be integers, not list
>>> x[slice(1, 5, 2)]
[1, 3]
>>> range(1, 5, 2)
[1, 3]
>>> slice(1, 5, 2)
slice(1, 5, 2)
\end{pythonConsole}
\end{frame}
%...............................................................................
%...............................................................................
\begin{frame}[fragile]
\frametitle{Syntaxe}
\framesubtitle{Structure de contrôle - for}
\begin{itemize}
\item boucles : for 
\end{itemize}
\begin{pythonConsole}
>>> for (i, j) in [(1, 2), (2, 3), (3, 4)]: 
...   print((i,j))
... 
(1, 2)
(2, 3)
(3, 4)
\end{pythonConsole}
\begin{pythonConsole}
>>> for [i, j] in [[1, 2], [2, 3], [3, 4]]: 
...   print([i, j])
... 
[1, 2]
[2, 3]
[3, 4]
\end{pythonConsole}
\end{frame}
%...............................................................................
%...............................................................................
\begin{frame}[fragile]
\frametitle{Syntaxe}
\framesubtitle{Structure de contrôle - break, continue}
\begin{itemize}
\item possibilité de break et continue. 
\end{itemize}
\begin{pythonConsole}
>>> for i in range(10):
...   if (i £\%£ 2 == 0): 
...     continue
...   if (i == 9): 
...     break
...   print(i)
...
1
3
5
7
\end{pythonConsole}
\end{frame}
%...............................................................................
%...............................................................................
\begin{frame}[fragile]
\frametitle{Syntaxe}
\framesubtitle{Structure de contrôle - try except}
\begin{itemize}
\item Gestion des erreurs ou exceptions : try except 
\end{itemize}
\begin{pythonConsole}
>>> a = "abc"
>>> a += 1
Traceback (most recent call last):
  File £"£<stdin>£"£, line 1, £in£ <module>
TypeError: cannot concatenate £'£str£'£ £and£ £'£int£'£ objects
\end{pythonConsole}
\begin{pythonConsole}
>>> a = 'abc'
>>> try: 
...     a += 1
... except : 
...     print("problem" )
... 
problem
\end{pythonConsole}
\end{frame}
%...............................................................................
%...............................................................................
\begin{frame}[fragile]
\frametitle{Syntaxe}
\framesubtitle{Structure de contrôle - try except else finally}
\begin{itemize}
\item Gestion des erreurs ou exceptions : try except else finally 
\end{itemize}
\begin{pythonConsole}
>>> a = 'abc'
>>> try: 
...     a += 1
... except IOError as e :
...     print("problem entr£{\color{magenta}é}£es sorties : " + str(e))
... except TypeError as e :
...     print("problem de types : " + str(e))
... finally:
...     print("Avec ou sans erreurs, on continue")
... 
problem de types : cannot concatenate £'str'£ £and£ £'int'£ objects
Avec ou sans erreurs, on £continue£
\end{pythonConsole}
\begin{pythonConsole}
>>> a = 1
>>> try: 
...     a += 1
... except IOError as e :
...     print("problem entr£{\color{magenta}é}£es sorties : " + str(e))
... except TypeError as e :
...     print("problem de types : " + str(e))
... else: 
...     print("a = " + str(a))
... finally:
...     print("Avec ou sans erreurs, on continue")
...
a = 2
Avec ou sans erreurs, on £continue£
\end{pythonConsole}
\end{frame}
%...............................................................................
%...............................................................................
\begin{frame}[fragile]
\frametitle{Syntaxe}
\framesubtitle{Fonctions - Définition de la fonction}
\begin{itemize}
\item définition d'une fonction : "def" funcname "(" [parameter\_list] ")" ":" suite
\item indentation nécessaire pour le bloc d'instructions. 
\end{itemize}
\begin{pythonConsole}
>>> def sayHello(): 
...    print("Hello")
...
>>> sayHello()
Hello
\end{pythonConsole}
\end{frame}
%...............................................................................
%...............................................................................
\begin{frame}[fragile]
\frametitle{Syntaxe}
\framesubtitle{Fonctions - passage de paramètres obligatoires}
\begin{itemize}
\item Les arguments (dit args) sont définis entre parenthèse. 
\item args : obligatoires et ordonnés
\end{itemize}
\begin{pythonConsole}
>>> def sayHello(titre, prenom, nom): 
...    print("Hello " + str(titre) + " " + str(prenom) + " " + str(nom))
...
>>> sayHello("d£{\color{magenta}é}£tective priv£{\color{magenta}é}£ de classe R", "John", "Difool")
Hello d£é£tective priv£é£ de classe R John Difool
\end{pythonConsole}
\begin{pythonConsole}
>>> sayHello()
Traceback (most recent call last):
  File £"£<stdin>£"£, line 1, £in£ <module>
TypeError: sayHello() takes exactly 3 arguments (0 given)
\end{pythonConsole}
\end{frame}
%...............................................................................
%...............................................................................
\begin{frame}[fragile]
\frametitle{Syntaxe}
\framesubtitle{Fonctions - passage de paramètres optionnels}
\begin{itemize}
\item arguments mots clés (keywords arguments dits kwargs) non obligatoires. 
\item args avant les kwargs
\item kwargs : optionnels et non ordonnés
\end{itemize}
\begin{pythonConsole}
>>> def sayHello(titre, prenom="", nom=""): 
...    print("Hello " + str(titre) + " " + str(prenom) + " " + str(nom))
...
>>> sayHello("old")
Hello old
>>> sayHello("old", prenom="Nick")
Hello old Nick
>>> sayHello("old", "Tom", "Bombadil")
Hello old Tom Bombadil
>>> sayHello("old", nom="Bombadil", prenom="Tom")
Hello old Tom Bombadil
>>> sayHello(nom="Bombadil")
Traceback (most recent call last):
  File £"£<stdin>£"£, line 1, in <module>
TypeError: sayHello() takes at least 1 argument (1 given)
>>> sayHello("old", nom="Bombadil", "Tom")
  File £"£<stdin>£"£, line 1
SyntaxError: non-keyword arg after keyword arg
\end{pythonConsole}
\end{frame}
%...............................................................................
%...............................................................................
\begin{frame}[fragile]
\frametitle{Syntaxe}
\framesubtitle{Fonctions - renvoi de valeurs}
\begin{itemize}
\item renvoi de valeurs : return
\end{itemize}
\begin{pythonConsole}
>>> def fun(x): 
...    y = 2 * x
...    return y
...
>>> fun(3)
6
\end{pythonConsole}
\begin{pythonConsole}
>>> def fun(x): 
...    y1 = x
...    y2 = 2 * x
...    y3 = 3 * x
...    name = "1, 2, 3 * " + str(x) + " = "
...    return name, y1, y2, y3 # the tuple (name, y1, y2, y3)
...
>>> fun(3)
('1, 2, 3 * 3 = ', 3, 6, 9)
>>> (name, y1, y2, y3) = fun(3)
>>> name, y1, y2, y3 = fun(3)
\end{pythonConsole}
\end{frame}
%...............................................................................
%...............................................................................
\begin{frame}[fragile]
\frametitle{Syntaxe}
\framesubtitle{Fonctions - autres spécificités}
\begin{itemize}
\item bloc vide : pass 
\item None values
\item variables de fonctions 
\end{itemize}
\begin{pythonConsole}
>>> def fonctionQuiNeFaitRienEtQuiAUnNomHorribleMaisDescriptif(x): 
...    pass
...
>>> def fonctionQuiNeFaitRienEtQuiAUnNomHorribleMaisDescriptif(x): 
...    return
...
>>> def fonctionQuiNeFaitRienEtQuiAUnNomHorribleMaisDescriptif(x): 
...    return None
...
>>> a = fonctionQuiNeFaitRienEtQuiAUnNomHorribleMaisDescriptif(3)
>>> a == None
True
>>> pfff = fonctionQuiNeFaitRienEtQuiAUnNomHorribleMaisDescriptif
>>> pfff(1)
>>> pfff("abc")
>>> pfff
<function fonctionQuiNeFaitRienEtQuiAUnNomHorribleMaisDescriptif at 0x2d49b0>
\end{pythonConsole}
\end{frame}
%...............................................................................
%...............................................................................
\begin{frame}[fragile]
\frametitle{Syntaxe}
\framesubtitle{Objets - Notion de programmation orientée objet}
\begin{itemize}
\item Un objet possède : 
 \begin{itemize}
  \item des propriétés ou attributs, variables spécifiques à l'objet. 
  \item des méthodes, fonctions qui permettent de communiquer entre objets ou avec l'extérieur. 
 \end{itemize}
\item Un objet appartient à une classe: {\em Un objet est une instance de classe. }
\item Les propriétés et méthodes sont définies dans la classe. 
\end{itemize}

\begin{pythonConsole}
3 canards : riri, fifi, loulou

classe Canard : 
propri£é£t£é£s, attributs : 2 pattes, 2 ailes, £é£tat d£'£activit£é£
m£é£thodes : manger, chanter, ne rien faire.  

Canard : 
 £à£ 2 pattes
 £à£ 2 ailes
 status par d£é£faut : ne fais rien
 
 mange(aliments)
  v£é£rification aliments : herbivore, carnivore occasionnel 
 
 chante()
  g£é£n£è£re un son qui fait "couin couin"  

 ne fait rien()
  
\end{pythonConsole}
\end{frame}
%...............................................................................
%...............................................................................
\begin{frame}[fragile]
\frametitle{Syntaxe}
\framesubtitle{Objets : définition en Python}
\begin{itemize}
 \item Définition d'une nouvelle : "class" classname ":" suite
 \item propriétés : variables avec une valeur par défaut. 
 \item méthodes : fonctions avec le premier argument l'objet lui même par convention 'self'
\end{itemize}
\begin{pythonConsole}
>>> class Canard: 
...    pattes = 2
...    ailes = 2
...    status = "en attente"
...    def mange(self, aliment):
...       self.status = "mange"
...       if aliment in {"grain", "feuille", "fleur", "tige", "racine"}: 
...          print("miam")
...       elif aliment in {"ver", "insecte", "friture"}: 
...          print("j'avais faim")
...       else : 
...          print("non merci")
...    def chante(self): 
...       self.status = "chante"
...       print("couin couin")
...    def espere(self): 
...       self.status = "en attente"
...       return 0 
...
\end{pythonConsole}
\end{frame}
%...............................................................................
%...............................................................................
\begin{frame}[fragile]
\frametitle{Syntaxe}
\framesubtitle{Objets : définition en Python}
\begin{itemize}
 \item \_\_init\_\_ : création lors de l'instanciation. Il est conseillé d'initialiser les attributs dans ce bloc. 
\end{itemize}
\begin{pythonConsole}
>>> class Canard: 
...    def __init__(self): 
...       self.pattes = 2
...       self.ailes = 2
...       self.status = "en attente"
...
\end{pythonConsole}
\end{frame}
%...............................................................................
%...............................................................................
\begin{frame}[fragile]
\frametitle{Syntaxe}
\framesubtitle{Objets : instanciation et usage}
\begin{itemize}
 \item instanciation : objet = Classname()
 \item propriétés obtenues par : objet.property (cf structure en C)
 \item méthodes obtenues par : objet.method(argument)
\end{itemize}
\begin{pythonConsole}
>>> riri = Canard()
>>> riri
<__main__.Canard instance at 0x2c8760>
>>> fifi = Canard()
>>> loulou = Canard()
>>> riri.ailes
2
>>> riri.status
en attente
>>> riri.chante()
couin couin
>>> loulou.mange("grain")
miam
>>> riri.status, loulou.status, fifi.status
('chante', 'mange', 'en attente')
>>> fifi.espere()
0
>>> dir(riri)
['__doc__', '__module__', 'ailes', 'chante', 'espere', 'mange', 'pattes',
'status']
\end{pythonConsole}
\end{frame}
%...............................................................................
%...............................................................................
\begin{frame}[fragile]
\frametitle{Syntaxe}
\framesubtitle{Objets : exemple avec des nombres}
\begin{itemize}
\item En python toutes les variables sont des objets. 
\end{itemize}
\begin{pythonConsole}
>>> a = 123
>>> dir(a)
['__abs__', '__add__', '__and__', '__class__', '__cmp__', '__coerce__',
'__delattr__', '__div__', '__divmod__', '__doc__', '__float__', '__floordiv__',
'__format__', '__getattribute__', '__getnewargs__', '__hash__', '__hex__',
'__index__', '__init__', '__int__', '__invert__', '__long__', '__lshift__',
'__mod__', '__mul__', '__neg__', '__new__', '__nonzero__', '__oct__', '__or__',
'__pos__', '__pow__', '__radd__', '__rand__', '__rdiv__', '__rdivmod__',
'__reduce__', '__reduce_ex__', '__repr__', '__rfloordiv__', '__rlshift__',
'__rmod__', '__rmul__', '__ror__', '__rpow__', '__rrshift__', '__rshift__',
'__rsub__', '__rtruediv__', '__rxor__', '__setattr__', '__sizeof__', '__str__',
'__sub__', '__subclasshook__', '__truediv__', '__trunc__', '__xor__',
'bit_length', 'conjugate', 'denominator', 'imag', 'numerator', 'real']
>>> type(a)
<type 'int'>
>>> a.real
123
>>> a.bit_length()
7
\end{pythonConsole}
\end{frame}
%...............................................................................
%...............................................................................
\begin{frame}[fragile]
\frametitle{Syntaxe}
\framesubtitle{Objets : exemple avec les listes et les tuples}
\begin{pythonConsole}
>>> a = [1, 2, 3, 1, 2]
>>> dir(a)
[..., 'append', 'count', 'extend', 'index', 'insert', 'pop', 
'remove', 'reverse', 'sort']
>>> type(a)
<type 'list'>
>>> a.append(3)
>>> a 
[1, 2, 3, 1, 2, 3]
>>> a.pop()
>>> a 
[1, 2, 3, 1, 2]
\end{pythonConsole}
\begin{pythonConsole}
>>> a = (1, 2, 3, 1, 2)
>>> dir(a)
[..., 'count', 'index']
>>> type(a)
<type 'tuple'>
>>> a.count(1)
2
>>> a.index(1)
0
\end{pythonConsole}
\end{frame}
%...............................................................................
%...............................................................................
\begin{frame}[fragile]
\frametitle{Syntaxe}
\framesubtitle{Objets : exemple avec les chaînes de caractères}
\begin{pythonConsole}
>>> a = "abc.def"
>>> dir(a)
[..., 'capitalize', 'center', 'count', 'decode', 'encode', 'endswith', 
'expandtabs', 'find', 'format', 'index', 'isalnum', 'isalpha', 'isdigit',
'islower', 'isspace', 'istitle', 'isupper', 'join', 'ljust', 'lower', 'lstrip',
'partition', 'replace', 'rfind', 'rindex', 'rjust', 'rpartition', 'rsplit',
'rstrip', 'split', 'splitlines', 'startswith', 'strip', 'swapcase', 'title',
'translate', 'upper', 'zfill']
>>> a.upper()
ABC.DEF
>>> a.index('.')
3
>>> (head, sep, tail) = a.partition('.')
('abc', '.', 'def')
>>> (head, sep, tail) = a.partition('.')
>>> head
'abc'
\end{pythonConsole}
\end{frame}
%...............................................................................
%...............................................................................
\begin{frame}[fragile]
\frametitle{Syntaxe}
\framesubtitle{Fichiers}
\begin{itemize}
 \item un fichier est considéré comme un objet de classe 'file'. 
\end{itemize}
\begin{pythonConsole}
>>> f = file("cuisinierFrancois.txt", 'r')
>>> f
<open file 'cuisinierFrancois.txt', mode 'r' at 0x2a2390>
>>> dir(f)
dir(f)
[..., 'close', 'closed', 'encoding', 'errors', 'fileno', 'flush', 
'isatty', 'mode', 'name', 'newlines', 'next', 'read', 'readinto', 
'readline', 'readlines', 'seek', 'softspace', 'tell', 'truncate', 
'write', 'writelines', 'xreadlines']
>>> f.readlines()
['LE CUISINIER FRANCOIS, \n', '\n', 'ENSEIGNANT LA MANIERE\n', 'de 
bien apprester & assaisonner\n', 'toutes sortes de Viandes grasses
\n', '& maigres, Legumes, Patisseries, \n', '& autres mets qui se 
servent tant\n', 'sur les Tables des Grands que des\n', 'particuli
ers. \n', '\n', '\n']
>>> f.seek(0)
>>> for line in f: 
...     print(line)
...
LE CUISINIER FRANCOIS, 

ENSEIGNANT LA MANIERE
de bien apprester & assaisonner
toutes sortes de Viandes grasses
& maigres, Legumes, Patisseries, 
& autres mets qui se servent tant
sur les Tables des Grands que des
particuliers. 
\end{pythonConsole}
\end{frame}
%...............................................................................
%_______________________________________________________________________________
%_______________________________________________________________________________
\subsection{Portées, modules et paquets}
%...............................................................................
\begin{frame}[fragile]
\frametitle{Module}
\framesubtitle{Exécution dans le shell}
\begin{itemize}
\item Enregistrement du code dans un fichier module1.py
\item Contient des définitions des classes ou de fonctions, des scripts, une documentation (docstring).  
\item Exécution du code dans le shell : python module1.py
\end{itemize}
\begin{python}
def f1(x):
    y = 10 * x
    return y
def f2(x):
    y = 100 * x
    return y
x = 3
y1 = f1(x)
y2 = f2(x)
print("(x, y1, y2) : " + str((x, y1, y2)))
\end{python}

\begin{shell}
~/python/example/module> python module1.py
(x, y1, y2) : (3, 30, 300)
\end{shell}
\end{frame}
%...............................................................................
%...............................................................................
\begin{frame}[fragile]
\frametitle{Module}
\framesubtitle{Importation de code : import}
\begin{itemize}
\item Importation du code dans la console ou dans un autre module : import  
\item Variante en modifiant l'espace de nom (namespace) : import \dots\ as fun
\end{itemize}
\begin{python}
import module1
x = 4
y1 = module1.f1(x)
y2 = module1.f2(x)
print("module -> (x, y1, y2) : " + (x, y1, y2))
\end{python}
\begin{python}
import moduleQuiAUnNomBeaucoupTropLong as myFun
x = 4
y1 = myFun.f1(x)
y2 = myFun.f2(x)
print("module -> (x, y1, y2) : " + (x, y1, y2))
\end{python}
\begin{shell}
(x, y1, y2) : (3, 30, 300)
module -> (x, y1, y2) : (4, 40, 400)
\end{shell}
\end{frame}
%...............................................................................
%...............................................................................
\begin{frame}[fragile]
\frametitle{Module}
\framesubtitle{Importation de code : import}
\begin{itemize}
\item importation spécifique d'une ou plusieurs fonctions : from \dots\ import (f1, f2) ; from \dots\ import (f1 as fun1, f2 as fun2) ; from \dots\ import f1 as fun1, f2 as fun2
\item from \dots\ import *
\end{itemize}
\begin{pythonConsole}
>>> from module1 import f1 
>>> f1(5)
50
\end{pythonConsole}
\begin{pythonConsole}
>>> from module1 import f1 as fois10 
>>> fois10(5)
50
\end{pythonConsole}
\begin{pythonConsole}
>>> from module1 import *
>>> dir()
[..., 'f1', 'f2', 'x', 'y1', 'y2']
\end{pythonConsole}
\begin{pythonConsole}
>>> import module1
>>> dir()
[..., 'module1']
>>> type(module1)
<type 'module'>
>>> dir(module1)
[..., 'f1', 'f2', 'x', 'y1', 'y2']
\end{pythonConsole}
\end{frame}
%...............................................................................

%...............................................................................
\begin{frame}[fragile]
\frametitle{Module}
\framesubtitle{Espace : \_\_name\_\_}
\begin{itemize}
\item attribut système : '\_\_name\_\_'
\end{itemize}
\begin{python}
print("Affichage en import ou en ex£{\color{magenta}\'e}£cution")
print("__name__ : " + str(__name__))

if ("__name__" == "__main__"): 
    print("Que si execut£{\color{magenta}\'e}£ de l'ext£{\color{magenta}\'e}£rieur")
\end{python}
\begin{pythonConsole}
>>> import moduleMain
Affichage en import ou en ex£\'e£cution
__name__ : moduleMain
\end{pythonConsole}
\begin{shell}
~/python/example/module> python moduleMain.py
Affichage en import ou en ex£é£cution
__name__ : __main__
\end{shell}
\end{frame}
%...............................................................................
%...............................................................................
\begin{frame}[fragile]
\frametitle{Paquet}
% \framesubtitle{\_\_init\_\_.py}
\begin{itemize}
 \item Un dossier contenant : 
 \begin{itemize} 
  \item des modules
  \item un fichier '\_\_init\_\_.py', vide ou non.
 \end{itemize}
 \item exemple : packet1/subpacket1/module1.py ; packet1/subpacket2/module2.py 
 \item possibilité de faire des sous-paquet.  
\end{itemize}
\begin{pythonConsole}
>>> from packet1.subpacket1 import module1 as mod1
\end{pythonConsole}
\begin{center}
 \myFig{width=3cm}{./fig/packet}
\end{center}
\end{frame}
%...............................................................................
%...............................................................................
\begin{frame}[fragile]
\frametitle{Portée}
\begin{minipage}[c]{3cm}
 \myFig{width=3cm}{./fig/scope}
\end{minipage}
\begin{minipage}[c]{6cm}
\begin{itemize}
 \item paquet, module, def, class : chaque niveau à sa portée (scope).
 \item Attention à l'espace des noms (namespace) : par exemple, f1 est disponible dans module1 et module2.    
\end{itemize}
\begin{pythonConsole}
>>> import paquet1
>>> x = 10
>>> a1 = paquet1.module1.A
>>> a2 = paquet1.module1.A
>>> a2.f()
>>> resF1 = paquet1.module1.f1(x)
>>> resF2 = paquet1.module2.f2(x)
>>> from paquet1.module1 import * # A EVITER
>>> resF1 = f1(x)
\end{pythonConsole}
\end{minipage}
\end{frame}
%...............................................................................
%_______________________________________________________________________________
%_______________________________________________________________________________
\subsection{Librairies}
%...............................................................................
\begin{frame}
\frametitle{Standard library}
\begin{itemize}
    \item Un ensemble de paquets et modules sont déjà livrés avec Python (Battery included) : math, os, sys, datetime \dots
    \item Avant de recoder quelque chose, vérifier si cela existe dans la documentation : 'The Python Library Reference'. 
\end{itemize}
\end{frame}
%...............................................................................
%_______________________________________________________________________________
%_______________________________________________________________________________
\subsection{Exemple}
%...............................................................................
\begin{frame}[fragile]
\frametitle{Exemple}
\framesubtitle{Math}
\begin{pythonConsole}
>>> import math
>>> dir(math)
['__doc__', '__file__', '__name__', '__package__', 'acos', 'acosh', 
'asin', 'asinh', 'atan', 'atan2', 'atanh', 'ceil', 'copysign', 'cos', 'cosh', 
'degrees', 'e', 'erf', 'erfc', 'exp', 'expm1', 'fabs', 'factorial', 'floor', 
'fmod', 'frexp', 'fsum', 'gamma', 'hypot', 'isinf', 'isnan', 'ldexp', 
'lgamma', 'log', 'log10', 'log1p', 'modf', 'pi', 'pow', 'radians', 'sin', 
'sinh', 'sqrt', 'tan', 'tanh', 'trunc'] 
>>> math.cos(math.pi/3)
0.5000000000000001
\end{pythonConsole}
\end{frame}
%...............................................................................
%...............................................................................
\begin{frame}[fragile]
\frametitle{Exemple}
\framesubtitle{Graphical User Interface / GUI}
\begin{itemize}
\item interface graphique pour Tk (Tkinter), GTK (PyGTK), wxWidgets (wxPython), Qt (PyQt).  
\end{itemize}
\begin{pythonConsole}
>>> import Tkinter
>>> from Tkconstants import *
>>> tk = Tkinter.Tk()
>>> tk.title("My App - Hello World !")
>>> frame = Tkinter.Frame(tk, relief=RIDGE, borderwidth=2)
>>> frame.pack(fill=BOTH,expand=1)
>>> label = Tkinter.Label(frame, text="Hello, World")
>>> label.pack(fill=X, expand=1)
>>> button = Tkinter.Button(frame,text="Exit",command=tk.destroy)
>>> button.pack(side=BOTTOM)
>>> tk.mainloop()
\end{pythonConsole}
\begin{center}
\myFig{width=5cm}{./fig/Tkinter.png}
\end{center}
\end{frame}
%...............................................................................
%...............................................................................
\begin{frame}[fragile]
\frametitle{Exemple}
\framesubtitle{Graphical User Interface / GUI}
\begin{pythonConsole}
>>> import turtle
>>> for i in range(30): 
...     turtle.forward(200)
...     turtle.rt(20)
...     turtle.forward(200)
...     turtle.rt(160)
...     turtle.forward(200)
...     turtle.rt(20)
...     turtle.forward(200)
...     turtle.rt(172)
... 
\end{pythonConsole}
\begin{center}
\myFig{width=5cm}{./fig/turtle.png}
\end{center}
\end{frame}
%...............................................................................
%...............................................................................
\begin{frame}[fragile]
\frametitle{Exemple}
\framesubtitle{Base de données}
\begin{pythonConsole}
>>> import sqlite3
>>> db = sqlite3.connect("myDB.sq3")
>>> cur = db.cursor()
>>> cur = cur.execute("CREATE TABLE membres 
    (nom TEXT, prenom TEXT, institut TEXT)")
>>> cur = cur.execute("INSERT INTO membres(nom,prenom,institut) 
    VALUES('Michel','Olivier','INPG')")
>>> cur = cur.execute("INSERT INTO membres(nom,prenom,institut)
    VALUES('Brossier','Jean-Marc','UJF')")
>>> cur = cur.execute("INSERT INTO membres(nom,prenom,institut)
    VALUES('Amblard','Pierre-Olivier','CNRS')")
>>> db.commit()
>>> cur = cur.execute("SELECT * FROM membres WHERE institut = 'CNRS'")
>>> list(cur)
[(u'Amblard', u'Pierre-Olivier', u'CNRS')]
>>> db.close()
\end{pythonConsole}
\end{frame}
%...............................................................................
%_______________________________________________________________________________
%_______________________________________________________________________________
\subsection{Documentation du code}
%...............................................................................
\begin{frame}[fragile]
\frametitle{Documentation}
\framesubtitle{docstrings}
\begin{itemize}
 \item Commentaires : ligne qui commence par \#
 \item Texte de documentation (Docstring): {\em Une chaîne de caractère qui apparaît comme première expression dans un module, une fonction ou une classe}. 
 \item """ """ recommandé (see PEP257 : \url{http://www.python.org/dev/peps/pep-0257/})
\end{itemize}

\begin{python}
def times(x, y): 
    """
    Multiply x by y
    
    Compute z = x * y

    Input parameters
    x and y, any numbers

    Output parameters
    z 
    
    Copyright 2013 G. Becq, Gipsa-lab, UMR 5216, CNRS.
    """
    return x * y
\end{python}
\end{frame}
%...............................................................................
%...............................................................................
\begin{frame}[fragile]
\frametitle{Documentation}
\framesubtitle{docstrings -- exemple}
\begin{itemize}
 \item Première ligne est un résumé suivi d'une ligne vide. 
 \item Première ligne vide et indentation seront effacées. 
 % \item Parfois notations de type markup liées au package Sphinx. 
\end{itemize}
\begin{python}
# -*- encoding:utf-8 -*-
"""
Module pour tester les docstrings.

Contient deux fonctions et une classe. 
Ne fait rien.

Copyright 2013 G. Becq, Gipsa-lab, UMR 5216, CNRS.
"""

def fun1(): 
    """
    fun1 c'est bien. 
    
    Fonction sans parametres
    Ne fait rien. 
    """
    print("blah blah")
    """
    Ce texte ne sera pas visible
    """
    print("blah blah")
    "Ce texte non plus, les declarations vides non plus"
    1
    print("blah blah")
    # Ca c'est un vrai commentaire delimite par un hash
    return
\end{python}
\end{frame}
%...............................................................................
%...............................................................................
\begin{frame}[fragile]
\frametitle{Documentation}
\framesubtitle{docstrings -- exemple}
\begin{python}
def fun2(): 
    """
    fun2 c'est mieux. 
    
    Sans parametres
    Ne fait rien
    """
    return

class A(): 
    """
    A est une classe. 
   
    C'est une classe vide qui ne contient qu'un docstring.
    """

\end{python}
\end{frame}
%...............................................................................
%...............................................................................
\begin{frame}
\frametitle{Documentation}
\framesubtitle{help}
\begin{minipage}[c]{3cm}
\begin{itemize}
 \item Génération automatique de documentation : help(nomDuModule)
\end{itemize}
\end{minipage}
\hfill%
\begin{minipage}[c]{6cm}
\begin{center}
\myFig{width=6cm}{./fig/help.png}
\end{center}
\end{minipage}
\end{frame}
%...............................................................................
%...............................................................................
\begin{frame}
\frametitle{Documentation}
\framesubtitle{pydoc}
\begin{minipage}[c]{4cm}
\begin{itemize}
\item Dans le shell 'pydoc -g', lance un serveur de documentation accessible via un navigateur. 
\item 'pydoc' génère aussi d'autres sorties. 
\end{itemize}
\end{minipage}
\hfill%
\begin{minipage}[c]{6cm}
\begin{center}
\myFig{width=6cm}{./fig/pydoc.png}
\end{center}
\end{minipage}
\end{frame}
%...............................................................................
%_______________________________________________________________________________
%_______________________________________________________________________________
\subsection{Distribution des paquets}
%...............................................................................
\begin{frame}[fragile]
\frametitle{Packaging}
\begin{itemize}
 \item Repose sur \emph{distutils} de la bibliothèque standard.
 \item Créer un fichier 'setup.py' qui contient des metadonnées. 
\end{itemize}
\begin{python}
from distutils.core import setup
setup(name="distrib1", 
    description="une distribution de d£{\color{magenta}é}£mos", 
    version="1.0", 
    author="Guillaume Becq", 
    author_email="guillaume.becq@gipsa-lab.grenoble-inp.fr", 
    url="http://example.iana.org",
    py£\_£modules=["module1", "module2"], 
    )
\end{python}
\end{frame}
%...............................................................................
%...............................................................................
\begin{frame}[fragile]
\frametitle{Packaging}
\begin{itemize}
 \item Distribution : fichier compressé de type 'zip' ou '.tar.gz' 
 \item Exécuter dans un terminal "python setup.py sdist"
 \item Génère : 
 \begin{itemize}
  \item un dossier 'dist' : contenant la distribution.  
  \item un fichier MANIFEST : liste des fichiers inclus dans le paquet. 
 \end{itemize}
 \item possibilité d'utiliser MANIFEST.in pour dire quels fichiers à inclure.
 \item Faire un README.txt sinon warning.
\end{itemize}
\begin{shell}
~/myPackage> python setup.py sdist
writing manifest file 'MANIFEST'
creating packet1-1.0
making hard links in packet1-1.0...
hard linking setup.py -> packet1-1.0
creating dist
Creating tar archive
removing 'packet1-1.0' (and everything under it)
~/myPackage> cd dist
~/myPackage/dist> ls
packet1-1.0.tar.gz
\end{shell}
\end{frame}
%...............................................................................
%...............................................................................
\begin{frame}[fragile]
\frametitle{Installation des paquets}
\begin{itemize}
\item Décompresser le paquet : il existe une bibliothèque 'tarfile' dans la librairie standard de Python qui permet de décompresser les fichiers 'tar'. 
\item Exécuter dans un terminal : "python setup.py"
\item L'installation se fait dans "PYTHONHOME/site-packages".
\end{itemize}
\begin{shell}
~/tempdir/packageToInstall> python setup.py install
\end{shell}
\end{frame}
%...............................................................................
%...............................................................................
\begin{frame}
\frametitle{Python Package Index}
\frameCC{%
\begin{itemize}
\item Python Package Index : PyPI \url{https://pypi.python.org/pypi}
\item Base de données de paquets disponibles.
\end{itemize}
}{%
\myFig{width=7cm}{./fig/PyPI.png}
}
\end{frame}
%...............................................................................
%...............................................................................
\begin{frame}
\frametitle{Installation des paquets}
\begin{itemize}
\item A la base {\bfseries distutils}, ne gère pas les dépendances : python setup.py  
\item {\bfseries setuptools} : introduit la commande 'easy\_install' qui opère sur de fichers 'egg': easy\_install example.egg
\item setuptools a généré {\bfseries distribute} qui gère les dépendances. 
\item {\bfseries pip} : remplace 'easy\_install' : "pip install paquetExample" 
\end{itemize}
\end{frame}
%...............................................................................
%_______________________________________________________________________________
%===============================================================================
