%===============================================================================
\section{Distributions et Environnements de travail}
%===============================================================================
%_______________________________________________________________________________
\subsection{Python Software Foundation}
%...............................................................................
\begin{frame}
\frametitle{Python.org}
\framesubtitle{Official}
\frameCC{%
\begin{itemize}
 \item Python 2.7 ou 3.x téléchargeable sur \urlPython.
 \item Livré uniquemenent avec la bibliothèque standard. 
 \item Inclus l'interpréteur Python natif accessible à partir de l'environnement système. 
 \item Parfait pour tester des petits bouts de codes. 
\end{itemize}
}%
{\myFig{width=7cm}{./fig/interpreter.png}}
\end{frame}
%...............................................................................
%...............................................................................
\begin{frame}
\frametitle{Python IDLE}
\framesubtitle{official}
\frameCC{%
\begin{itemize}
 \item Python livré avec un Integrated DeveLopment Environment (IDLE) : 
 \begin{itemize}
 \footnotesize
  \item Une console Python : coloration automatique, autocomplétion \dots 
  \item Un éditeur de texte : indentation automatique, coloration syntaxique, debuggueur, \dots
 \end{itemize}
\end{itemize}
}%
{\myFig{width=7cm}{./fig/Idle.png}}
\end{frame}
%_______________________________________________________________________________
%_______________________________________________________________________________
\subsection{Enthought}
\begin{frame}
\frametitle{Enthought Canopy}
\begin{itemize}
 \item \myFig{height=0.5cm}{./fig/enthought-logo.png}\ : \url{https://www.enthought.com/products/canopy/} 
 \item Contient Python et +100 librairies orientées applications scientifiques. 
 \item Canopy express \em{free license for all users}.
\end{itemize}
\end{frame}
%_______________________________________________________________________________
%_______________________________________________________________________________
\subsection{Enthought}
\begin{frame}
\frametitle{Enthought Canopy}
\begin{itemize}
\footnotesize
 \item \myFig{height=0.5cm}{./fig/enthought-logo.png}\ : \url{https://www.enthought.com/products/canopy/} 
 \item \em{Easy installation and update}.
\end{itemize}
\centering\myFig{width=7cm}{./fig/enthought_canopy_manager.png}
\end{frame}
%_______________________________________________________________________________
%_______________________________________________________________________________
\subsection{Enthought}
\begin{frame}
\frametitle{Enthought Canopy}
\begin{itemize}
\footnotesize
 \item \myFig{height=0.5cm}{./fig/enthought-logo.png}\ : \url{https://www.enthought.com/products/canopy/} 
 \item + Jupyter notebook + Canopy editor. 
\end{itemize}
\centering\myFig{width=7cm}{./fig/enthought_canopy_editor.png}
\end{frame}
%_______________________________________________________________________________
%_______________________________________________________________________________
\subsection{Anaconda}
%_______________________________________________________________________________
%_______________________________________________________________________________
\begin{frame}
\frametitle{Anaconda}
\framesubtitle{}
\begin{itemize}
 \item \myFig{height=0.5cm}{./fig/anaconda_logo_web.png}\, \url{https://www.anaconda.com}
\end{itemize}
\only<1>{\centering\myFig{width=8cm}{./fig/anaconda_distribution.png}}
\only<2>{\centering\myFig{width=8cm}{./fig/anaconda_launcher.png}}
\end{frame}
%_______________________________________________________________________________
%_______________________________________________________________________________
\begin{frame}
\frametitle{Spyder}
\framesubtitle{}
\frameCC{%
\begin{itemize}
\footnotesize
\item Spyder (multiplateforme) : environnement de type Matlab pour Python. 
\item \url{https://github.com/spyder-ide/spyder}
\end{itemize}
}
{\myFig{width=8cm}{./fig/Spyder.png}}
\end{frame}
%_______________________________________________________________________________
%_______________________________________________________________________________
\subsection{Autres}
%_______________________________________________________________________________
%_______________________________________________________________________________
\begin{frame}
\frametitle{Autres distributions}
\begin{itemize}
\item SageMath \myFig{height=0.5cm}{./fig/sage_logo.png} \, \url{http://www.sagemath.org}
\item PythonXY \myFig{height=0.5cm}{./fig/PythonXYLogo.png}\ \url{https://code.google.com/p/pythonxy/}  
\item ? 
\end{itemize}
\end{frame}
%_______________________________________________________________________________
%_______________________________________________________________________________
\begin{frame}
\frametitle{Editeurs de texte et Integrated Development Environment (IDE)}
\framesubtitle{}
\frameCC{%
\begin{itemize}
\small
\item Tout éditeur de texte avec coloration syntaxique : Emacs, Vim, jEdit, gedit, Textpad, Eclipse, Atom, Visual Studio... 
\end{itemize}
}
{\myFig{width=7cm}{./fig/jEdit.png}}
\end{frame}
%_______________________________________________________________________________
%===============================================================================