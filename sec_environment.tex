%===============================================================================
\section{Distributions et Environnements de travail}
%===============================================================================
%_______________________________________________________________________________
\subsection{Python Software Foundation}
%...............................................................................
\begin{frame}
\frametitle{Python.org}
\framesubtitle{Official}
\frameCC{%
\begin{itemize}
 \item Python 2.7 ou 3.x téléchargeable sur \urlPython.
 \item Livré uniquemenent avec la bibliothèque standard. 
 \item Inclus l'interpréteur Python natif accessible à partir de l'environnement système. 
 \item Parfait pour tester des petits bouts de codes. 
\end{itemize}
}%
{\myFig{width=7cm}{./fig/interpreter.png}}
\end{frame}
%...............................................................................
%...............................................................................
\begin{frame}
\frametitle{Python IDLE}
\framesubtitle{official}
\frameCC{%
\begin{itemize}
 \item Python livré avec un Integrated DeveLopment Environment (IDLE) : 
 \begin{itemize}
 \footnotesize
  \item Une console Python : coloration automatique, autocomplétion \dots 
  \item Un éditeur de texte : indentation automatique, coloration syntaxique, debuggueur, \dots
 \end{itemize}
\end{itemize}
}%
{\myFig{width=8cm}{./fig/Idle.png}}
\end{frame}
%...............................................................................
%_______________________________________________________________________________
%_______________________________________________________________________________
\subsection{IPython}
%...............................................................................
\begin{frame}
\frametitle{IPython}
\framesubtitle{Intro}
\frameCC{%
\begin{itemize}
 \item \myFig{height=0.5cm}{./fig/IPy_header.png}\ IPy: \url{http://ipython.org}
 \item Console interactive accessible via le shell : coloration syntaxique, fonctions magiques, mémorisation des commandes, débuggueur, profileur, calculs parallèles \dots  
 \item Plusieurs options cf. 'man ipython' ou 'ipython -help'. 
\end{itemize}}
{\myFig{width=7cm}{./fig/IPython.png}}
\end{frame}
%...............................................................................
%...............................................................................
\begin{frame}
\frametitle{IPython}
\framesubtitle{Qtconsole}
\frameCC{%
\begin{itemize}
 \item Commande dans le shell : ipython qtconsole  
 \item Environnement graphique Qt qui permet de tracer des figures via matplotlib ou pylab.
 \item Commande directe : ipython qtconsole --pylab
\end{itemize}}
{\myFig{width=9cm}{./fig/IPythonQTConsole.png}}
\end{frame}
%...............................................................................
%...............................................................................
\begin{frame}
\frametitle{IPython}
\framesubtitle{notebook}
\frameCC{%
\begin{itemize}
 \small
 \item Commande dans le shell : ipython notebook  
 \item Editeur dans le navigateur HTML, \emph{Web-based interactive computational environment}. 
\end{itemize}}
{\myFig{width=9.5cm}{./fig/IPythonNotebook.png}}
\end{frame}
%...............................................................................
%...............................................................................
\begin{frame}
\frametitle{IPython}
\framesubtitle{notebook}
\frameCC{%
\begin{itemize}
 \small
 \item Cellules de codes ou de documentation : \emph{Cell Mode} à la Matlab, \emph{Document-Based Workflow} à la Mathematica. 
 \item Balisage du texte en LaTeX, HTML ou Markdown. 
\end{itemize}}
{\myFig{width=9.5cm}{./fig/IPythonNotebook.png}}
\end{frame}
%...............................................................................
%...............................................................................
\begin{frame}
\frametitle{IPython}
\framesubtitle{notebook}
\frameCC{%
\begin{itemize}
 \item Commande pour importer pylab et graphique dans l'interface html : 'ipython notebook' '\%pylab inline'
 \item Génération de rapports dans le menu 'notebook > action > print' puis impression en pdf pour le navigateur HTML ou commande 'convert' 
 \item ou via nbconvert en HTML : 'ipython nbconvert notebook.ipynb'
 \item ou via nbconvert en PDF via LaTeX : 'ipython nbconvert notebook.ipynb --to latex --post PDF'
\end{itemize}}
{\myFig{width=6cm}{./fig/IPythonNotebookInline.png}}
\end{frame}
%_______________________________________________________________________________
%_______________________________________________________________________________
\subsection{Enthought}
\begin{frame}
\frametitle{Enthought Canopy}
\begin{itemize}
 \item \myFig{height=0.5cm}{./fig/enthought-logo.png}\ : \url{https://www.enthought.com/products/canopy/}
 \item Contient Python et +100 librairies orientées applications scientifiques. 
 \item Multi-plateformes, \emph{Easy installation and update}.
 \item Gratuit pour les étudiants et les universitaires. 
 \item Anciennement Enthought Python Distribution EPD. 
 \item QtConsole, iPython. 
\end{itemize}
\end{frame}
%_______________________________________________________________________________
%_______________________________________________________________________________
\begin{frame}
\frametitle{Enthought Canopy}
\begin{center}
 \myFig{width=10cm}{./fig/enthought-canopy.jpg}
\end{center}
\end{frame}
%_______________________________________________________________________________
%_______________________________________________________________________________
\subsection{Python(x,y)}
%_______________________________________________________________________________
%_______________________________________________________________________________
\begin{frame}
\frametitle{Python(x,y)}
\frameCC{%
\begin{itemize}
\item \myFig{height=0.5cm}{./fig/PythonXYLogo.png}\ Python(x,y) : \url{https://code.google.com/p/pythonxy/}
\item \emph{\footnotesize Python(x,y) is a free scientific and engineering development software for numerical computations, data analysis and data visualization based on \emph{Python} programming language, \emph{Qt} graphical user interfaces and \emph{Spyder} interactive scientific development environment.}
\item Un grand nombre de librairies scientifiques, entre autres. 
\item Interface avec Eclipse (IDE principalement pour Java mais aussi Python). 
\end{itemize}
}
{\myFig{width=7cm}{./fig/pythonxy_home.png} \\ {\tiny source web}}
\end{frame}
%...............................................................................
%...............................................................................
\begin{frame}
\frametitle{Python(x,y)}
\framesubtitle{Spyder}
\frameCC{%
\begin{itemize}
\item Spyder (multiplateforme) : environnement de type Matlab pour Python. 
\end{itemize}
}
{\myFig{width=8cm}{./fig/Spyder.png} \\ {\tiny source web}}
\end{frame}
%_______________________________________________________________________________
\subsection{Anaconda}
%_______________________________________________________________________________
\begin{frame}
\frametitle{Anaconda}
\framesubtitle{}
\frameCC{%
\begin{itemize}
 \item \myFig{height=0.5cm}{./fig/continuum_analytics_logo.png} \, \url{http://continuum.io/downloads}
 \item \myFig{height=0.5cm}{./fig/anaconda_logo_web.png} propose Spyder. 
\end{itemize}
}
{\myFig{width=8cm}{./fig/anaconda_launcher.png}
}
\end{frame}
%_______________________________________________________________________________
%_______________________________________________________________________________
\subsection{SageMath}
\begin{frame}
\frametitle{SageMath}
\framesubtitle{}
\begin{itemize}
 \item \myFig{height=0.5cm}{./fig/sage_logo.png} \, \url{http://www.sagemath.org}
\end{itemize}
\myFigCentered{width=9cm}{./fig/SageMath_home.png}
\end{frame}
%_______________________________________________________________________________
%_______________________________________________________________________________
\begin{frame}
\frametitle{Autres Integrated Development Environment}
\framesubtitle{}
\frameCC{%
\begin{itemize}
\item Tout éditeur de texte avec coloration syntaxique : Emacs, Vim, jEdit, gedit, Textpad... 
\end{itemize}
}
{\myFig{width=9cm}{./fig/jEdit.png}}
\end{frame}
%_______________________________________________________________________________
%_______________________________________________________________________________

%_______________________________________________________________________________
%_______________________________________________________________________________
%===============================================================================
